% Страница 17 - разделы 2.5 и 2.6

\section{Идея о гравитации}
\label{sec:gravitation-idea}

В этом отношении вводится краткое рассмотрение гравитации. Если мы принимаем
справедливость уравнений Максвелла в их переданном смысле также и для
гравитации, то простое объяснение распространения гравитационных полей
движущимися массами и инвариантность законов небесной механики, основанная
на этом распределении, применяется к специальной теории относительности.
Поскольку относительные скорости небесных тел в пределах нашего диапазона
наблюдения лежат в порядке величины 1/10\,000 от скорости света,
гравитационные магнитные поля были просто настолько слабыми, что они были
неизмеримыми. Разумеется, должно рассматриваться небольшое затухание
планетарных движений. Автор был бы весьма благодарен за критическое
рассмотрение этих мыслей физиками.

\section{Дифференциальные уравнения и разностные уравнения, дискретизация}
\label{sec:digitalization}

Если дифференциальные уравнения выражены в форме «вывода» согласно теории
автоматов, то они могут быть смоделированы технической моделью (автоматом)
и решены. Сам по себе аналоговый компьютер является идеальным автоматом.
Он работает, по крайней мере в теории, с непрерывными значениями и работает
постоянно; другими словами, мы имеем непрерывный поток состояний, последнее
из которых всегда определяется тем, что ему предшествует. На практике
аналоговые компьютеры используются главным образом для вычисления
дифференциальных уравнений. Тем не менее, существует довольно узкий предел
возможностей аналогового компьютера. Для уравнений в частных производных
аналогичные технические модели доступны только при особых обстоятельствах.

Решение дифференциальных уравнений с помощью цифрового автомата немедленно
осложняется ранее упомянутыми трудностями: дифференциальные уравнения
оперируют непрерывными значениями и бесконечными плотностями поля. Цифровые
инструменты оперируют прерывистыми значениями. Бесконечная плотность поля
потребовала бы бесконечной ёмкости памяти и бесконечного времени вычисления.
Поэтому необходимо достигать компромиссов в обоих отношениях.

Обычно переходят от дифференциальных уравнений к разностным уравнениям,
когда ищутся численные решения. В этом процессе полученные значения всё ещё
рассматриваются как непрерывные. Фактически, переход от дифференциальных
уравнений к разностным уравнениям включает два граничных перехода:
(1) $\Delta x \to dx$, и (2) увеличение числа разрядов включённых величин.

Первый граничный переход постоянно ведёт к предельному значению, которое
предвосхищает второй переход; другими словами, построение разностных
