% Страница 52 - Глава 4: Общие рассмотрения

\section{О детерминизме и причинности}
\label{sec:determinism-causality}

Вопрос о детерминизме и причинности тесно связан с наблюдениями из теории информации и теории автоматов. Выражение <<причинность>> строго не применяется в литературе. В дальнейшем оно всегда используется для обозначения того, что обычно называют <<детерминизмом>>, а именно определение следующего состояния замкнутой системы как функции предыдущего состояния. Всю вселенную можно рассматривать как замкнутую систему, в той мере, в какой учитываются необходимые следствия этого допущения.

Теория автоматов оперирует концепцией состояния автомата. Конечные автоматы могут иметь ограниченное число состояний. При отсутствии входного сигнала результирующее состояние вытекает из предыдущего из-за алгоритма, лежащего в основе автомата. Поскольку теория автоматов оперирует абстрактными понятиями, переход из одного состояния в другое происходит в теории без промежуточных шагов. Теория автоматов не ставит вопрос о том, как именно этот переход происходит в функционирующем автомате. Она занимается исключительно тем фактом, что, например, триггер переходит из одного состояния в другое в течение определённого времени — периода импульса. Технологический анализ процесса переключения, который возможен, выходит за пределы рассмотрения теории автоматов, если только он не относится к пониманию таких деталей.

Некоторые физики, например Артур Марх, полагают, что прямой переход атома из одного устойчивого состояния в другое трудно согласовать с принципом причинности. Он понимает идею причинности таким образом, что переход из одной замкнутой системы в другую требует непрерывного процесса. Это толкование едва ли может устоять перед теоретико-автоматным рассмотрением физических процессов. Нельзя предположить, что эта идея основана на действительности. Мышление в целых числах и в дискретных состояниях требует мыслительного процесса с неконтинуальными переходами, в котором закон причинности формулируется в алгоритмах. Работа с дискретными состояниями и квантификацией как таковой не обязательно требует отказа от причинного способа наблюдения.
