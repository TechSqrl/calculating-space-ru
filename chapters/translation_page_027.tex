% Страница 27 - Конрад Цузе "Вычисляющее пространство"

Без знания точной пространственной структуры можно только определить, что в нашем примере возможны два фундаментальных случая взаимодействия частиц, для каждого из которых вероятность возникновения равна 1/2.

\begin{figure}[htbp]
\centering
\includegraphics[width=0.6\textwidth]{images/page_027_img_01.png}
\caption{Сводка восьми возможных случаев взаимодействия цифровых частиц}
\label{fig:21}
\end{figure}

\begin{figure}[htbp]
\centering
\includegraphics[width=0.6\textwidth]{images/page_027_img_02.png}
\caption{Схематические идеализированные траектории частиц для двух различных паттернов взаимодействия a и b}
\label{fig:22}
\end{figure}

На рис. 21 показана сводка восьми возможных случаев взаимодействия частиц; рис. 22 представляет схематические, идеализированные траектории частиц для двух различных паттернов взаимодействия a и b. Необходимо явно подчеркнуть, что траектории являются идеализированными траекториями частиц. В действительности наша модель представляет не непрерывное движение, а процесс пошагового продвижения.

Интересно отметить, что в нелинейном вычислительном правиле (рис. 20) изолированная точка давления приводит к испусканию двух частиц. Установление предельных значений очевидно ограничивает свободные процессы суперпозиции. В случае неограниченных значений частицы, соответствующие рис. 15, также теоретически суперпозиционируемы. Это означает, что мы можем построить гору давления любой высоты с её сопутствующим распределением скорости, которое удовлетворяет правилу пошагового расширения; то есть, которое остаётся стабильным. Эти стабильные "крупные" частицы всегда делимы на элементарные частицы. Это уже не верно, когда применяется правило, соответствующее рис. 20 (система вычисления с нелинейным правилом).

Наша исходная позиция, в которой мы выбрали коэффициент 1 относительно значения $\Delta$, соответствует очень жёсткой среде в назначенном физическом паттерне газонаполненного цилиндра. Более гибкая ситуация получается, когда коэффициент уменьшается. В этом случае в более точных расчётах возникают нецелые числа. Если мы хотим продолжить работу с целыми числами или вводить только незначительные градации, необходимо вводить округление вверх и округление вниз.
