% Перевод страницы 66 из "Rechnender Raum" Конрада Цузе
% Послесловие: страница 66

Пока он был в Hinterstein, он написал трактат, озаглавленный <<Freedom and Causality in the Light of the Computing Machine>> (<<Свобода и Причинность в свете вычислительной машины>>). В своей автобиографии он пишет: <<Я думаю, что большинство исследователей, вовлечённых в развитие компьютера, в какой-то момент своей жизни так или иначе рассматривали вопрос об отношении между свободной волей человека и причинностью>>. Это должно было стать основным импульсом для работы, которая привела к переводу, представленному в этом томе:

\begin{quote}
<<Рассматривая причинность, мне внезапно пришло в голову, что вселенная могла бы быть представлена как гигантская вычислительная машина. Я имел в виду релейный калькулятор: релейные калькуляторы содержат цепочки реле. Когда реле срабатывает, импульс распространяется через всю цепь. Мне в голову пришла мысль, что это также должна быть форма распространения кванта света. Мысль прочно осела; на протяжении многих лет я развивал её в концепцию Rechnender Raum, или <<вычисляющей вселенной>>. Однако потребовалось ещё тридцать лет, прежде чем я смог правильно сформулировать эту идею.>>
\end{quote}

В 1967 году Цузе предположил, что сама вселенная работает на клеточном автомате или подобной вычислительной структуре, метафизической позиции, известной сегодня как цифровая физика, предметом которой занимался сам Ed Fredkin до того, как познакомился с работами Цузе. Возбужденный открытием этой работы, Fredkin пригласил Цузе в Cambridge, MA. Перевод Rechnender Raum, воспроизведённый здесь, из немецкой (опубликованной) версии идей Цузе, был фактически заказан во время работы Ed Fredkin в качестве директора Project MAC\textsuperscript{10} в MIT (лаборатории AI, которая была предшественницей нынешних лабораторий AI в MIT).

Более чем через двадцать лет после своего Rechnender Raum в автобиографии Цузе он написал:

\begin{quote}
<<В конечном анализе, концепция вычисляющей вселенной требует переосмысления идей, для которых физики ещё не готовы. Однако ясно, что прежние концепции достигли пределов своих возможностей; но никто не осмеливается перейти на принципиально новый путь. Однако при квантизации уже были предприняты предварительные шаги в направлении цифровизации физики; но только несколько физиков попытались думать вдоль линий этих новых категорий компьютерной науки. [\ldots] Это было иллюстрировано совершенно ясно на конференции по Physics of Computation, проведённой 6--8 мая 1981 года [в MIT]. Что было типично на этой конференции, так это то, что, хотя связь между
\end{quote}

{\footnotesize
\noindent
\textsuperscript{10}Ed Fredkin также является автором, внёсшим вклад в \textit{A Computable Universe: Understanding \& Exploring Nature as Computation}.
}
