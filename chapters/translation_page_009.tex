% Страница 9 - окончание раздела 2.1 и начало раздела 2.2

быть представлен, следовательно, машиной, которая работает независимо после
запуска. Его состояния следуют линейно в последовательности, как только запущена
начальная комбинация, и операционный процесс не может быть подвержен внешнему
влиянию из-за отсутствия входов.

Клеточный автомат представляет особую форму автомата, построенного из
взаимосвязанных, периодически повторяющихся ячеек. Этот тип автомата имеет
особое значение для последующих наблюдений. Позже он будет обсуждаться более
подробно.

Под термином «автоматно-теоретический способ мышления» мы понимаем способ
наблюдения, согласно которому техническая, математическая или физическая
модель рассматривается с точки зрения последовательности состояний, которые
следуют друг за другом согласно предопределённым правилам.

\section{О компьютерах}
\label{sec:about-computers}

Теория автоматов может использоваться как абстрактная математическая система,
однако эти мыслительные структуры также могут быть соотнесены с техническими
моделями, и аналогично теория автоматов может использоваться для описания
автоматов, особенно тех, которые подходят для обработки информации. В текущем
расширенном употреблении термин «вычислять» идентичен «обработке информации».
По аналогии термины «компьютер» и «машина обработки информации» могут
рассматриваться как идентичные.

Мы различаем два класса компьютеров: аналоговые компьютеры и цифровые
компьютеры. В аналоговом компьютере шаги вычисления выполняются в «аналоговой»
модели. Величины, представляющие численные значения, теоретически представлены
через непрерывные физические величины, такие как положения механических частей
(угол кручения), напряжение, скорости и тому подобное. Машина работает по
существу бесконечно. Представленные значения очевидно лежат ниже определённых
технических пределов. Они устанавливаются максимальными значениями и точностью
системы. Максимальные значения задаются чётко определённым верхним пределом,
который соответствует техническим ограничениям системы. Напротив, точность не
имеет чётко определённой величины, поскольку она зависит от изменений и от
внешних влияний (температура, влажность, присутствие возмущающих полей и т.д.).
Одним из хорошо известных аналоговых компьютеров является логарифмическая
линейка. На Рис. 5 показан механический суммирующий механизм в форме рычага,
который может быть заменён вращающим механизмом с шестернями, как на Рис. 6.
Этот механизм известен в технике под неподходящим термином «дифференциальный
механизм» и используется в задней оси каждого автомобиля.

Типичный конструктивный элемент аналоговых машин представлен интеграционным
механизмом, показанным на Рис. 7. Он работает с фрикционным диском A