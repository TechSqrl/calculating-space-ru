% Страница 42: Продолжение раздела 4.1 - Клеточные автоматы
% Конрад Цузе "Rechnender Raum"

определить возможное дискретное направление распространения фотонов, если мы предположим точность такого изменения направления (в круговой мере) того же порядка величины, что и то, что мы способны различить между частотами, а именно $10^{-12}$ (эффект Мёссбауэра).

Результаты такого рода можно ожидать только в очень высоких энергетических диапазонах, когда длина волны и период длины подходят к постоянной решетки. Только сегодня мы имеем способность проводить такие эксперименты. Автор должен оставить физикам решение, могут ли эти явления быть наблюдаемы с помощью имеющихся в настоящее время экспериментальных техник и в каких пределах.

\noindent\textbf{(b)} Искривленные объемы, как они предполагаются общей теорией относительности, трудно представить со структурой решетки пространства.

Боппа выбрал способ, предполагая декартово пространство, в котором три пространственные координаты каждая сходятся на себе. Это можно себе представить в двумерном пространстве, предположив тор.

Конечно, существуют многочисленные возможные отклонения от этих последствий. Весь предмет еще слишком молод для того, чтобы можно было сделать окончательные положительные или отрицательные выводы. Следующие возможности можно упомянуть:

\noindent\textbf{($\alpha$)} Предположение о фиксированных схемах в форме клеточных автоматов не единственная логическая возможность для определения логических связей между дискретными значениями в пространстве. Если мы введем изменение схем как функцию результатов предыдущего процесса, переменные схемы могут быть правильно разработаны.

\noindent\textbf{($\beta$)} Концепция растущего автомата тесно связана с правильной вариабельностью схем.

Обе возможности требуют вначале очень хорошо подготовленной теории. Так как теория автоматов является молодой областью, возможности которой ни в коем случае не исчерпаны, мы можем ожидать дальнейших развитий в рассматриваемом направлении.

\noindent\textbf{($\gamma$)} Предположение о решетке неявно предполагает наличие инерциальной системы, что противоречиво со строгой интерпретацией теории относительности. Это будет рассмотрено более подробно.

В этом свете использование ортогональной сети является наиболее удобным способом начала исследований. Результаты, полученные таким образом, будут, конечно, столь же справедливы, когда со временем теория автоматов даст новые методы для использования.
