% Страница 56 - Глава 4: Общие рассмотрения

\subsection{О вероятности}

Проблема детерминизма в современной физике тесно связана с законами вероятности. Наблюдение из теории автоматов может быть вставлено здесь. Конечно, возможно строить математические системы, такие как матричная механика и волновая механика, в которых значения вероятности играют значительную роль. Теоретик автоматов может ввести идею вероятности в свои теории и может установить последовательное состояние, зависящее от значений вероятности. До этого момента процесс является простой математической игрой на бумаге. Это становится критичным, когда мы пытаемся построить готовые формы таких механизмов, которые работают в соответствии с законами вероятности. Такие расчёты выполнялись в наших вычислительных автоматах с значительным успехом уже некоторое время (метод Монте-Карло). Элемент случайности вводится в расчёт в виде <<значений случайности>>. Генерация этих значений случайности — решающая проблема. Есть два способа это осуществить.

(a) Значения генерируются путём моделирования метода костей и тех типов числовых рядов, в которых не существует никакой зависимости между числами. Такой числовой ряд может быть разработан из расчёта иррациональных чисел (например, π). В действительности, этот процесс строго детерминирован. Тем не менее, мы говорим о значениях псевдослучайности. Этот процесс совершенно достаточен, когда правило генерации для таких значений случайности тщательно выбрано.

(b) Механизм берётся из природы, который либо настолько сложен, что его нельзя показать регулярным, либо о котором можно сказать, что в соответствии с действительными законами физики он предоставляет <<реальные>> значения вероятности. Механизм костей принадлежит к первому сорту, где причинные правила играют роль, но в случае достаточно тщательно построенной кости можно показать равную вероятность для каждого случая. То же самое верно для всех игр случая (рулетка и т. д.). В другом случае мы полагаемся на тот факт, что, например, радиоактивность определённого материала подчиняется строгим законам вероятности. Значительно ли, что вероятностный процесс в действительности детерминирован в этих атомах, не имеет значения, так как опыт показывает, что в любом случае законы вероятности могут быть предположены без ведения к неправильным результатам. В этом случае вычисляющий автомат рассматривает значения вероятности в известной степени как внешние входные значения. Остаётся верным, однако, что реальные значения вероятности едва ли возможны в технических автоматах.
