% Страница 30 - Конрад Цузе "Вычисляющее пространство"

\begin{wrapfigure}{r}{0.30\textwidth}
    \centering
    \includegraphics[width=0.28\textwidth]{images/page_030_img_01.png}
    \caption{Процесс взаимодействия двух цифровых частиц (Рис.~27)}
    \label{fig:27}
\end{wrapfigure}

\begin{wrapfigure}{l}{0.30\textwidth}
    \centering
    \includegraphics[width=0.28\textwidth]{images/page_030_img_02.png}
    \caption{Идеализированные траектории частиц для взаимодействия по схеме (Рис.~28)}
    \label{fig:28}
\end{wrapfigure}

\begin{wrapfigure}{r}{0.30\textwidth}
    \centering
    \includegraphics[width=0.28\textwidth]{images/page_030_img_03.png}
    \caption{Блок-диаграмма для вычисляющего пространства}
    \label{fig:29}
\end{wrapfigure}

На рис. 29 показана блок-диаграмма для вычисляющего пространства, соответствующего ранее введённому правилу расчётов. Квадраты $v$ и $p$ представляют регистры, к которым могут быть добавлены числа. Сдвигающие части системы, которые служат для выполнения вычитания, представлены кругами, отмеченными $\Delta$. Вертикальная линия на выходе членов $\Delta$ означает отрицание. Блок-диаграмма может, конечно, быть подразделена на её отдельные сдвигающие элементы. Используемые в настоящее время символы сводят сдвиг к его отдельным элементам, которые соответствуют основным операциям булевой алгебры (конъюнкция, дизъюнкция и отрицание). Используемые здесь элементы информации с трёхзначной логикой должны были быть преобразованы в двоичные элементы посредством двух булевых переменных (2 бита). Из 4 возможных комбинаций этих двух значений используются только три. По этой причине более подробное представление опущено. Чтобы сделать блок-диаграмму на рис. 29 работоспособной, необходима чистая импульсная работа. Поэтому импульсные такты представлены на рис. 29 буквами I и II. При этом предполагается, что члены чистого сложения работают без временной задержки для построения значений $\Delta$, в то время как регистры передают свою информацию дальше только с добавлением следующего импульса. Эта импульсная работа соответствует тонкой структуре временного измерения.

\section{Двумерные системы}
\label{sec:2d-systems}

Рассмотрим кратко двумерную систему. Наиболее простая структура --- это сетка, соответствующая ортогональной системе координат. Система обладает двумя определёнными осями, которые входят даже в простое распространение импульса. Начнём с простого правила, где каждая точка сетки может иметь состояния 0 и 1. В каждом временном интервале такая 1 передаётся каждой соседней точке сетки. Комбинация импульсов, возникающих из различных соседних точек, выполняется в соответствии с правилом дизъюнкции. Если состояние точки сетки $(x, y)$ равно $\varphi_{x,y}$, мы получаем следующее уравнение:
