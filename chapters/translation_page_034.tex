% Страница 34 - Конрад Цузе "Вычисляющее пространство"

\begin{wrapfigure}{r}{0.40\textwidth}
    \centering
    \includegraphics[width=0.38\textwidth]{images/page_034_img_01.png}
    \caption{Ортогональная сетчатая структура}
    \label{fig:38}
\end{wrapfigure}

Теперь мы устанавливаем, что две стрелки в действительности распространяются вперёд в своих соответствующих направлениях в сторону точек B и C, и в точках B и C они обмениваются направлением. Мы получаем таким образом стабильную частицу с периодом $2\Delta t$, которая распространяется диагонально вперёд.

Интересно отметить, что из этого правила возникают карманы, которые закреплены на 4 соседних точках сетки; они имеют период $2\Delta t$. Также возможен двойной стабильный карман с периодом $\Delta t$. Как видно из рисунков и логического развития, которое можно заполнить из примеров на предыдущих страницах, из такого правила может быть создана целая серия частиц. Понимание всех деталей этого правила и его многочисленных следствий требует углубленного анализа, который здесь не может быть полностью проведён.

Однако можно видеть, что цифровые частицы, которые могут быть построены из таких правил, имеют много свойств, которые соответствуют некоторым свойствам физических частиц. Чтобы полностью понять этот важный факт, необходимо ещё много исследований.
