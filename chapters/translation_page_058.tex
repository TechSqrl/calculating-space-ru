% Страница 58 - Глава 4: Общие рассмотрения

Гораздо более рациональный метод предлагается принципом значений мест. Это не приводит к идее построить вычисляющие автоматы в соответствии с принципом Рис.~69. На Рис.~70 показано идеальное устройство суммирующей машины, состоящей из соседних ячеек, среди которых виден иерархический порядок. Отдельные ячейки координированы с числами разного значения. Это отражается в односторонней конструкции процесса передачи u_0 -- u_6.

На Рис.~71 показана передача этого мыслительного процесса линейному клеточному автомату. Каждая ячейка объединена с полной суммирующей машиной. Каждая ячейка C_i подразделяется на отдельные этапы сложения A_{0...5}. При конструировании такой системы сдвига необходимо помнить, что передачи между уровнями внутри ячейки должны быть координированы по времени с передачей информации между отдельными ячейками.

\begin{figure}[htbp]
  \centering
  \includegraphics[width=0.6\textwidth]{images/page_058_img_01.png}
  \caption{Устройство суммирующей машины с иерархическим порядком}
  \label{fig:70}
\end{figure}

\begin{figure}[htbp]
  \centering
  \includegraphics[width=0.6\textwidth]{images/page_058_img_02.png}
  \caption{Передача принципа к линейному клеточному автомату}
  \label{fig:71}
\end{figure}


Этот принцип относительно легко применить на практике для одномерных и двумерных клеточных автоматов. Теоретически его можно применять к трёхмерным и более высокомерным автоматам без каких-либо изменений. В дополнение к измерениям, которые соответствуют топологическому устройству соседних ячеек (пространственное измерение), существует также измерение уровня. Это воображаемо только в трёхмерном пространстве и должно быть конструктивно встроено (спроецировано) в трёхмерное пространство.

Можно задать дальнейший вопрос: может ли в симметрично построенном клеточном автомате быть введён иерархический порядок путём способа занятости? На Рис.~72 демонстрируется принцип. Отдельные ячейки могут содержать, например, отдельные этапы сложения и не могут принимать многозначные числа. Они распределены между несколькими соседними ячейками в соответствии с принципом значения мест. Трудность возникает в том, что такое устройство имеет природу занятости. Если концепция применяется к многомерному автомату, легко видно, что возникают крупные осложнения.
