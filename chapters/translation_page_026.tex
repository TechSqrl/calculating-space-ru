% Страница 26 - Конрад Цузе "Вычисляющее пространство"

Это означает, что две противоположно движущиеся частицы не влияют друг на друга, но проходят мимо или проходят друг через друга без изменения формы. В системе, строго описываемой принципом суперпозиции, невозможны результаты, которые соответствуют реакциям между элементарными частицами, известными в физике. Это свидетельствует о том, что нет необходимости встраивать линейные элементы в наши модели. Наиболее простой и грубый вид --- это общее ограничение значений сверху и снизу. Это может быть продемонстрировано примерами на рис. 19.

\begin{figure}[htbp]
\centering
\includegraphics[width=0.6\textwidth]{images/page_026_img_01.png}
\caption{Примеры цифровых частиц с различными результатами суперпозиции при ограничении значений}
\label{fig:19}
\end{figure}

Здесь у нас есть две приближающиеся цифровые частицы, а именно в примерах (1) и (2) слева, соответствующие предыдущей реакции согласно принципу суперпозиции. Мы видим, что в примере (1) возникают значения $+2$ и $-2$. В примере (3) частицы проходят друг через друга без возникновения значений больше $+1$ и $-1$.

В этой ситуации может быть наблюдён интересный результат грубой дискретизации. Ход процесса столкновения отличается от фазового состояния расстояния между двумя частицами. Это не видно внешне. На рис. 19 показан пример (1) с законом ограничения, соответствующим рис. 20.

\begin{figure}[htbp]
\centering
\includegraphics[width=0.6\textwidth]{images/page_026_img_02.png}
\caption{Система вычисления с нелинейным правилом, где $1 + 1$ дает значение 1}
\label{fig:20}
\end{figure}

Здесь есть только три значения: $-$, 0 и $+$. На рис. 20 показана соответствующая система расчётов. Она построена таким образом, что $1 + 1$ даёт значение 1. Мы видим, что несмотря на это ограничение, частицы свободно могут пересекаться друг с другом, результат, который сам по себе был бы неожиданным с первого взгляда, так как были сделаны грубые сокращения вычислительного правила. Применение вычислительного правила (рис. 20) к примеру (2) не даёт ничего нового, конечно, потому что в примере не встречаются значения $-2$ и $+2$.

Интересно то, что несмотря на это, может быть отмечен определённый процесс реакции при взаимодействии частиц. Если рассмотреть примеры (2) и (3), например, можно увидеть, что в случае (3), в отличие от (2), может быть наблюдена определённая замедленность процесса. В (2) частицы пересекаются и удаляются друг от друга беспрепятственно. В (3) можно утверждать, что частицы сначала реагируют друг с другом и что в результате этой реакции испускаются две новые цифровые частицы. Вопрос о том, происходит ли (2) или (3), снова зависит от фазового состояния расстояния и внешне является вопросом случайности.
