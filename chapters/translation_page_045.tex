% Страница 45: Продолжение раздела 4.3 и начало 4.4
% Конрад Цузе "Rechnender Raum"

\begin{figure}[h]
\caption*{Рис. 67}
\end{figure}

$x', t$ согласно специальной теории относительности, относительно которой движущаяся частица неподвижна. Инверсия, соответствующая преобразованиям Лоренца, дает объем сдвига $P_0, P'_1, P'_2, P'_3$.

Это равно по площади объему сдвига $P_0, P_1, P_2, P_3$. Мы можем, следовательно, говорить об инвариантности объема сдвига.

\section{Рассмотрения теории информации}
\label{sec:information-theory-considerations}

Термин информация получает значительное значение в ходе этих различных рассмотрений. Теория информации сформулировала термин «содержание информации» с ясностью в отношении систем передачи новостей. По этой причине мы склонны рассматривать теорию информации как теорию обработки информации.

Это, однако, не совсем правильно. Легко совершаемое применение терминов из теории информации в соседней области передачи новостей, к сожалению, приводит к частой путанице. Даже в настоящем рассмотрении нам необходимо ясно представлять, что понимается под содержанием информации. Сложно говорить о физических процессах в терминах передачи новостей. Это было бы интересно само по себе только поскольку мы могли бы включить людей в наше рассмотрение. Если мы предположим бесконечно тонкое распространение наших новостей, передаваемых посредством электромагнитных волн, она должна быть бесконечно сохранена, пока границы не будут установлены для них временной конечностью вселенной. Метафорически мы можем также рассмотреть лучи из вселенной, приближающиеся к нам от других звезд, как новости для людей, в этом случае вопрос о содержании информации этих новостей имеет смысл.

Такое отношение между человеком и природой находится в современном утверждении квантовой теории, которая стремится соотнести все измеримые величины в математической системе. Информация, которую мы получаем от природы о структуре атомных оболочек, состоит во многом из частот испущенного света кванта. В этом случае использование термина «содержание информации» является значимым. Вопрос не будет далее исследоваться здесь.

Если мы пренебрегаем этим определением информации как средства передачи новостей, все еще невозможно говорить о содержании информации необитаемых систем, если мы рассматриваем ширину вариации возможных форм объекта, паттерна или тому подобного. Таким образом, карточка с перфорацией может содержать, благодаря своей вариабельности, определенное содержание информации, измеренное в битах.

Технические характеристики самой карточки с перфорацией, включая сопровождающие системы пробивания и считывания, устанавливают верхние пределы количества информации, которая может быть введена, что определяется как емкость информации. В передаче новостей эта емкость не нуждается быть полностью использована, так что информация, передаваемая от отправителя к получателю на карточке с перфорацией
