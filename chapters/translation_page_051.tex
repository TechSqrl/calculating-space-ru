% Перевод страницы 51 "Rechnender Raum" Конрада Цузе
% Раздел 4.4 и начало раздела 4.5

содержание которой можно уже предсказать, не имеет информационного содержания). Всякая конфигурация по необходимости представляет через свои правила ограничение возможных средств представления и тем самым уменьшает информационное содержание. Сохранение информации и сохранение конфигурации в определённой степени противоречивы.

Вопрос о том, можно ли интерпретировать проверенные в физике термины (энергия, действующий квант, элементарный заряд, масса и т.\,д.) терминами информационной теории или обработки информации, пока ещё не может быть решен.

В модели клеточного автомата, построенной так, чтобы в нём происходили процессы, которые можно связать с перечисленными физическими величинами, эти величины должны быть представлены конструкцией схем; то есть величинами, представленными в схемах.

Ещё более важным, чем понятие информационного содержания, является понятие информационного обмена. Из принципов схемы вытекает нечто динамическое, а не нечто статическое. Возможно, это можно было бы назвать сохранением событий или усложнением событий (доктор Реше предложил идею <<сохранения сложности>>, хотя в другом контексте). Рассматриваемый таким образом, процесс сдвига приобретает дополнительное значение. Если действующему кванту приписать размерность <<процесс сдвига>>, то для энергии мы получим размерность <<процесс сдвига в единицу времени>>. Принцип сохранения энергии можно тогда интерпретировать как принцип сохранения событий. Термин <<действующий квант>> уже указывает на тесную связь с эффектами сдвигового типа, а именно с процессом сдвига. Представление энергии как <<события>> делает связь между энергией и частотой более легко понятной. Эти мысли пока что только простые предположения. Их цель --- стимулировать применение средств наблюдения автоматной теории в физике.

Рассмотрение принципа неопределённости Гейзенберга с точки зрения информационной теории следует далее. Если имеется запоминающая ёмкость в $m$ бит для цифрового представления двух величин $A$ и $B$, то мы имеем свободу распределить две величины различным числом мест и даже различающимися точностями по числу мест. Если величине $A$ отведено $n$ мест, то $B$ имеет $m - n$ мест. Ошибка в $A$ имеет порядок величины $2^{-n}$, ошибка в $B$ --- порядок величины $2^{-(m-n)}$. Произведение обеих ошибок даёт постоянную $2^{-m}$.

Возможно предположить, что обе сопряжённые величины $A$ и $B$ представляются не непосредственно паттерном цифровых частиц, а представляют выведённые величины, которые появляются только в определённых процессах. Ограничения на информационное содержание цифровых частиц не позволяют обеим величинам быть представленными с максимально возможной точностью. В случае цифровых частиц, даже если одна из величин полностью неопределена, другая
