% Страница 10 - продолжение раздела 2.2

\begin{wrapfigure}{r}{0.3\textwidth}
  \includegraphics[width=\linewidth]{images/page_010_img_01.png}
  \caption{Механический суммирующий механизм в форме рычага}
  \label{fig:5}
\end{wrapfigure}


\begin{wrapfigure}{l}{0.4\textwidth}
  \includegraphics[width=\linewidth]{images/page_010_img_02.png}
  \caption{Вращающий механизм с шестернями}
  \label{fig:6}
\end{wrapfigure}


находящимся в контакте с фрикционным колесом B. Расстояние $r$ фрикционного
колеса B от оси A может изменяться. Таким образом, механизм может использоваться
для интегрирования. В современных аналоговых приборах эти механические элементы
заменены электронными. Интегрирование может, например, выполняться путём
зарядки конденсатора.

\begin{wrapfigure}{r}{0.48\textwidth}
  \includegraphics[width=\linewidth]{images/page_010_img_03.png}
  \caption{Интеграционный механизм с фрикционным диском}
  \label{fig:7}
\end{wrapfigure}


Прерывистые процессы, как правило, не воспроизводимы с помощью аналоговых
приборов; другими словами, аналоговые компьютеры плохо спроектированы для
этих процессов.

В цифровых компьютерах все значения представлены числами. Поскольку цифровой
компьютер может содержать только определённую ограниченную сумму чисел, для
представления непрерывных значений доступен только ограниченный запас значений.
Это подразумевает значительное расхождение с математическими моделями.
Математические значения подчиняются концепции бесконечности в двух отношениях.

Во-первых, абсолютная величина чисел неограничена; кроме того, между любыми
двумя заданными значениями может предполагаться существование бесконечного
числа промежуточных значений. По этой причине компьютеры имеют (независимо
от используемого числового кода) максимальные значения, которые из технических
соображений (количество разрядов регистра и памяти) не могут быть превышены.
Кроме того, значения идут ступенчатым образом. Существуют соседние значения,
между которыми не могут быть вставлены дополнительные промежуточные значения. Это
