% Страница 8 - продолжение раздела 2.1

\begin{wrapfigure}{r}{0.4\textwidth}
  \includegraphics[width=\linewidth]{images/page_008_img_01.png}
  \caption{Элементарные логические операции}
  \label{fig:2}
\end{wrapfigure}


\begin{wrapfigure}{l}{0.3\textwidth}
  \includegraphics[width=\linewidth]{images/page_008_img_02.png}
  \caption{Схема автомата для двухразрядного двоичного регистра}
  \label{fig:3}
\end{wrapfigure}


автомата для двухразрядного двоичного регистра. На рисунке $E_1$ и $E_0$ представляют
входы, на которые может быть подано двухразрядное двоичное число, а $A_2$, $A_1$
и $A_0$ представляют выходы, которые имеют значение трёхразрядного двоичного
числа. Двухразрядное двоичное число, образованное из разрядов $A_1$ и $A_0$,
передаётся обратно в автомат и представляет текущее состояние двоичного числа.
(В данном случае состояния символизируют число, уже введённое в процесс сложения,
к которому должно быть добавлено число $E_1$, $E_0$).

\begin{wrapfigure}{r}{0.3\textwidth}
  \includegraphics[width=\linewidth]{images/page_008_img_03.png}
  \caption{Таблица состояний для автомата на Рис. 3}
  \label{fig:4}
\end{wrapfigure}


Алгоритм, заданный автоматом, в простых случаях может быть представлен таблицами
состояний. Они имеют форму матрицы и для каждого состояния и каждой комбинации
входов дают результирующее состояние или комбинацию выходов. На Рис. 4 показана
таблица состояний для автомата на Рис. 3. В данном конкретном случае таблица
состояний представляет собой таблицу сложения. Теория автоматов исследует
различные возможные преобразования такого автомата и устанавливает ряд общих
правил, касающихся его метода работы. Для дальнейшего важно понимать термины
\textit{конечный}, \textit{автономный} и \textit{клеточный автомат}. Конечный автомат работает
с дискретным числом дискретных состояний; он приблизительно эквивалентен цифровой
машине обработки данных, которая состоит из ограниченного числа элементов, причём
каждый элемент способен принимать ограниченное число состояний (по крайней мере два),
в результате чего весь автомат может принимать только ограниченное число состояний.
Аналогичные условия справедливы для входов и выходов. Автономный автомат не может
принимать никаких входов (выходы также относительно несущественны). Он может
