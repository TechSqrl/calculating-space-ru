% Страница 48: Продолжение раздела 4.4 - Рассмотрения теории информации
% Конрад Цузе "Rechnender Raum"

Рассмотрение замкнутых процессов, т.е. сдвигающихся процессов, включающих цифровую частицу, представляется более плодотворным. Мы уже наблюдали, что цифровая частица состоит из серии периодически повторяющихся паттернов в клеточном автомате и что они не закреплены на месте, а могут двигаться в пространстве отдельных ячеек, подобно движущейся пишущей машине. Термин «текущее состояние» уже был введен.

Вопрос о содержании информации цифровой частицы можно рассматривать с нескольких точек зрения. Сначала цифровая частица занимает установленную позицию в пространстве в определенный момент времени. Содержание информации цифровой частицы не может быть больше, чем информационная емкость этой позиции в пространстве, которая определяется суммой возможных состояний этой области. Крайне маловероятно, что каждое изменение состояния такой ограниченной области соответствует цифровой частице. Гораздо более вероятно, что ограниченное выделение растворяет отдельные стабильные периодические паттерны.

Мы можем запросить, полностью независимо от пространства, связанного с цифровой частицей, сколько вариаций паттерна, представляющих фазы цифровой частицы, на самом деле возможны? Выгодно классифицировать паттерны вдоль различных линий:

\begin{enumerate}
\item тип;
\item направление и скорость (импульс);
\item состояние фазы;
\item позиция частицы.
\end{enumerate}

Ответ на вопрос 1 предполагает наличие модели, которая допускает различные типы цифровых частиц, как мы имеем в природе с фотонами и электронами и т.д.

Ответ на вопрос 2 требует, чтобы наша модель допускала различные скорости и направления распространения периодического паттерна.

Последовательность фаз результирует из последовательности периодического паттерна, связанной со специальным типом частицы и импульса.

Вопрос 4 имеет смысл только когда рассматривается взаимоотношение частиц. Конечно, невозможно, чтобы замкнутая область пространства держала информацию о собственном состоянии.

Примеры на рис. 42--66 из Главы 3 удовлетворяют этим условиям только в ограниченной мере. Во-первых, модель допускает представление только одного типа частиц. Далее только направление может быть варьировано, но не скорость. Длина периодов отдельных частиц не постоянна, но это не представляет интереса для нашего рассмотрения. Содержание информации этого типа частиц зависит от точности представления длины стрелки или от числа мест, с которым она цифрово представлена. Если мы предположим абсолютные длины компоненты для примера 4, то мы получаем 9 различных длин стрелки, включая нулевое значение, для этого компоненты; в двумерном пространстве есть 81 различные вариации импульса. На основе этих возможных вариаций в частицах, даже в пределах данных ограничений, возможно определить содержание информации частицы. Каждая из этих частиц имеет серию связанных состояний фазы, так что число возможных паттернов цифровых частиц все еще больше. Частица на рис. 59 имеет, например, 7 различных состояний фазы ($\tau_0 - \tau_6$).

Вопрос о сохранении информации при реакции между цифровыми частицами является интересным. В примерах, приведенных в Главе 3, импульсные стрелки складываются в ходе реакции. Это означает, что число
