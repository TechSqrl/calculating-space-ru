% Страница 28 - Конрад Цузе "Вычисляющее пространство"

В этом отношении тернарная система превосходит двоичную. Значение 1/2 лежит точно посередине между 0 и 1. Значения 1/3 и 2/3 также могут быть точно вставлены между значениями 0 и 1.

Отсюда мы хотим начать со следующего:

\[
v - \frac{\Delta p}{3} \Rightarrow v
\]

\[
p - \frac{\Delta v}{3} \Rightarrow p
\]

Значения $\Delta p/3$ и $\Delta v/3$ округляются вверх или вниз до целых чисел.

\begin{figure}[htbp]
\centering
\includegraphics[width=0.6\textwidth]{images/page_028_img_01.png}
\caption{Расширение импульса давления в тернарной системе}
\label{fig:23}
\end{figure}

\begin{figure}[htbp]
\centering
\includegraphics[width=0.6\textwidth]{images/page_028_img_02.png}
\caption{Система вычислений с коэффициентом 1/3}
\label{fig:24}
\end{figure}

На рис. 25 (1) показана стабильная частица в этой системе с периодом $3\Delta t$. Скорость распространения составляет 1/3 от скорости частицы в соответствующем рисунке (рис. 15). Это также соответствует физической модели, в которой мягкая среда имеет более медленную скорость звука. Здесь мы имеем ситуацию, при которой "скорость переключения" между соседними частицами значительно выше (в примере в три раза больше), чем скорость частицы.

\begin{figure}[htbp]
\centering
\includegraphics[width=0.6\textwidth]{images/page_028_img_03.png}
\caption{Стабильная частица в тернарной системе: (1) частица с периодом $3\Delta t$; (2) детальная система расчётов}
\label{fig:25}
\end{figure}

В более сложных моделях "вычисляющего пространства" было бы понятно, что скорости света, соответствующие максимальным скоростям частиц, которые значительно медленнее скорости переключения, существуют. Однако это не означает, что в такой модели возможны "скорости сигналов", превышающие скорость света (в модели). Скорость переключения имеет чисто локальное значение.

Интересно то, что цифровая частица принимает различные конфигурации в течение периода. Импульс давления появляется частично только с значением $+2$, частично как пара со значениями $+1$ и $+1$. Положение частицы может быть определено для следующего периода, но не без дополнительной информации для отдельных фаз данного периода. Разве это не аналогично квантовой теории, которая связывает положение и импульс через принцип неопределённости? В любом случае компьютерная модель, несмотря на кажущуюся ошибку, характеризуется строго предопределёнными событиями.
