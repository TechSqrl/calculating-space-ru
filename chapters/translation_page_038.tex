% Страница 38: Завершение раздела 3.3 и начало 3.4
% Конрад Цузе "Rechnender Raum"

\begin{figure}[htbp]
    \centering
    \includegraphics[width=0.6\textwidth]{images/page_038_img_01.png}
    \caption*{Рис. 51}
\end{figure}

\begin{figure}[htbp]
    \centering
    \includegraphics[width=0.6\textwidth]{images/page_038_img_02.png}
    \caption*{Рис. 52}
\end{figure}

\begin{figure}[htbp]
    \centering
    \includegraphics[width=0.6\textwidth]{images/page_038_img_03.png}
    \caption*{Рис. 53}
\end{figure}

\begin{figure}[htbp]
    \centering
    \includegraphics[width=0.6\textwidth]{images/page_038_img_04.png}
    \caption*{Рис. 54}
\end{figure}

могли бы построить временную линию фазы $R$, которая представляет обе частицы. Если она проходит через точку пересечения траекторий частиц $S$, то реакция возможна (рис. 65 и 66).

Разумеется, эти примеры очень просты и примитивны. Однако даже эти простые формы дают обилие подсказок; они показывают, что избранный основной метод дискретизации представляет наибольший интерес и что развитие правил принесет дополнительные концепции.

\section{О трехмерных системах}
\label{sec:three-dimensional-systems}

Концепции, развитые в разделах 3.2 и 3.3, могут быть также применены к трехмерным системам. Исследования автора в этой области еще не завершены и должны быть отложены для дальнейшего изучения.
