% Страница 44: Продолжение раздела 4.3 - О теории относительности
% Конрад Цузе "Rechnender Raum"

чем скорость распространения сигнала, полученного из этой передачи. Эта более высокая скорость передачи имеет только локальное значение. Из-за анизотропии вычисляющего пространства она различна в различных направлениях. В любом случае «цифровая» модель, по сравнению с аналоговой моделью теории относительности, дает значительное различие: чем ближе скорость инерциальной системы приближается к стандарту скорости света, тем более критичной становится цифровая имитация процессов. В случае богатых энергией частиц мы приходим к процессам, которые можно охарактеризовать (по крайней мере до некоторой степени) как «ошибка вычисления» вычисляющего пространства. Таким образом, существенно различное поведение частиц очень высокой энергии (высокая скорость, высокая частота) может быть объяснено.

Строгая интерпретация специальной теории относительности имеет следствием, что для каждой инерциальной системы может быть воображена другая, которая движется с начальной скоростью меньшей, чем $c$. Физические правила столь же справедливы во второй системе, как и в первой. Этот процесс может быть повторен столько раз, сколько желательно, по крайней мере в принципе. Полная чудовищность этой мысли только смутно понятна. Здесь должно быть сказано опять, что всякое представление бесконечности предполагает ограничивающий процесс. Здесь мы имеем дело с бесконечно частым повторением реакции другой инерциальной системы, которая движется относительно предыдущей. Этот процесс имеет немного последствий, если применяются наблюдения информационно-теоретического характера, как мы будем рассматривать далее.

Следующее утверждение также представляет интерес.

\begin{figure}[htbp]
  \centering
  \includegraphics[width=0.6\textwidth]{images/page_044_img_01.png}
  \caption*{Рис. 67}
\end{figure}

Мы сначала введем термин «объем сдвига». Это равно числу сдвигающихся частей, вовлеченных в процесс, умноженному на число сдвигающихся импульсов, которые участвуют в данном процессе, например в периоде цифровой частицы.

На рис. 67 показано упрощенное представление, в котором можно предположить, что возмущение, представляющее цифровую частицу, простирается на расстояние $P_0 - P_1$. Предполагается, что частица неподвижна в инерциальной системе $x, t$. В этом случае пространство $P_0, P_1, P_2, P_3$ равно объему сдвига одного периода. Если эта частица движется относительно системы $x, t$, мы можем говорить о второй инерциальной системе
