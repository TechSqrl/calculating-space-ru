% Страница 32 - Конрад Цузе "Вычисляющее пространство"

Поскольку мы не можем предположить бесконечное число малых значений в цифровом пространстве, минимальное значение вскоре достигается; то есть пик затухает. Было бы интересно проследить прогресс такого расширения с помощью вычислительной машины. Вопрос особого интереса состоит в том, возможно ли и как быстро значения сходятся в круговой схеме расширения.

Одно ясно: невозможно построить цифровые частицы из такого правила. Мы должны найти другие правила.

\begin{wrapfigure}{r}{0.40\textwidth}
    \centering
    \includegraphics[width=0.38\textwidth]{images/page_032_img_01.png}
    \caption{Расположение значений $v$ и $p$ в шахматном порядке}
    \label{fig:31}
\end{wrapfigure}

\begin{wrapfigure}{l}{0.30\textwidth}
    \centering
    \includegraphics[width=0.28\textwidth]{images/page_032_img_02.png}
    \caption{Отдельные появляющиеся значения}
    \label{fig:32}
\end{wrapfigure}

\begin{wrapfigure}{r}{0.40\textwidth}
    \centering
    \includegraphics[width=0.38\textwidth]{images/page_032_img_03.png}
    \caption{Волновой фронт параллельно одной из осей координат}
    \label{fig:33}
\end{wrapfigure}

Можно взять правила для линейного пространства, которые порождают стабильные частицы, и применить их к двумерному пространству. Конечно, тогда нам нужна взаимосвязь между двумя измерениями, так как без неё отдельные ортогональные точки сетки имели бы независимое существование. На рис. 31 показана одна возможность расположения значений $v$ и $p$ в шахматном порядке. На рис. 32 показаны отдельные появляющиеся значения. Два компонента, $v_x$ и $v_y$, должны быть рассмотрены для $v$. Одного значения достаточно для $p$.
