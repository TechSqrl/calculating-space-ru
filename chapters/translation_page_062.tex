% Перевод страницы 62 из "Rechnender Raum" Конрада Цузе
% Послесловие: страница 62

потому, что Цузе должен был иметь дело с мельчайшими деталями действительного построения физической машины (например, стандарт IEEE для кодирования чисел с плавающей запятой почти идентичен представлению, использованному в Z1 и Z3 Цузе). Цузе никогда не думал об универсальности, как Тьюринг, но, как доказал Рохас, не без определённого творчества, Z1 и Z3 случайным образом (потому что это никогда не было целью Цузе, и он даже не сформулировал вопрос) оказались способны к универсальному вычислению\textsuperscript{4}. Цузе никогда не думал о том, как машина могла бы перейти в неограниченное вычисление (необходимое для универсальности), например, и если бы это произошло, как заставить её остановиться (Рохас предполагает, что потребовался бы механический/электрический трюк для произвольной остановки машин, с требуемым вычислением завершённым и каким-то образом закодированным среди других вычислений в выходе, если бы неограниченное вычисление было разрешено, например, путём зацикливания перфокарты).

После выпуска в 1935 году Цузе стал анализатором напряжений в авиастроительной компании Henschel, где работал над проблемами вибрации летательных аппаратов. Анализ напряжений требовал грозных расчётов, которые в то время могли быть выполнены только с большим трудом с помощью команд людей-вычислителей, оснащённых портативными счётными машинами\textsuperscript{5}. Цузе считал, что многие расчёты, которые он выполнял, могли бы быть просто автоматизированы. Получив грант на исследования в 1936 году от Reichsluftfahrtministerium (немецкого министерства авиации), он случайным образом построил свою первую вычислительную машину между 1936 и 1938 годами, а в 1938 году он строил свою вторую, используя телефонные реле в отличие от первой, которая была механической. Его Z3 был завершён в 1941 году, был полностью функционален и мог выполнять вычисления\textsuperscript{6}. Его Z1 уже был программируемым, несмотря на механическую конструкцию, используя перфоленту.

Его основная мотивация переключиться с механической на электронную форму была обусловлена заботой о надёжности~--- он хотел построить устойчивые и отказоустойчивые машины~--- но Z3, построенный с электронными реле, был логически эквивалентен Z1. Z1 и Z3 могли быть запрограммированы и могли выполнять все арифметические вычисления, могли загружать и сохранять информацию в двоичном коде и были способны к вычислениям с плавающей запятой (тогда как Mark I и ENIAC в США всё ещё представляли данные в десятичном коде, несмотря на то, что оба работали

\footnotesize
\noindent
\textsuperscript{4}См. <<The Architecture of Konrad Zuse's Early Computing Machines>> Рауля Рохаса в <<The First Computers -- History and Architecture>>, MIT Press, 2000, pp. 237-262, под редакцией R. Rojas и Ulf Hashagen.

\noindent
\textsuperscript{5}\url{http://www.independent.co.uk/news/people/obituary--konrad-zuse-1526795.html} (доступ апрель 2012).

\noindent
\textsuperscript{6}Онлайн видео, сделанное в Deutsches Museum München, показывает, как работал Z3, на примерах арифметического деления и вычисления квадратных корней: \url{http://www.youtube.com/watch?v=J98KVfeC8fU} (доступ апрель 2012)
