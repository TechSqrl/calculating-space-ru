% Страница 46: Продолжение раздела 4.4 - Рассмотрения теории информации
% Конрад Цузе "Rechnender Raum"

может быть ниже емкости.

Также возможно говорить о максимальной возможной информационной емкости конечного автомата, если мы рассматриваем число его возможных состояний как меру. Если это равно $n$, то содержание информации равно $\log_2(n)$ (логарифм по основанию два). Запрограммированная вычислительная машина представляет этот тип автомата, как мы знаем. Если такой инструмент имеет $m$ элементов, для каждого из которых есть две возможные позиции (например, триггеры, ферритовые сердечники в памяти и т.д.), то число возможных состояний равно $2^m$, а информационная емкость равна $m$. В этом процессе не делаются различия между отдельными возможными состояниями. Из всего $2^m$ возможных состояний каждое состояние, в котором каждый регистр и запоминающее устройство удален (т.е. установлен в ноль), считается так же, как состояния, в результате которых решение очень сложного дифференциального уравнения хранится в памяти. Эмоционально мы естественно склонны предполагать, что оборудование не содержит информации в его нулевом состоянии, хотя во втором упомянутом состоянии доступны чрезвычайно интересные научные результаты для использования математиками. Этот пример показывает необходимость большой осторожности в определении терминов в теории информации.

Различие в этой ситуации состоит в том, что для получателя два состояния имеют принципиально различное значение. Состояние «всё удалено» является только расширением знания получателя о том, что машина в данный момент находится в основном состоянии, в то время как во втором случае знание получателя увеличивается в отношении значительных результатов.

Если не учитывать эти отдельные значения информации для получателя, то можно сделать вывод, что содержание информации конечного автомата не может быть увеличено во время выполнения вычисления. Потому что вычисление производится полностью автоматически после введения программы и входных значений, результаты установлены с самого начала. Результаты имеют большую ценность для лица, использующего оборудование: для чего бы он позволил компьютеру выполнить вычисление, если не для увеличения своего знания, что возможно только, если конечное состояние автомата имеет большее содержание информации, чем начальное состояние.

Первый результат рассмотрения космоса как клеточного автомата состоит в том, что отдельные ячейки представляют конечный автомат. Вопрос о том, в какой степени возможно рассматривать всю вселенную как конечный автомат, зависит от допущения, которое мы делаем в отношении ее размеров. Если мы возьмем тор более высокого порядка, как уже предложено Боппой, мы имеем дело с конечным автоматом в целом. Это изначально справедливо, что отдельные ячейки могут принять ограниченное число состояний и, следовательно, имеют только ограниченное содержание информации. Это равно справедливо и для всего космоса, если мы делаем подходящие допущения о его границах.
