% Страница 41: Раздел 4 - Общие рассмотрения, 4.1 Клеточные автоматы
% Конрад Цузе "Rechnender Raum"

\chapter{Общие рассмотрения}
\label{ch:general-considerations}

\section{Клеточные автоматы}
\label{sec:cellular-automata}

Примеры дискретизации полей и частиц, которые были представлены, в своем настоящем незавершенном виде все еще далеки от возможности служить в формулировке физических правил. Тем не менее, они дают грубое впечатление о возможностях использования инструментов теории автоматов для ответа на физические вопросы.

Примеры в основном рассматривали точечные решетки. Отдельный клеточный автомат состоит, следовательно, из точечной решетки, которая связана с соседними точками посредством обмена информацией. В случаях, показанных на рис. 34 и 35, решетки представляют собой шахматные доски двух различных значений $p$ и $v$ в решеточной форме. Существуют различные возможности их комбинации, так что разделение на отдельные автоматы не специфично. Это не влияет на поведение всей системы.

В целом разделение континуума на дискретные клеточные автоматы имеет различные последствия в зависимости от точного способа разделения. Идея структуры пространственной решетки уже рассматривается в различных контекстах физиками, хотя не применительно к теории автоматов. Вообще говоря, идея того, что космос действительно может быть подразделен на такие ячейки, физиками резко отвергается. Мы согласны с тем, что пространство не может рассматриваться как континуум даже в бесконечно малых сечениях. Концепция наименьшей длины уже широко принята сегодня, хотя не в отношении идеи подразделения на точечную решетку, а скорее как принципиальный предел в различении двух различных частиц. Сомнения, касающиеся структуры решетки, существенно следующие:

\noindent\textbf{(a)} Структура решетки устранила бы изотропию пространства.

Ясно, что регулярная решётчатая структура задаёт привилегированные направления. Это имеет влияние, например, на расширение полей (рис. 31, 38) и на дискретные возможные направления, в которых цифровая частица может двигаться (рис. 60). Мы не знаем физических экспериментов, которые дали бы ключ к предпочтительным направлениям такого типа, но область не была систематически изучена на предмет этого эффекта. Трезвое размышление, тем не менее, показывает, что стоит рассмотреть правила для подобной решеточно-пространственной структуры, которые не позволяют структуре решетки становиться видимой в областях меньшей и промежуточной энергии и частот. Постоянная решетки должна быть значительно меньше, чем элементарная наименьшая длина приблизительно $10^{-13}$ см (Боппа предполагает даже $10^{-56}$ см). Область обычной оптики, например, работает с длинами волн экстраординарной величины в сравнении с этими длинами. Едва ли возможно представить себе эксперимент, который смог бы
