% Страница 23 - Конрад Цузе "Вычисляющее пространство"

\chapter{Примеры дискретной обработки полей и частиц}
\label{ch:digital-fields}

\section{Выражение <<Цифровая частица>>}
\label{sec:digital-particle}

Рассмотрим сначала одномерное пространство. В этом отношении мы можем привести пример из гидромеханики и один из инженерии приборов. Рассмотрим поведение газов без трения в прямом цилиндре. После исключения и сбора слагаемых, которые для наших целей неуместны (плотность и т.д.), мы можем получить несколько упрощённое соотношение реальных физических сил.

У нас есть две величины: $p$ (давление), которое мы фиксируем в дискретных точках 1, 2 и 3, и $v$ (скорость), которую мы выражаем в промежуточных точках 1', 2' и 3'.

\[
\begin{array}{cccccc}
p & 1 & 2 & 3 & 4 & 5 \\
v & 1' & 2' & 3' & 4' & 5'
\end{array}
\]

где $\Delta_s p$ и $\Delta_s v$ представляют разность значений $p$ и $v$ между соседними точками, $\Delta_t p$ и $\Delta_t v$ соответствуют разностям между $p$ и $v$ в последовательных временных интервалах.

Следующие дифференциальные уравнения затем имеют место:

\[
k_0 \Delta_s p \Rightarrow \Delta_t v
\]

\[
k_1 \Delta_s v \Rightarrow \Delta_t p
\]

Выраженное словами: изменение скорости пропорционально изменению давления, и разность давления пропорциональна изменению скорости. Во втором уравнении член $\Delta_t p$ переписан так, чтобы указать, что он относится к $\Delta p$ после того, что в первом уравнении. Два коэффициента $k_0$ и $k_1$, содержащие физические характеристики $\Delta x$ (пространственный компонент) и $\Delta t$ (временной компонент), могут быть для наших целей объединены в один коэффициент $k$. Мы затем получаем:

\[
- \Delta_s p \Rightarrow \Delta_t v
\]

\[
- k \Delta_s v \Rightarrow \Delta_t p
\]

Символ $\Rightarrow$ используется, чтобы указать, что $\Delta p$ во втором уравнении не идентичен тому, что в первом уравнении.

Ясно, что эти уравнения могут быть преобразованы из дифференциальных уравнений в разностные уравнения, когда $\Delta x$ и $\Delta t$ стремятся к нулю.
