% Страница 21 - Конрад Цузе "Вычисляющее пространство"

\begin{wrapfigure}{r}{0.30\textwidth}
    \centering
    \includegraphics[width=0.28\textwidth]{images/page_021_img_01.png}
    \caption{Модель столкновения частиц}
    \label{fig:13}
\end{wrapfigure}

Процессы столкновения являются наиболее интересной частью. Предполагается равенство массы и упругости частиц. Сначала рассмотрим случай, когда частицы встречаются точно; то есть, во-первых, траектории лежат в одной плоскости и взаимно пересекаются, и во-вторых, центры обеих частиц одновременно встречаются в точке пересечения. Этот случай неинтересен, так как упругое столкновение не отличается значительно от ситуации, в которой обе частицы продолжают двигаться беспрепятственно по своим путям, если каждую частицу рассматривать отдельно. Кроме того, в общих ситуациях вероятность возникновения такого случая стремится к нулю по мере повышения точности расчётов. Поэтому интерес представляют только те случаи, в которых траектории не пересекаются точно, или в которых центры подходят к приблизительному пересечению только примерно одновременно.

\begin{wrapfigure}{l}{0.30\textwidth}
    \centering
    \includegraphics[width=0.28\textwidth]{images/page_021_img_02.png}
    \caption{Траектории частиц после столкновения}
    \label{fig:14}
\end{wrapfigure}

В этом случае частицы имеют различные траектории после столкновения, чем до него. Здесь нет необходимости останавливаться и прочно устанавливать закон столкновения. Поведение зависит от размера частиц и закона упругости. Крупные частицы сталкиваются чаще, чем маленькие. Жёсткие частицы ведут себя иначе, чем мягкие. Статистический результат поведения большого числа частиц остаётся одним и тем же.

Если сравнить такую вычислительную модель с физической моделью, возникают следующие интересные аспекты. В обоих случаях можно увидеть, что в целом упорядоченные состояния приводят к неупорядоченным состояниям, то есть энтропия возрастает. В любом случае можно создать определённые исключительные случаи, в которых данная энтропия остаётся постоянной. Возьмём, например, сосуд с точно параллельными стенками и серию частиц, траектории которых точно перпендикулярны этим стенкам и достаточно далеко отстоят друг от друга, так что взаимодействия частиц между собой нет. В этом случае траектории остаются неизменными в смысле классической механики. То же самое справедливо и для компьютерной модели, если система координат, на которой основаны расчёты, установлена параллельно или ортогонально стенкам. Конечно, существуют и другие интересные частные случаи, в которых происходят процессы столкновения между частицами, но тем не менее остаётся определённый порядок (рис. 14).
