% Страница 31 - Конрад Цузе "Вычисляющее пространство"

\[
\varphi_{x-1,y} \vee \varphi_{x+1,y} \vee \varphi_{x,y-1} \vee \varphi_{x,y+1} \Rightarrow \varphi_{x,y}
\]

\begin{wrapfigure}{r}{0.50\textwidth}
    \centering
    \includegraphics[width=0.48\textwidth]{images/page_031_img_01.png}
    \caption{Расширение импульса в двумерной сетке}
    \label{fig:30}
\end{wrapfigure}

Расширение вдоль осей координат происходит быстрее, чем вдоль диагоналей. С таким правилом можно развить мало, так как через короткое время оно приводит к состоянию, при котором все пространственные точки достигают состояния "1" и тем самым никакие конфигурации, частицы и т.д. невозможны (рис. 30).

Далее мы рассмотрим аналогичное правило, в котором, тем не менее, допускаются многоместные значения, а комбинация происходит путём сложения. При передаче между точками сетки значения умножаются на коэффициент $k$. Мы получаем формулу для этого правила:

\[
K(\varphi_{x-1,y} + \varphi_{x+1,y} + \varphi_{x,y-1} + \varphi_{x,y+1}) \Rightarrow \varphi_{x,y}
\]
На рисунках 32 и 33 даны два примера для коэффициентов 1/4 и 1/2. По причинам симметрии необходимо рассмотреть только 45° сечение. Как на рис. 32, значения введены только для фронта импульса. Римские цифры соответствуют отдельным временным фазам с временным интервалом разделения $\Delta t$. Из примеров видно, что фронт движется так, как представлено на рис. 31; то есть, с его пиком вдоль оси координат, хотя значения вдоль диагоналей больше. Устремляющаяся вперёд точка очень скоро достигает своего пика.
