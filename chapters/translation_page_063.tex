% Перевод страницы 63 из "Rechnender Raum" Конрада Цузе
% Послесловие: страница 63

с двоичными логическими элементами и были неспособны к вычислениям с плавающей запятой).

Цузе решил использовать двоичную систему и металлические пластины, которые могли двигаться только в одном направлении, то есть могли только менять положение, точно так же, как это делают современные цифровые компьютеры на своём самом низком уровне работы (Цузе, казалось, верил, что механические устройства и вычисления на цифровой основе были более надёжны по сравнению, например, с вакуумными трубками, как предполагал его друг Helmut Schreyer).

\vspace{0.3cm}

\textbf{Рисунок 75:} Реплика первого механического компьютера, разработанного Конрадом Цузе~--- Z1, завершённого в 1938 году. Это был двоичный электромеханический калькулятор, который использовал логику Буля и двоичные числа с плавающей запятой. Фотография сделана H. Zenil в Deutsches Technikmuseum (<<Немецком музее технологии>>), Берлин.

\vspace{0.3cm}

Цузе и Тьюринг никогда не встречались, но они познакомились с работами друг друга. Цузе упоминает работу Тьюринга в своей автобиографии, и известно, что Тьюринг входил в программный комитет/комитет рецензентов по крайней мере одного коллоквиума, который посещал Цузе~--- но не Тьюринг~--- в Max-Planck-Gesellschaft в Гёттингене в 1947 году. Если бы Тьюринг присутствовал, они действительно встретились бы.

Но если Цузе не открыл концепцию универсального вычисления, он был заинтересован в другом очень глубоком вопросе~--- вопросе о природе природы: <<Является ли природа цифровой?>> Он склонялся к утвердительному ответу, и его идеи были опубликованы, согласно Хорсту Цузе (старший сын Конрада), в Nova Acta Leopoldina. Хорст родился именно тогда, когда Конрад впервые думал о Rechnender Raum (общий перевод на английский~--- <<Calculating Space>>, но фраза на его родном немецком языке несёт намного больше когнитивного веса, чем её простой английский эквивалент, в свете идей, рассматриваемых в работе Цузе: вычисление, вычисление природы, пространство
