% Страница 35 - Конрад Цузе "Вычисляющее пространство"

\begin{wrapfigure}{r}{0.40\textwidth}
    \centering
    \includegraphics[width=0.38\textwidth]{images/page_035_img_01.png}
\caption{Ортогональная сетчатая структура (Рис.~39)}
    \label{fig:39}
\end{wrapfigure}

\begin{wrapfigure}{l}{0.40\textwidth}
    \centering
    \includegraphics[width=0.38\textwidth]{images/page_035_img_02.png}
\caption{Четыре возможных примера одиночного изолированного импульса (Рис.~40)}
    \label{fig:40}
\end{wrapfigure}

\begin{wrapfigure}{r}{0.40\textwidth}
    \centering
    \includegraphics[width=0.38\textwidth]{images/page_035_img_03.png}
\caption{Два случая взаимодействия между двумя стрелками (Рис.~41)}
    \label{fig:41}
\end{wrapfigure}

\begin{wrapfigure}{l}{0.40\textwidth}
    \centering
    \includegraphics[width=0.38\textwidth]{images/page_035_img_04.png}
\caption{Правило для случая пересекающихся стрелок (Рис.~42)}
    \label{fig:42}
\end{wrapfigure}

\begin{wrapfigure}{r}{0.40\textwidth}
    \centering
    \includegraphics[width=0.38\textwidth]{images/page_035_img_05.png}
\caption{Стабильная частица с периодом $2\Delta t$ (Рис.~43)}
    \label{fig:43}
\end{wrapfigure}

\begin{wrapfigure}{l}{0.40\textwidth}
    \centering
    \includegraphics[width=0.38\textwidth]{images/page_035_img_06.png}
\caption{Карманы, закреплённые на четырёх соседних точках решётки (Рис.~44)}
    \label{fig:44}
\end{wrapfigure}

\begin{wrapfigure}{r}{0.30\textwidth}
    \centering
    \includegraphics[width=0.28\textwidth]{images/page_035_img_07.png}
\caption{Двойной стабильный карман с периодом $\Delta t$ (Рис.~45)}
    \label{fig:45}
\end{wrapfigure}
