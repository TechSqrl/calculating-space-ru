% Страница 12 - окончание раздела 2.2 и начало раздела 2.3

\begin{figure}[htbp]
  \centering
  \includegraphics[width=0.6\textwidth]{images/page_012_img_01.png}
  \caption{Гибридная система с цифро-аналоговым и аналого-цифровым преобразователями}
  \label{fig:8}
\end{figure}

Это может быть просто осуществлено через систему, в которой два компьютера
работают параллельно. Они соединены цифро-аналоговым преобразователем и
аналого-цифровым преобразователем (Рис. 8). В системах этого типа отдельные
части задачи разделены таким образом, что для каждого подраздела задачи
выбирается более подходящее устройство.

\begin{wrapfigure}{l}{0.4\textwidth}
  \includegraphics[width=\linewidth]{images/page_012_img_02.png}
  \caption{Представление величин через плотность импульсов}
  \label{fig:9}
\end{wrapfigure}


Объединение двух систем также может быть осуществлено через представление
самих значений. Так, например, величина может характеризоваться плотностью
импульсов (Рис. 9). Сами импульсы имеют цифровой характер, поскольку они
нормализованы по интенсивности и длительности; следовательно, они цифровые,
но их плотность (количество импульсов в единицу времени) может иметь любое
количество промежуточных значений, и поэтому она имеет аналоговый характер.
Сегодня распространено мнение, что человеческая нервная система работает
по этому принципу.

\begin{figure}[htbp]
  \centering
  \includegraphics[width=0.6\textwidth]{images/page_012_img_02.png}
  \caption{Представление величин через плотность импульсов}
  \label{fig:9}
\end{figure}

\section{Дифференциальные уравнения с точки зрения теории автоматов}
\label{sec:differential-equations}

Наблюдение за несколькими дифференциальными уравнениями показывает, что
этот способ мышления отнюдь не является самоочевидным для математиков и
физиков. В нашем распоряжении имеется ряд моделей физических данных,
которые могут быть представлены дифференциальными уравнениями. Например,
мы можем взять простое дифференциальное уравнение для представления формы
верхней поверхности жидкости во вращающемся сосуде, согласно которому в
каждой точке поверхности нормаль к поверхности определяется векторной
суммой гравитационного и центробежного ускорений (Рис. 10).

Это уравнение записывается:
\[
y' = \frac{r\omega^2}{g}
\]
где $\omega$ — угловая скорость контейнера.

Решение очень легко получить аналитически:
\[
y = \frac{\omega^2}{2g} \cdot r^2
\]
