% Страница 47: Продолжение раздела 4.4 - Рассмотрения теории информации
% Конрад Цузе "Rechnender Raum"

Теория автоматов демонстрирует, что различные характеристические ходы выполнения возможны для конечного автомата, несколько из которых будут рассмотрены.

Для каждого заданного состояния существует последующее состояние. Поэтому возможно выразить отношение «состояние $A$ растворяет состояние $B$» как отношение $F(A, B)$ и представить его в форме диаграммы стрелок. Такая диаграмма стрелок часто называется «графом». На рис. 68a--d показаны различные типы диаграмм стрелок. Важно помнить, что каждое состояние может иметь только одно последующее состояние, хотя существует несколько предшествующих состояний, которые могут его растворить. Диаграммы процессов показывают, что автономный автомат должен в каждом случае завершиться периодическим циклом, который при определенных условиях также может вырождаться в единственное конечное состояние.

\begin{figure}[htbp]
  \centering
  \includegraphics[width=0.6\textwidth]{images/page_047_img_01.png}
  \caption*{Рис. 68}
\end{figure}

Это знание не может быть перенесено на отдельные ячейки клеточного автомата, так как они связаны с соседними ячейками посредством обмена информацией и, следовательно, не приводят к автономному конечному автомату.

При допущениях границ космоса во вселенной мы имеем дело с конечным автономным автоматом, как только мы исключаем любого рода влияния более крупного внешнего мира. Первый результат является несколько разочаровывающим следствием того, что космический процесс с необходимостью должен закончиться периодическим циклом. Это осознание, в себе логически неоспоримое, имеет другие значения при количественном исследовании.

Размеры вселенной предполагаются некоторыми физиками быть на порядке величины $10^{41}$ элементарных длин ($10^{-13}$ см) приблизительно (приблизительно 10 миллионов световых лет). Мы имеем дело, следовательно, с объемом приблизительно $10^{123}$ элементарных кубиков элементарной длины на стороне.

Если отдельный бит содержания информации назначен каждому из этих элементарных кубиков, то у нас уже есть $2^{10^{123}}$ различных состояний вселенной, которые должны рассматриваться. Это число представляет только нижний предел. На самом деле, должна быть предположена гораздо более тонкая решетка, для которой еще неизвестно, сколько вариаций возможны в каждой точке решетки. Далее должно рассматриваться, что пространство вычисляет чрезвычайно точно. Соотношение электростатических взаимодействий к взаимодействиям гравитационных полей составляет приблизительно $10^{40} : 1$. Взаимодействие ядерных сил снова на порядки сильнее. Более высокое из двух значений представляет в действительности только нижний предел, который наиболее вероятно на много порядков слишком мал.

Если мы предположим, что число временных импульсов приближается к порядку величины пространственного расширения, в действительности $10^{41}$, то получается результат, что несмотря на это долгое время, только исчезающе малая часть возможных состояний космоса может существовать. Существуют $2^{10^{82}}$ типов путей реакции возможны, каждый из которых независим от любого другого. Это также означает, что число отклонений и ветвлений не поддается пониманию велико. Ранее рассмотренные наблюдения теории автоматов, относящиеся к рис. 68, теряют всю предсказательную ценность. Какова ценность осознания того, что эволюция вселенной следует периодическому циклу, когда даже в уже очень большом диапазоне времени, рассматриваемом одиночный период в лучшем случае может пройти, и в большинстве случаев вообще нет?
