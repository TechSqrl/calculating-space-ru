% Страница 43: Раздел 4.2 - Цифровые частицы и клеточные автоматы
% Конрад Цузе "Rechnender Raum"

\section{Цифровые частицы и клеточные автоматы}
\label{sec:particles-and-automata}

Цифровые частицы можно рассматривать как возмущения в нормальных условиях клеточного автомата. Это возмущение имеет отчетливый паттерн, который подвергается периодическим изменениям. Согласно теории автоматов, каждое состояние развивается из предшествующего; тем не менее, весь паттерн может колебаться в процессе. До некоторой степени мы имеем дело с «текущими состояниями». В соответствии с этим, цифровые частицы можно рассматривать как «самовоспроизводящиеся системы». Данный паттерн генерируется в соседней области клеточного автомата.

В примерах в Главе 3 цифровые поля и цифровые частицы рассматриваются раздельно. Современная теория поля стремится объяснить даже элементарные частицы через сингулярности и специальные формы полей. Теория автоматов, понятно, хорошо подходит для дискретизации таких интерпретаций и их подчинения правилам теории автоматов. Автор надеется иметь возможность рассмотреть этот предмет более глубоко в другом вкладе.

\section{О теории относительности}
\label{sec:relativity-theory}

Вопрос об изотропии пространства, очевидно, требует борьбы с теорией относительности. Преобразования Лоренца, столь важные для специальной теории относительности,, очевидно, могут быть бесконечно приближены посредством численных оценок. Тем не менее, крайне сложно имитировать в цифровом виде согласованную форму модели теории относительности. Наш физический опыт немедленно говорит нам, что не может быть доказана никакая превосходная система координат, и что мы оправданы в рассмотрении каждой системы координат столь же справедливой, как и любая другая, в этом случае преобразования Лоренца формулируют отношения между этими инерциальными системами. Строгая интерпретация специальной теории относительности приводит, однако, к заключению, что в действительности превосходная система координат не существует, и что бесполезно искать такую систему экспериментально. В любом представлении космоса как клеточных автоматов почти невозможно избежать допущения превосходной системы движения. Мы можем конструировать структуру клеточных автоматов таким образом, чтобы было доступно большее число, хотя все еще конечное количество, превосходных систем координат. Постоянство скорости света во всех инерциальных системах представлено посредством цифровой имитации преобразований Лоренца и связанного с этим сокращения тел.

В любом случае, соотношение между скоростью света и скоростью передачи между отдельными ячейками клеточного автомата должно следовать из такой модели. Они не обязательно должны быть идентичны. Напротив, можно предположить, что скорость передачи от ячейки к ячейке должна быть больше
