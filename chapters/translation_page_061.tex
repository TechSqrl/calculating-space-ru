% Перевод страницы 61 из "Rechnender Raum" Конрада Цузе
% Послесловие: страница 61

\section*{Послесловие к <<Rechnender Raum>> Конрада Цузе}

\noindent
\textit{Adrian German\textsuperscript{1} и Hector Zenil\textsuperscript{2}}

\noindent
\textsuperscript{1}Школа информатики и вычислений,\\
Университет Индианы, Блумингтон, США

\noindent
\textsuperscript{2}Отделение компьютерных наук,\\
Университет Шеффилда, Великобритания

\vspace{0.3cm}

Существует много параллелей между интересами Цузе и Тьюринга. В середине 1930-х годов некоторые исследователи были заняты тем, что по сути являлось исследованием природы вычисления и попыткой выяснить, было ли бы возможно построить вычислительную машину. Отчасти это было следствием программы Гильберта, но несомненно также было обусловлено определённой цепью исторических событий. Как указал Рауль Рохас\textsuperscript{1}, люди начали думать о компьютерах именно тогда, когда настало время их строить. Конечно, были Шёнфинкель (SKI комбинаторы), Чёрч ($\lambda$-исчисление), Пост (системы тегов), Клини (рекурсивные функции), Тьюринг (a-машины), и некоторые другие.

Возможно, основное различие между всеми остальными подходами и подходом Цузе заключается в том, что Цузе был инженером-строителем, стремящимся решать конкретные проблемы, и поэтому его подход был по сути чисто практическим. Таким образом, цель Цузе с самого начала была в построении конкретной, механической реализации вычисления. Подход Тьюринга находился на полпути между чистой абстракцией и практической реализацией. Этот факт один может объяснить, почему работа Тьюринга в итоге была более видна, чем работы других. Подход Цузе, будучи ответом инженера на вопрос о природе вычисления, принял форму действительной машины\textsuperscript{2}.

Цузе, возможно, не понимал, что существует фундаментальная концепция, лежащая в основе вопроса, который все эти люди задавали и в итоге пытались ответить (Цузе работал в относительной изоляции, в отличие от других, которые в основном знали друг о друге). Тьюринг в конечном счёте предоставил наиболее близкий ответ на вопрос через свою концепцию универсальности вычисления, основополагающее понятие информатики. Парадоксально, но сегодняшние цифровые компьютеры в некоторых отношениях могут быть более подобны машинам Цузе, чем идеализации Тьюринга, особенно

{\footnotesize
\noindent
\textsuperscript{1}На недавнем докладе \textit{Zuse and Turing in Context} в Кембридже, Великобритания, 18 февраля 2012 года.

\noindent
\textsuperscript{2}Самый полный источник информации~--- Интернет-архив Конрада Цузе, хранимый Раулем Рохасом, доступный онлайн по адресу \url{http://www.zib.de/zuse/home.php} (доступ апрель 2012). Его сын, Хорст Цузе, ведёт домашнюю страницу своего отца по адресу \url{http://www.horst-zuse.homepage.t-online.de/konrad-zuse.html} (доступ апрель 2012). Юрген Шмидхубер также ведёт веб-сайт, посвящённый Цузе, доступный по адресу \url{http://www.idsia.ch/~juergen/zuse.html} (доступ апрель 2012).
}
