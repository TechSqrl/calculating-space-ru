% Страница 16 - продолжение раздела 2.4

\begin{wrapfigure}{r}{0.3\textwidth}
  \includegraphics[width=\linewidth]{images/page_016_img_01.png}
  \caption{Распределение поля между двумя противоположными зарядами}
  \label{fig:11}
\end{wrapfigure}


Мы можем представить функциональную природу этого бокового расширения
следующим образом: предположим, что мы хотим вычислить поле между двумя
противоположными зарядами $+e$ и $-e$, допустим, что мы не знаем распределение
поля само по себе, хорошо известное и также легко выводимое. Мы начинаем,
как показано на Рис. 11, с распределения, заведомо являющегося ложным,
просто соединяя $+e$ и $-e$ линейно-постоянной силой от начала до конца.

Применение уравнений Максвелла к этому распределению поля приводит к
многошаговому асимптотическому приближению к определяемому полю.

В этом процессе также демонстрируется, что мы получаем результаты без
использования уравнения
\[
-\text{div} \, \mathbf{E} \Rightarrow \frac{\partial \gamma}{\partial t}
\]
при рассмотрении электромагнитных полей, хотя, как мы видели, это уравнение
необходимо для рассмотрения сжимаемых жидкостей. Нам даже не нужно вводить
плотность электрического поля $\gamma$. Тот факт, что результаты получаются
без этого члена, не является доказательством того, что природа работает без
обращения к плотности поля. Предполагая, однако, что такое условие
существовало, тем не менее было бы почти невозможно продемонстрировать его
существование, поскольку оба уравнения «ротора» устанавливают сами по себе
такое распределение поля, что
\[
\text{div} \, \mathbf{E} = 0
\]
удовлетворяется в общем случае. В результате дивергенция не вносит вклада
в распределение поля. Поскольку невозможно создавать или уничтожать заряды,
у нас нет экспериментальных средств для проверки справедливости закона
продольного расширения в природе.

Каково же тогда обоснование для исследования этого закона? Вопрос
интересен в связи с концепцией численной устойчивости, и он будет
рассмотрен снова ниже.
