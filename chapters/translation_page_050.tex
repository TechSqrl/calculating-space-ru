% Страница 50: Завершение раздела 4.4 - Рассмотрения теории информации
% Конрад Цузе "Rechnender Raum"

фигурации).

Этот конфликт между требованием максимального содержания информации и одновременным требованием надежности передачи (минимум ошибок) затрагивает самую сердцевину теории информации.

Если физические правила должны быть предполагаемы детерминированными, то все процессы в физическом мире должны быть полностью определены начальными условиями. Это означает, что информационное содержание начального состояния кодирует информацию всех будущих состояний. В отношении информации, это является источником беспокойства. Это означает, что для вселенной, постигаемой целиком, информационное содержание начального состояния должно быть чрезвычайно велико.

На практике мы имеем только неполное знание начального состояния, и в результате процесс становится для нас вероятностным. Что здесь происходит, однако, это несколько философское по своей природе, и это касается границы, на которой детерминированная система становится вероятностной для нас, потому что нам не хватает информации. В области обработки информации в живых организмах, этот вопрос имеет значение, хотя не в контексте нашего настоящего рассмотрения.

Если мы полностью приняли концепцию, что космос может рассматриваться как клеточный автомат, то это имело бы следствия в отношении самого фундаментального отношения между энергией и материей. В инженерии информационной обработки, энергия используется для целей информационной обработки. Однако энергия может быть рассмотрена как носитель информации. Сам процесс передачи энергии может быть рассмотрен как процесс передачи информации. В этом свете материал и энергия представляют собой различные проявления одного основного явления - информации. Естественно, вещество кажется проявлением информационного содержания через механизмы изоляции и интеграции. Это является спекулятивным, безусловно, но такого рода спекуляция кажется оправданной в этом контексте.

Таким образом, результаты, которые автор попытался представить, вы видите как заключение, как попытку связать понятия современной физики с понятиями из области теории вычислений и автоматов. Концепция дискретизации представляет определённо новый подход к формулировке физических процессов. Какое практическое значение, если таковое имеется, могут иметь эти соображения, остаётся открытым вопросом. Однако по крайней мере можно указать, что соответствующие результаты показывают определённое логическое соответствие и не являются категорически исключёнными современными физическими теориями.
