% Страница 57 - Глава 4: Общие рассмотрения

Необходимо также помнить, что выбор алгоритма для создания значений псевдослучайности имеет высокое значение в случае (a). Это означает, что возможны только те выборы из диапазона основных числовых рядов, которые следуют друг за другом максимально нерегулярно и которые имеют максимально равномерно возможное распределение вероятности. Это означает, что более длинные ряды одного и того же числа и ряды чисел с одинаковым разделением (1, 2, 3) должны быть исключены, хотя эти ряды в реальных рядах значений случайности так же вероятны или маловероятны, как и любой другой числовой ряд.

Конечно, мы можем задать чисто спекулятивный вопрос, допустимы ли истинные вероятностные законы к теоретико-автоматному наблюдению физических процессов. Этот вопрос является философским и отмечен здесь только без ответа.

\subsection{Представление интенсивности}

Представление интенсивности напряженности полей и других численных величин в клеточных автоматах должно быть специально рассмотрено. По этой причине здесь рассматриваются несколько основных возможностей.

\begin{wrapfigure}{r}{0.3\textwidth}
  \includegraphics[width=\linewidth]{images/page_057_img_01.png}
  \caption{Представление интенсивности в двумерной сетке (Рис.~69)}
  \label{fig:69}
\end{wrapfigure}


На Рис.~\ref{fig:69} показана двумерная сетка, в которой отдельные узлы сетки заняты элементарными логическими значениями; например, значения да-нет. Если мы приписываем этим значениям числа 0 и 1, то статистическое распределение значений 1 представляет шкалу для напряженности поля. Такой тип представления может достичь немного, конечно, если необходимо учитывать много порядков величины плотности. Как уже упоминалось, отношение электростатических взаимодействий к гравитационным взаимодействиям находится в порядке 10^40 : 1. Если бы мы захотели представить эти различия интенсивности в трёхмерном пространстве, соответствующему Рис.~\ref{fig:69}, используя значения да-нет, потребовался бы куб с длиной стороны приблизительно 10^13 сеточных единиц. Это представляет только нижний предел, так как в действительности напряженности полей могут отличаться на ещё большие порядки величины. Если мы возьмём сетку с элементарной длиной 10^{-13} см, принятой физиками, это означало бы, что согласно этим расчётам потребовалось бы пространство в несколько кубических сантиметров, чтобы представить напряженность поля.

Этот тип модели не может быть очень полезным, совершенно независимо от того, что чрезвычайно трудно установить законы для стабильных цифровых частиц с таким типом статистического распределения.
