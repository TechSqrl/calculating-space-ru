% Страница 29 - Конрад Цузе "Вычисляющее пространство"

\begin{wrapfigure}{r}{0.30\textwidth}
    \centering
    \includegraphics[width=0.28\textwidth]{images/page_029_img_01.png}
    \caption{Выдержка из системы расчётов с представлением процесса взаимодействия двух частиц}
    \label{fig:26}
\end{wrapfigure}

На рис. 26 показан процесс взаимодействия двух таких частиц. Рисунок демонстрирует, что частицы не просто проходят мимо друг друга, а что они реагируют, на этот раз с сокращением времени взаимодействия (в отличие от рис. 22). Процесс также может быть представлен как отталкивание. Здесь можно видеть в режиме просмотра рисунков, что такие термины, как "прохождение сквозь" и "отталкивание", теряют значение, когда применяются к реакции цифровых частиц. Квантовая теория дала соответствующие результаты, хотя и не в цифровой форме.

При взаимодействии частиц, соответствующем рис. 26, конечно, существует значительно больше различаемых случаев, очевидных из систематического исследования, в сравнении с примером на рис. 23. Мы должны сначала исследовать, какие частицы возможны в этой системе. Влияние фаз разделения также должно быть учтено, и наконец возможности взаимодействия частиц в различных фазах должны быть рассмотрены.

Целью этого документа не является проведение исчерпывающего обследования. Предыдущее наблюдение нескольких простых примеров стимулирует целый ряд интересных концепций.