% Перевод страницы 68 из "Rechnender Raum" Конрада Цузе
% Послесловие: страница 68

\begin{wrapfigure}{r}{0.4\textwidth}
  \includegraphics[width=\linewidth]{images/page_068_img_01.png}
  \caption{(Как) Природа вычисляет? Панельная дискуссия (Рис.~76)}
  \label{fig:76}
\end{wrapfigure}


Панельная дискуссия, показанная на Рис.~\ref{fig:76}, была организована A. German и H. Zenil в последний день конференции 2008 NKS Midwest Conference, с участием (в порядке): Greg Chaitin, Ed Fredkin, Rob de Ruyter, Anthony Leggett, Cristian Calude, Tommaso Toffoli и Stephen Wolfram, модерировали (слева направо) Gerardo Ortiz, George Johnson и Hector Zenil, в Университете Индианы, Блумингтон.

См. \url{http://www.cs.indiana.edu/~dgerman/2008midwestNKSconference/}.

\vspace{0.3cm}

физикой и компьютерной наукой и/или компьютерным оборудованием была рассмотрена в деталях, физические возможности и пределы компьютерного оборудования всё ещё доминировали в обсуждениях. Более глубокий вопрос~--- в какой степени процессы в физике могут быть объяснены как компьютерные процессы~--- рассматривался только в маргинальной степени на этой иначе очень передовой конференции.
