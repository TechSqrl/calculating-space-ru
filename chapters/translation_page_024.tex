% Страница 24 - Конрад Цузе "Вычисляющее пространство"

Совершенно противоположное условие представляет интерес для нас. Хотя математики и программисты обычно пытаются установить разностные уравнения таким образом, чтобы дифференциальное уравнение, лежащее в основе разностного уравнения, приближалось как можно ближе, мы можем разрешить вопрос, используя наиболее общую дискретизацию.

Мы теперь способны преобразовать физический закон импульса в инженерный закон приборов. Если мы позволим величинам $p$ и $v$ и соответствующим значениям $\Delta p$ и $\Delta v$ принимать только целые значения, мы должны выбрать целое числовое значение $k$, чтобы разностное уравнение давало целые результаты. Если мы сначала примем $k = 1$, мы получаем уравнения:

\[
- \Delta_s p \Rightarrow \Delta_t v
\]

\[
- \Delta_s v \Rightarrow \Delta_t p
\]

Мы далее пытаемся отнести $p$ и $v$ наименьшим возможным значениям, то есть $-1$, 0 и $+1$, и изучить поведение системы, которая удовлетворяет этим условиям. Мы получаем в результате следующее арифметическое соотношение:

\[
v - \Delta_s p \Rightarrow v
\]

\[
p - \Delta_s v \Rightarrow p
\]

На рис. 15 показана простая схема вычислений для этого правила. Мы имеем четыре значения $v$, $-\Delta v$, $p$ и $-\Delta p$ за единицу времени. Пространственные сектора противоположны друг другу. Нулевые значения не записываются для простоты. Четыре стабильные элементарные формы представлены [(1), (2), (3) и (4)], которые мы будем считать взаимно независимыми "цифровыми частицами". Есть два временных интервала, $t_1$ и $t_2$, соответственно для значений $v$, $-\Delta v$, $p$ и $-\Delta p$; $v$ и $p$ предполагаются для $t_1$. Отсюда следует, что $-\Delta v$ и $-\Delta p$ соответствуют временному интервалу $t_2$, и, следуя приведённому выше уравнению, значения $v$ и $p$ соответствуют следующему временному интервалу $t_2$.

Уравнения относятся к распространению простого импульса. Частицы стабильны только при этой скорости. В то же время эта скорость является наивысшей возможной для системы. Система не допускает никаких других скоростей. На рис. 16 показана графическая версия этого импульса.

С точки зрения теории автоматов, мы имеем дело с линейно-расширенным бесконечным автоматом, который периодически повторяется в автомате (клеточный автомат). Значения $v$ и $p$ представляют состояния автомата; $\Delta v$ и $\Delta p$ выводятся из них. Приведённое выше уравнение устанавливает функцию, согласно которой последующее состояние вытекает из предыдущего.