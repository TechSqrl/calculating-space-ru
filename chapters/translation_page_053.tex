% Страница 53 - Глава 4: Общие рассмотрения

Это непрерывный переход в смысле теории автоматов должен быть отличен от мысли о непрерывном переходе между отдельными устойчивыми состояниями атома. Поскольку мы не способны анализировать такой процесс переходов экспериментально, все теории по этому вопросу относятся к области спекулятивного. В теоретико-автоматном смысле естественная цель состоит в создании моделей, которые позволяют отследить эти переходы поотдельности и позволяют объяснить излучение или поглощение фотонов в соответствующем процессе. Мы не можем предсказать, будет ли эта цель когда-либо достигнута. Часто высказываемое мнение о том, что такие переходы принципиально не поддаются анализу и такие эксперименты следует подчинить более плодотворным начинаниям, может, однако, быть опровергнуто. Квантовая физика предоставляет статистические законы для таких процессов, посредством которых отдельные детерминированные определения замещаются статистическими определениями. Этот вопрос будет рассмотрен дальше в связи с обсуждением вероятности.

Важно выяснить, действительна ли детерминированность в обоих направлениях времени; иными словами, являются ли более поздние состояния системы ясными функциями предыдущих состояний и наоборот. Классическая модель механики идеально удовлетворяет требованию симметрии относительно времени. Статистическая квантовая механика вводит идею вероятности и наблюдает отклонение от временной симметрии в увеличении энтропии. В общем, конечные автоматы следуют законам, определённым только в положительном направлении. Алгоритм устанавливает только то, какое состояние вытекает из данного, но не обратное. Возможно построить автоматы, в которых предыдущее состояние определяется следующим за ним, но это не обязательно подразумевает симметрию во времени. Рассмотрение компьютеров может это прояснить. Компьютер, при условии безупречной работы, детерминирован в положительном направлении времени. В общем, вычислительные процессы необратимы, что видно из рассмотрения основных операций, на которых основаны все высшие вычисления, и которые необратимы (например, a ∨b ⇒ c). Вычислитель является одним примером вычислительной машины, которая эффективно детерминирована в обоих направлениях, поскольку она считает вперёд в одном направлении времени и назад в другом, в той мере, в какой мы рассматриваем только таблицы состояний и не анализируем процессы поотдельности.

Различные характерные типы работы автономного автомата уже были рассмотрены в разделе 4.4 в связи с Рис.~68. Тип 68б соответствовал бы автомату, детерминированному в обоих направлениях, как упомянутый вычислитель.
