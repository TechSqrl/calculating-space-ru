% Страница 59 - Глава 4: Общие рассмотрения

Клеточные автоматы предоставляют элегантное решение, когда каждая ячейка содержит полную вычислительную систему, как символически представлено на Рис.~73. Эти отдельные вычислительные системы содержат как элементы обработки информации, так и элементы хранения информации.

\begin{wrapfigure}{r}{0.4\textwidth}
  \includegraphics[width=\linewidth]{images/page_059_img_01.png}
  \caption{Клеточный автомат с полной вычислительной системой в каждой ячейке}
  \label{fig:72}
\end{wrapfigure}


Сетевой автомат, представленный на Рис.~74, является дальнейшим развитием клеточного автомата, соответствующего Рис.~73. Отдельные ячейки здесь отвечают только за обработку информации. Ветвящиеся линии B соединяют отдельные ячейки и служат как для передачи информации, так и для хранения информации. Отдельные ячейки могут состоять из одноместных суммирующих блоков в соответствии с принципом серии, действительным для вычислительных машин. Предварительные исследования автора показали, что этот тип автомата весьма успешен, в частности при решении численных задач, а также при моделировании физических процессов. Более специальное рассмотрение будет предметом другой статьи.

\begin{wrapfigure}{l}{0.4\textwidth}
  \includegraphics[width=\linewidth]{images/page_059_img_02.png}
  \caption{Клеточный автомат с полной вычислительной системой}
  \label{fig:73}
\end{wrapfigure}


\begin{wrapfigure}{r}{0.4\textwidth}
  \includegraphics[width=\linewidth]{images/page_059_img_03.png}
  \caption{Сетевой автомат с элементами обработки информации}
  \label{fig:74}
\end{wrapfigure}

