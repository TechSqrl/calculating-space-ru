% Страница 15 - продолжение раздела 2.4

Оба уравнения, которые содержат дифференциальный оператор rot, могут быть
легко преобразованы в форму вывода:
\[
\mathbf{E} + c(\text{rot} \, \mathbf{H})dt \Rightarrow \mathbf{E}
\]
\[
\mathbf{H} - c(\text{rot} \, \mathbf{E})dt \Rightarrow \mathbf{H}
\]
(ротор $\mathbf{H}$ даёт приращение $\mathbf{E}$; ротор $\mathbf{E}$ даёт приращение $\mathbf{H}$).

Оба уравнения дивергенции, с другой стороны, не имеют формы вывода. Если
принять во внимание волновую область поля, мы получаем:
\[
\text{div} \, \mathbf{E} = 4\pi\rho
\]

Это уравнение недостаточно для алгоритмического описания закона
распространения волн. Являются ли поэтому уравнения Максвелла неполными?
Они используются для описания распространения поперечных, но не продольных
волн. Причина того, что уравнения Максвелла в их обычной форме достаточны
для описания всех процессов, происходящих в электромагнитных полях,
основывается на том факте, что в природе не существует растущих, вновь
появляющихся или исчезающих волн. Происходят только смещения заряда. При
такого рода смещении уравнения Максвелла достаточны для описания изменений
в полях, связанных со смещениями. Автор не смог найти точного математического
доказательства этого ни в одном тексте, но это должно предполагаться.
Интересное замечание в этом отношении находится в «Beckersauter» (страница 186),
где развивается поле для равномерно движущегося заряда. Это приводит,
что достаточно интересно, к эллиптической деформации ранее сферически-симметричного
поля. Эта деформация соответствует гипотезе сокращения Лоренца.
Возможно переформулировать утверждение, что «уравнения Максвелла инвариантны
относительно специальной теории относительности»: «В результате использования
природой трюка бокового расширения (ротор) в расширяющемся поле, система
специальной теории относительности логически обоснована».