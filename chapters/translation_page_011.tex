% Страница 11 - продолжение раздела 2.2

приводит, среди прочих последствий, к ограниченной точности. В отличие от
аналоговых компьютеров, точность цифровых компьютеров строго определена и
не подвержена каким-либо случайным влияниям.

Дальнейший вывод состоит в том, что никакой цифровой компьютер не может
точно воспроизвести результаты процессов, определённых арифметическими
аксиомами. Так, например, математическая формула
\[
\frac{a \cdot b}{a} = b
\]
имеет общую применимость с единственным исключением, что $a$ не может быть
равно 0. Не существует конечного автомата, способного воспроизвести этот
факт точно и в общем виде. Тем не менее, возможно, увеличивая количество
разрядов до и после десятичной точки, для цифрового компьютера бесконечно
близко приблизиться к законам арифметики.

Мы в области математики уже настолько привыкли к концепции бесконечности,
что принимаем её, не рассматривая, что каждый бесконечный член связан с
разложением в ряд или с предельным процессом («для каждого числа существует
следующее за ним»). Соотнося этот процесс с теорией автоматов, мы получаем
вместо статического, предопределённого, конечного автомата ряд автоматов,
которые построены согласно определённому плану и отличаются друг от друга
только числом разрядов. Дан план построения автомата с $n$ разрядами;
кроме того, имеются инструкции для преобразования $n$-разрядного автомата
в автомат с $n + 1$ разрядами. С помощью предельного процесса $\lim_{n \to \infty}$
с использованием разложения в ряд получается правило автомата для
арифметических операций.

Цифровой компьютер, благодаря своей особой способности обрабатывать не
только числа, но и общую информацию (в отличие от аналогового компьютера),
открыл совершенно новые области, обсуждаемые ниже более подробно. В общем,
все вычислительные задачи могут быть решены на цифровом компьютере, тогда
как аналоговые компьютеры лучше подходят для специальных задач. Необходимо
подчеркнуть, что цифровые компьютеры работают строго детерминированным
образом. Используя один и тот же алгоритм (т.е. ту же программу) и вводя
одни и те же входные значения, всегда должны быть получены одинаковые
результаты. Ограниченная точность всегда приводит к одной и той же степени
неточности в результатах, когда операция выполняется несколько раз с
одними и теми же входными данными. Это в отличие от аналогового компьютера,
в котором ограниченная точность имеет различный эффект каждый раз при
выполнении программы и может быть выражена только в терминах статистической
вероятности.

В качестве дополнительных замечаний можно отметить, что были разработаны
гибридные системы, которые состоят из смеси принципов цифрового и
аналогового компьютера.