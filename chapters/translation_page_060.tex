% Страница 60 - Глава 5: ЗАКЛЮЧЕНИЯ

\chapter{Заключения}

Хотя эти наблюдения не приводят к новым легко понимаемым решениям, тем не менее может быть продемонстрировано, что предложенные методы открыли несколько новых перспектив, достойных продолжения. Включение концепций информации и теории автоматов в физические наблюдения станет ещё более критичным, так как всё более будут использоваться целые числа, дискретные состояния и тому подобное.

В следующей таблице предпринимается попытка связать различные возможные концептуализации:

\begin{center}
\begin{tabular}{|l|l|l|}
\hline
\textbf{КЛАССИЧЕСКАЯ ФИЗИКА} & \textbf{КВАНТОВАЯ МЕХАНИКА} & \textbf{ВЫЧИСЛЯЮЩЕЕ ПРОСТРАНСТВО} \\
\hline
Механика точки & Волновая механика & Теория автоматов \\
\hline
Алгебра счётчика & Частицы & Волна-частица, состояние счётчика, цифровая частица \\
\hline
Аналоговая & Гибридная & Цифровая \\
\hline
Анализ & Дифференциальные уравнения & Разностные уравнения и логические операции \\
\hline
Все значения непрерывны & Ряд значений квантован & Все значения имеют только дискретные значения \\
\hline
Без предельных значений & За исключением скорости света, нет предельных значений & Минимальные и максимальные значения для каждой возможной величины \\
\hline
Бесконечно точны & Отношение вероятности & Ограничения точности вычисления \\
\hline
Причинность в обоих направлениях времени & Только статичная причинность, разделение на вероятности & Причинность только в положительном направлении времени; введение вероятностных членов возможно, но не необходимо \\
\hline
Классическая механика статистически аппроксимируется & Объяснимы ли пределы вероятности квантовой физики определёнными структурами вычисляющего пространства? & \\
\hline
Основаны на формулах & Основаны на счётчиках & \\
\hline
\end{tabular}
\end{center}

Ввиду перечисленных выше возможностей ясно, что есть несколько различных точек зрения возможны:

(1) <<Идеи вычисляющего пространства противоречат некоторым признанным концепциям современной физики (например, изотропии пространства); поэтому фундаментальный базис должен быть ложным>>.

(2) <<Законы вычисляющего пространства должны быть пересмотрены с целью устранения существующих противоречий>>.
