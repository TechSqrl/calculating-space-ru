% Страница 36: Продолжение раздела 3.3
% Конрад Цузе "Rechnender Raum"

дополнительные примеры, эти карманы не могут быть уничтожены.

Теперь у нас имеются частицы, которые могут распространяться в восьми дискретных направлениях в плоскости, а также стоячие карманы. На рис. 46--57 представлена серия интересных примеров взаимодействия таких частиц.

\begin{figure}[htbp]
    \centering
    \includegraphics[width=0.6\textwidth]{images/page_036_img_01.png}
    \caption*{Рис. 46}
\end{figure}

Сначала мы сохраняем условие, что стрелки могут иметь только значения $-$, $0$, $+$. Две противоположно направленные стрелки взаимно уничтожают друг друга в одной точке решетки, а две стрелки одной ориентации действуют как единая изолированная стрелка.

\begin{figure}[htbp]
    \centering
    \includegraphics[width=0.6\textwidth]{images/page_036_img_02.png}
    \caption*{Рис. 47}
\end{figure}

Видно, что ход различных взаимодействий зависит как от времени, так и от разности фаз. Частицы могут проходить друг через друга, взаимно уничтожаться или образовывать новые частицы. Карманы коварны, так как они могут уничтожать частицы, не исчезая сами. С другой стороны, карманы могут возникать из определенных форм взаимодействия (рис. 55 и 57). В модели космоса, функционирующей в соответствии с этим правилом, все частицы в конечном итоге были бы преобразованы в твердые карманы. Поэтому эта модель имеет малую практическую ценность.

\begin{figure}[htbp]
    \centering
    \includegraphics[width=0.6\textwidth]{images/page_036_img_03.png}
    \caption*{Рис. 48}
\end{figure}

При взаимодействии весьма существенно, лежит ли точка пересечения траекторий частиц в дискретно определенной точке системы координат. В этом случае происходит реакция (например, рис. 52 и 53).

Возможности этой системы могут быть исследованы путем введения стрелок различной абсолютной длины. Для стрелок, указывающих в одном направлении, мы используем правило сложения. Гораздо сложнее расширить правило рис. 42, включив в него две пересекающиеся стрелки различных длин. Мы можем принять следующее соглашение.
