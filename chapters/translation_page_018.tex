% Страница 18 - продолжение раздела 2.6 и начало 2.7

частных имеет смысл только если градации между значениями много меньше,
чем выбранное значение $\Delta$. Этот факт имеет определённое влияние на
численную устойчивость вычисления.

Если переходы выполняются таким образом, что значения остаются
приблизительно того же порядка величины, что и шаговые значения, сохраняется
ступенчатая форма кривой, и невозможно построить дифференциальное частное.

В последующих наблюдениях это расстояние будет использоваться намеренно,
конкретно через последовательное дальнейшее развитие мыслей о дискретизации.

Систематическое сужение числа разрядов рассматриваемых величин приводит к
ограничению переменных теми, которые охватываются элементарной логикой;
например, значения да-нет или троично-переменные значения. Как мы обнаружим
позже, тройные значения и троичная система счисления, основанная на этих
значениях, имеет определённые преимущества, поскольку округление вверх и
округление вниз легче выполнять, и деление на 6, необходимое при делении
области поля на 6 соседних ячеек, также легче вычислить. Присваивая
значения +1, 0 и -1 числам, это соответствует возможным электрическим
частицам +e, 0, -e.

Непрерывная плотность поля должна быть разделена на отдельные значения для
численного решения — процесс, который проще всего выполнить с помощью сетки.
Простейшей сеткой, несомненно, является ортогональная. Существуют другие
возможные выборы: треугольные и шестиугольные сетки в двух измерениях,
например, и сетка в трёх измерениях, соответствующая наиболее плотной
упаковке сфер.

Если в вычислении возникает несколько различных значений поля (например,
векторы скорости и плотности), не обязательно, чтобы эти значения были
локализованы в одной и той же точке сетки. Нет необходимости локализовать
три компоненты пространственного вектора. В этом случае также возможно
разделение. Нет дальнейшей необходимости в построении цифровой структуры
пространства приближать законы евклидова пространства. Ряд общих наблюдений
о представлении физических проблем был представлен ранее с точки зрения
теории автоматов.

\section{Наблюдения физических теорий с точки зрения теории автоматов}
\label{sec:automaton-observations}

До этого момента мы рассматривали только проблему использования компьютеров
для приближения физических моделей и численного отслеживания физических
процессов. В этом контексте было бы возможно предложить фундаментально
отличный вопрос: в какой степени реализации, полученные из изучения
вычислимых решений, полезны при применении непосредственно к физическим
моделям? Является ли природа
