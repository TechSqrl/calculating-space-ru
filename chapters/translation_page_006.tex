% Страница 6 - продолжение Главы 1

На последующих страницах будет развито несколько идей в этом направлении.
Мы не претендуем на полноту в рассмотрении предмета.

Такой процесс влияния может исходить из двух направлений:

\begin{enumerate}
\item Разработка и предоставление алгоритмических методов, которые могут
служить физику в качестве новых инструментов, с помощью которых он может
переводить свои теоретические знания в практические результаты. К ним
относятся прежде всего численные методы, которые всё ещё являются основным
инструментом при использовании электронных вычислительных машин. Идеи,
изложенные в последующих главах, могли бы особенно способствовать решению
проблемы численной устойчивости.

К ним относятся символьные вычисления, которые приобретают всё большее
значение сегодня. Под этим подразумевается не численный расчёт формулы,
а алгебраическая обработка самих формул в том виде, как они выражены
символами. Именно в квантовой механике обширная разработка формул
необходима прежде, чем может быть выполнен фактический численный расчёт.
Эта весьма интересная область не будет рассматриваться в последующем
материале.

\item Можно постулировать прямой процесс влияния, в частности, мыслительных
моделей теории автоматов на сами физические теории. Этот предмет, без
сомнения, более сложный, но также и более интересный.
\end{enumerate}

В этом заключается понятная трудность, состоящая в том, что различные области
знания должны быть приведены во взаимосвязь друг с другом. Уже сама область
физики разделяется на специализированные направления. Одни только математические
методы современной физики больше не знакомы каждому математику, и их понимание
требует многолетнего специального изучения.

Но даже теории и области знания, связанные с обработкой данных, уже разделяются
на различные специальные отрасли. В качестве примеров можно привести формальную
логику, теорию информации, теорию автоматов и теорию формальных языков. Идея
объединения этих областей (в той мере, в какой они релевантны) под термином
«кибернетика» ещё не получила широкого признания. Концепция кибернетики как
моста между науками весьма плодотворна, совершенно независимо от различных
определений самого термина.

Автор разработал несколько основных идей в этом направлении, которые он считает
ценными для представления на обсуждение. Некоторые из этих идей в их нынешней,
всё ещё незрелой форме могут быть несовместимы с проверенными концепциями
теоретической физики. Цель достигнута, если будет инициировано обсуждение