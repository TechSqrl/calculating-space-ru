% Страница 54 - Глава 4: Общие рассмотрения

Тем не менее, остаётся различие: в положительном направлении времени правило, согласно которому следующее состояние связано с предыдущим, явно дано алгоритмом. В отрицательном направлении времени существует единственная корреляция, конечно, но эта корреляция дана только неявно; то есть, она не может быть непосредственно вычислена без дополнительного знания. Это различие не ясно видно в диаграммах, соответствующих Рис.~68, и в таблице состояний, соответствующей Рис.~4. В любом случае, такой тип представления возможен только для очень простых автоматов и служит больше для начальных экспериментов, чем для практических определений процесса работы автомата. Фактическое правило образования следующего состояния из предыдущего дано схемами автомата. Мы можем сказать, что автономный автомат детерминирован в положительном направлении времени и что в специальных случаях отрицательного направления времени существует <<псевдодетерминированность>>.

Взаимоотношения цифровых частиц подобны в случаях, рассмотренных в Главе 4.4. Пока такая частица следует своему пути независимо от внешних влияний, происходит единственная последовательность состояний. Как только мы рассмотрим последовательность двух частиц, условия немедленно становятся другими. В этом случае примеры из Главы 3, Рис.~42-66 относятся к необратимым процессам. Основное правило сдвига регулирует процессы взаимодействия частиц. Нет никакого стимула для частицы разделиться на две частицы в какой-то момент времени. Это утверждение делает только одно утверждение о моделях, использованных в Главе 3. Вопрос о том, возможно ли построить пригодные модели цифровых частиц, которые не имели бы такой характеристики, трудно ответить. Это то же самое проблем, с которой сталкивается физик при распаде элементарных частиц или атомных ядер. Современное состояние теоретической физики таково, что мы можем дать только вероятностные законы для таких процессов. В модели, которая следует заранее определённому процессу и исключает рабочие элементы, в соответствии с вероятностными законами, есть только два способа решения:

(a) цифровая модель построена таким образом, что она содержит своего рода часы, которые разрешают процесс, когда достигнуто определённое состояние;

(b) влияние окружающей среды (например, полей, через которые движется цифровая частица) принимается во внимание. В процессе прохождения через свои различные фазы частица может пройти через критические состояния, в которых влияние окружающей среды (частота и т. д.) вызывает разделение частицы.

Современное состояние физических теорий не позволяет делать выводы о физических законах из этих возможностей цифровых моделей. То, что уже было сказано о переходе из одного атомного состояния в другое, одинаково актуально и здесь: ни один эксперимент не позволяет заглянуть за кулисы, и все теории по сути дела спекулятивны.
