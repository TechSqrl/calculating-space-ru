% Страница 14 - продолжение 2.3 и начало 2.4

точки зрения теории автоматов может быть названо формой «вывода», поскольку
следующее состояние возникает из данного состояния через действие
дифференциала на данное состояние. В случае жидкостей и газов включение
члена сжатия сначала приводит к этой форме вывода. Состояние системы
задано распределением давления и скорости. Разности в давлении приводят
к силам, ведущим к новому распределению скоростей, которое само ведёт к
новому распределению плотности и, следовательно, давления через движение
масс. «Состояние» поля может быть описано, следовательно, скалярным полем
плотности $\gamma$ и полем скоростей $\mathbf{v}$. Уравнение может быть
выражено в форме вывода следующим образом:
\[
k \cdot \text{grad} \, \gamma \Rightarrow \frac{\partial \mathbf{v}}{\partial t}
\]
\[
-\text{div} \, \mathbf{v} \Rightarrow \frac{\partial \gamma}{\partial t}
\]
($k$ — фактор, который определяется физическими условиями). Алгоритмический
характер ещё более ясно выражен в следующей форме:
\[
\mathbf{v} + k(\text{grad} \, \gamma)dt \Rightarrow \mathbf{v}
\]
\[
\gamma - (\text{div} \, \mathbf{v})dt \Rightarrow \gamma
\]

В соответствии с обычными правилами языка программирования (алгоритмического
языка), одинаковые символы по обе стороны знака вывода относятся к различным
последовательным состояниям системы ($\mathbf{v}$, $\gamma$).

В случае несжимаемых жидкостей существует условие $\text{div} \, \gamma = 0$.
Это уравнение не имеет алгоритмического характера и в результате не может
быть преобразовано в форму вывода. Оно представляет лишь одно условие для
правильности решения, полученного другими средствами.

\section{Уравнения Максвелла}
\label{sec:maxwell-equations}

Уравнения Максвелла также могут быть изучены с этой точки зрения. Мы
ограничимся теми уравнениями, которые описывают распространение поля
в вакууме:
\[
\text{rot} \, \mathbf{H} = \frac{1}{c} \frac{\partial \mathbf{E}}{\partial t}
\]
\[
\text{div} \, \mathbf{E} = 0
\]
\[
\text{rot} \, \mathbf{E} = -\frac{1}{c} \frac{\partial \mathbf{H}}{\partial t}
\]
\[
\text{div} \, \mathbf{H} = 0
\]