% Страница 25 - Конрад Цузе "Вычисляющее пространство"

\begin{figure}[htbp]
\centering
\includegraphics[width=0.6\textwidth]{images/page_025_img_01.png}
\caption{Расширение импульса давления}
\label{fig:15}
\end{figure}

\begin{figure}[htbp]
\centering
\includegraphics[width=0.6\textwidth]{images/page_025_img_02.png}
\caption{Расширение импульса с учётом скорости}
\label{fig:16}
\end{figure}

\begin{figure}[htbp]
\centering
\includegraphics[width=0.6\textwidth]{images/page_025_img_03.png}
\caption{Неустойчивая форма расширения импульса}
\label{fig:17}
\end{figure}

На рисунках 17 и 18 показана неустойчивая форма расширения изолированного импульса давления, с которым не связан никакой импульс скорости (как это было в рисунках 15 и 16). На рисунке 17 значения $\Delta$ опущены по причинам обобщения.

\begin{figure}[htbp]
\centering
\includegraphics[width=0.6\textwidth]{images/page_025_img_04.png}
\caption{Форма расширения импульса с сохранением импульса}
\label{fig:18}
\end{figure}

Эта форма расширения импульса противоречит нашему пониманию расширения первоначально изолированной ячейки давления в газонаполненном цилиндре. Из этой модели мы вывели разностное уравнение. Дискретизация была проведена столь общим образом, что отклонения от дифференциального уравнения приводят к отклонениям от физических законов. Сохранение импульса, а не энергии, является ключом к вычислениям, лежащим в основе разностного уравнения. Графическое представление рис. 18 показывает, что среднее $(p = 1)$ остаётся постоянным, а среднее значение $v$ постоянно равно 0. С другой стороны, расширение чередующихся положительных и отрицательных $p$-значений в графическом представлении указывает на очевидное постоянное увеличение потенциальной энергии. Соответствующее верно для значений кинетической энергии, представленных значениями $v$.

Было бы интересно в этой точке выяснить, является ли такое отклонение обязательно связано с грубой дискретизацией или можно ли построить грубые цифровые модели, которые подчиняются всем условиям исходного дифференциального уравнения, в данном случае особенно сохранению энергии. Конечно, такая упрощённая модель требует точного определения термина <<энергия>>. Это просто отмечается без дальнейшего рассмотрения здесь.

Интересно то, что пара изолированных импульсов даёт устойчивую систему: испускание двух расходящихся цифровых частиц. Очевидно, только определённые конфигурации возможны, тогда как другие исключены или не обеспечивают стабильные результаты. Это имеет определённое сходство с некоторыми ситуациями в квантовой механике.

Так как наше выбранное вычислительное правило имеет чисто аддитивный характер, применяется принцип суперпозиции; то есть отдельные формы могут рассматриваться независимо друг от друга, в результате чего естественно, что появляются значения больше 1.
