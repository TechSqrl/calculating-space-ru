% Перевод страницы 67 из "Rechnender Raum" Конрада Цузе
% Послесловие: страница 67

\begin{quote}
физикой и информатикой и/или компьютерным оборудованием рассматривалась в деталях, физические возможности и пределы компьютерного оборудования всё ещё доминировали в обсуждениях. Более глубокий вопрос о том, в какой степени процессы в физике могут быть объяснены как компьютерные процессы, рассматривался только в незначительной степени на этой в остальном очень передовой конференции.>>
\end{quote}

Оригинал Rechnender Raum, похоже, был потерян. Насколько нам известно, перевод, заказанный Project MAC (предшественником нынешней MIT Computer Science and Artificial Intelligence Laboratory или CSAIL), никогда не был опубликован в журнале\textsuperscript{11}. Он воспроизведён здесь переведённым на современный \LaTeX, что потребовало значительной работы, несмотря на использование сначала методик OCR с Mathematica, чтобы избежать полного начала с нуля. Он публикуется в этом томе без изменений, за исключением, возможно, нескольких исправленных опечаток и перераспределения текста и изображений в соответствии с форматом книги. Материал одновременно устарелый и удивительно современный: <<Я предлагаю, что в информационно-теоретическом анализе объекты и элементарные размерности физики не должны быть дополнены концепцией информации, а скорее должны быть объяснены ею>>. Цузе всегда был осведомлён о гипотетической природе своего тезиса: <<Концепция вычисляющей вселенной всё ещё лишь гипотеза; ничего не было доказано. Однако я уверен, что эта идея может помочь раскрыть тайны природы>>.

Цузе ссылается на людей, более скептически настроенных к нам, на цитату из Freeman Dyson (<<Innovation in Physics>>, опубликовано в \textit{Scientific American}, Vol. 199, No. 3, (сентябрь 1958), pp. 74--82.): <<Несколько месяцев назад Werner Heisenberg и Wolfgang Pauli верили, что они сделали существенный шаг вперёд в направлении теории элементарных частиц. Pauli проходил через Нью-Йорк, и его убедили прочитать лекцию, объясняющую новые идеи аудитории, которая включала Niels Bohr. Pauli говорил час, а затем была общая дискуссия, во время которой его довольно резко критиковало младшее поколение. Наконец, Bohr был приглашен произнести речь, подытоживающую аргумент. <<Мы все согласны>>, сказал он, <<что ваша теория безумна. Вопрос, который разделяет нас, заключается в том, безумна ли она настолько, чтобы иметь шанс быть правильной. Моё собственное чувство состоит в том, что она недостаточно безумна.>>'

<<Воображение>>, говорил Цузе, <<это ключ ко всему прогрессу>>.

\vspace{0.3cm}

A. German и H. Zenil

Bloomington, IN, США и Sheffield, Великобритания

{\footnotesize
\noindent
\textsuperscript{11}Отсканированные копии краткой немецкой версии и перевода на английский, сопровождаемые дополнительным контекстным материалом, доступны онлайн на веб-сайте Schmidhuber <<Zuse's thesis>> по адресу \url{http://www.idsia.ch/~juergen/digitalphysics.html}. Немецкая версия также находится по адресу \url{http://www.zib.de/zuse/Inhalt/Texte/Chrono/60er/Pdf/76scan.pdf} (ссылки доступны апрель 2012)
}
