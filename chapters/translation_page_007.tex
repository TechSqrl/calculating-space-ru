% Страница 7 - окончание Введения и начало Главы 2

и провоцирует стимулирование, которое однажды приведёт к решениям, приемлемым
также и для физиков.

Метод, применяемый ниже, в настоящее время всё ещё носит эвристический характер.
Автор считает, что условия ещё не созрели для формулировки точной теоретической
системы. Прежде всего, в Главе 2 существующие математические и физические модели
будут рассмотрены с точки зрения теории автоматов. В Главе 3 представлено
несколько примеров цифровых моделей и вводится выражение «цифровая частица».
В Главе 4 будет развито несколько общих мыслей и соображений, основанных на
результатах Глав 2 и 3, а в Главе 5 кратко рассматриваются перспективы
возможности дальнейших разработок.

\chapter{Вводные наблюдения}
\label{ch:introductory-observations}

\section{О теории автоматов}
\label{sec:automaton-theory}

Теория автоматов сегодня уже является широко развитой и в определённой степени
весьма абстрактной теорией, о которой написана обширная литература. Тем не менее,
автор хотел бы провести различие между собственно теорией автоматов и мыслительными
моделями, связанными с этой теорией, которые будут широко использоваться в
последующих главах. Глубокое понимание теории автоматов не является необходимым
для понимания последующих глав.

Теория автоматов появилась примерно одновременно с развитием современного
оборудования для обработки данных. Конструкция и принцип работы этих устройств
потребовали теоретических исследований, основанных на различных математических
методах, например, на методах математической логики. Первым полезным результатом
этого развития стала математика соединений, в которой особенно важную роль может
играть исчисление высказываний математической логики. Особое значение имеет
осознание того, что вся информация может быть разложена на значения да-нет (биты).
«Истинностные значения» исчисления высказываний принимают только две оценки
(истина и ложь). Поэтому связующие операции и правила исчисления высказываний
можно рассматривать как элементарные операции обработки информации. На Рис. 2
показаны элементарные соединения, соответствующие трём основным операциям
исчисления высказываний: конъюнкции, дизъюнкции и отрицанию.

Дальнейшие исследования привели к введению термина «состояние» автомата.
Кроме того, роль играют входные данные и выходные данные. Из входных данных
и начального состояния получаются новое состояние и выходные данные в соответствии
с алгоритмом, встроенным в автомат. На Рис. 3 показана принципиальная схема