% Перевод страницы 65 из "Rechnender Raum" Конрада Цузе
% Послесловие: страница 65

Когда Z3 стал работоспособным, это была первая в мире практическая автоматическая вычислительная машина, и в течение двух лет она оставалась единственной. Вторая машина, Z4, была быстро введена в эксплуатацию. Во время войны Z3 была продемонстрирована перед несколькими ведомствами, но она так и не была введена в повседневную эксплуатацию. В 1944 году Z3 была уничтожена в авиационном налёте, но была реконструирована в 1960 году и установлена в Deutsches Museum в Мюнхене.

Цузе и Schreyer, однако, должны были покинуть здание, где была расположена их компьютерная машина. По мере окончания войны Цузе отступил в Hinterstein, деревню на юго-востоке Германии, где родился его старший сын (Horst). Там он восстановил свой компьютер Z4 в сарае, и он стал первой в мире коммерческой операционной вычислительной машиной, сданной в аренду ETH Zürich (одному из двух университетов Swiss Federal Institutes of Technology).

Затем он начал работать в области, которая не требовала физических ресурсов~--- программирование компьютеров. Он разработал язык~--- Plankalkül (означающий <<формальная система планирования>> или <<исчисление программ>>; <<универсальный язык>> согласно Цузе, который сравнивал его с <<искусственным мозгом>>)~--- который предвосхитил некоторые концепции программирования, которые появились позже, и может быть рассмотрен как первый язык высокого уровня, хотя для него никогда не был написан ни компилятор, ни интерпретатор. В 1945 году, возможно с той же мотивацией, которая привела Тьюринга к шахматам, а именно тем фактом, что игра считалась олицетворением человеческого интеллекта и в то же время казалась весьма алгоритмичной, Цузе работал над алгоритмами игры в шахматы, сформулированными как подпрограммы на его Plankalkül. Годом ранее, в 1944 году, он организовал свою работу в диссертацию\textsuperscript{9}, которая никогда не была защищена формально. Название, которое он выбрал для своей работы, было <<Beginnings of a Theory of General Computing>> (<<Начала теории общего вычисления>>), пытаясь установить основания того, что сегодня обычно понимается как обработка информации: <<Вычисление (Rechnen)>>, писал он, <<означает в общем случае формирование новых данных из заданных данных согласно некоторому правилу>>. Концепция алгоритма позже заменила его концепцию Vorschrift (или правила). Его язык программирования, как и логика, проектирование и построение его вычислительных машин, были полностью его собственной работой, выполненной в изоляции от развития, происходившего в других местах.

{\footnotesize
\noindent
\textsuperscript{9}См. <<The Plankalkül of Konrad Zuse -- Revisited>> Friedrich L. Bauer в <<The First Computers -- History and Architecture>>, упомянуто ранее.
}
