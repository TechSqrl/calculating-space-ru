% Страница 33 - Конрад Цузе "Вычисляющее пространство"

Две оси связаны через $p$. Теперь мы можем сформулировать следующее правило:

\[
v_x - \Delta p_x \Rightarrow v_x
\]

\[
v_y - \Delta p_y \Rightarrow v_y
\]

\[
p - (\Delta v_x + \Delta v_y) \Rightarrow p
\]

\begin{figure}[htbp]
    \centering
    \includegraphics[width=0.6\textwidth]{images/page_033_img_01.png}
    \caption{Размещение значений $v$ и $p$ в шахматной решётке (Рис.~34)}
    \label{fig:34}
\end{figure}

\begin{figure}[htbp]
    \centering
    \includegraphics[width=0.6\textwidth]{images/page_033_img_02.png}
    \caption{Получающиеся значения компонентов скорости $v_x$, $v_y$ и давления $p$ (Рис.~35)}
    \label{fig:35}
\end{figure}

\begin{figure}[htbp]
    \centering
    \includegraphics[width=0.6\textwidth]{images/page_033_img_03.png}
    \caption{Волновой фронт параллельно одной из осей координат}
    \label{fig:36}
\end{figure}

\begin{figure}[htbp]
    \centering
    \includegraphics[width=0.6\textwidth]{images/page_033_img_04.png}
    \caption{Волна, движущаяся по диагонали}
    \label{fig:37}
\end{figure}

Из-за связи через $p$ отдельные импульсы, соответствующие рисункам 15 и 16, исчезают. Стабильные, хотя и не бесконечно параллельные волновые фронты могут быть построены. На рис. 36 показан такой волновой фронт параллельно одной из осей координат, а рис. 37 показывает волну, движущуюся по диагонали. Скорости распространения являются функциями направления.

Было бы интересно рассмотреть различные следствия более или менее грубой дискретизации в этом случае. Так как правила связаны с уравнениями разреженной газовой динамики и гидродинамики, интересно, можно ли (например) гидродинамически стабильную структуру вихря грубо дискретизировать и можно ли построить "цифровые элементы". Это исследование может быть проведено только с помощью вычислительных машин.

Чтобы построить стабильные частицы в двумерном пространстве, мы сначала рассмотрим другой способ.

\section{Цифровые частицы в двумерном пространстве}
\label{sec:2d-particles}

Предположим ортогональную сетчатую структуру, соответствующую рис. 38. Мы больше не делаем различие между $v$-точками и $p$-точками, но позволяем каждой точке значения $p_x$, $p_y$. Для простоты сначала предположим, что значения $p$ могут принимать значения $-$, 0, $+$. Мы можем тогда говорить о $p$-стрелках или о коротких стрелках. Сначала мы устанавливаем, что изолированная стрелка (стрелка, у которой нет перпендикулярной стрелки, возникающей в той же точке сетки) непосредственно передаётся в следующую точку сетки. Четыре возможных примера такого одиночного изолированного импульса приводятся ниже. Он может быть передан вперёд только в ортогональном направлении.

Мы можем сначала определить, что есть два случая взаимодействия между двумя стрелками, приближающимися в одной и той же ортогонали. Оба они описаны ниже. В одном случае стрелки продолжают удаляться друг от друга; в другом они сокращаются. Какой случай происходит, зависит от фазы разделения. Нам ещё нужно правило для случая пересекающихся стрелок. Две пересекающиеся стрелки существуют в точке Z в момент времени I. Согласно нашим предыдущим правилам, они были бы распространены вперёд, каждая в своём собственном направлении, независимо от другой.
