% Страница 19 - продолжение раздела 2.7

цифровой, аналоговой или гибридной? И есть ли по существу какое-либо
оправдание для постановки такого вопроса?

Классические модели физики, несомненно, аналоговы по природе. Напряжённость
поля различных потенциалов, как и сила гравитации, не подчинены
«партикуляризации». Не существует таких пределов, как «пороговые значения»
(минимальный размер), предельные значения (максимальные значения) или пределы
плотности самого поля. Даже расширение классических законов теорией
относительности полностью находится в рамках концепции континуума. Только
для скорости предполагается существование абсолютного верхнего предела
(скорости света), и эта концепция полностью согласуется с «аналоговым»
мышлением.

Только с введением особой природы материи через её подразделение на молекулы,
атомы и элементарные частицы несколько величин приобрели дискретный характер,
но это не обязательно должно приравниваться к «цифровой» интерпретации
законов природы. Классическая проблема многих тел была аналоговой по природе,
даже когда каждое из отдельных тел обладало индивидуальными характеристиками
с дискретными свойствами (массами).

Квантовая физика первой отклоняется в нескольких отношениях от концепции
бесконечных величин, в той мере, в какой она предполагает только дискретные
значения для определённых физических величин. Наиболее известно соотношение
между частотой и энергией светового кванта, которое определяется формулой
$E = h \cdot \nu$, где $h$ — универсальная константа природы. Разумеется,
сама энергия не квантована, а только частное $E/\nu$. Это несколько
отличается от случая, когда энергия может иметь только дискретное число
значений из-за ограниченного числа разрядов в вычислителе цифрового
компьютера.

Постулаты квантовой теории имеют далеко идущие последствия в отношении
квантования различных физических величин. Концепции классического
пространственного континуума действительно отвергаются, но не через замену
континуума сеткой дискретных значений, а скорее через процесс, посредством
которого переходят к фундаментально отличным исходным точкам, подобным
конфигурационному пространству более высоких измерений, в котором определены
вероятностные значения (например, вероятность того, что частица находится
в определённом месте в определённое время). Даже в этой концепции идея
континуума не отвергается, поскольку дифференциальные уравнения квантовой
механики не управляются никакими ограничениями в отношении величин полей.

Модели современной физики, следовательно, имеют дело как с непрерывными,
так и с дискретными значениями. Казалось бы уместным рассмотреть гибридную
систему. Будет чрезвычайно трудно найти техническую модель гибридного
компьютера, которая ведёт себя согласно законам квантовой физики.

Мы признали предварительный вывод, что наши физические модели