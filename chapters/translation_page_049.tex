% Страница 49: Продолжение раздела 4.4 - Рассмотрения теории информации
% Конрад Цузе "Rechnender Raum"

мест в импульсной стрелке новой результирующей частицы должно быть больше, чем число мест в реагирующей частице. Если мы исключим длину стрелки 0 для простоты и предположим, что стрелка реагирующей частицы может быть представлена двоичными разрядами, то стрелка результирующей частицы должна быть представлена 4 двоичными разрядами. До реакции у нас есть 2 частицы, каждая из которых имеет содержание информации $2 \times 3$ бит (всего 12 бит). После реакции у нас есть частица с информационной емкостью только $2 \times 4 = 8$ бит. Во время реакции мы потеряли 4 бита информации.

В этом процессе мы позволили стрелке результирующей частицы быть представленной большим числом мест. Это уже само по себе означает допущение нового типа частицы. Если это не допускается, должно быть найдено правило, которое вступает в действие всякий раз, когда допустимое число мест превышается в процессе сложения. Если мы просто предположим, что максимальное значение не может быть превышено, то последовательные реакции приводят после определенного периода времени к результату того, что нам остаются частицы с абсолютными максимальными импульсными стрелками.

Выбранные здесь примеры для цифровых частиц все еще гораздо слишком просты, чтобы быть строго связанными с физическими процессами. На самом деле, мы никогда не сталкиваемся в природе с ситуацией, в которой частицы одного типа реагируют друг с другом, не говоря уже о результате, что две такие частицы реагируют, чтобы дать частицу более высокого типа. Сохранение энергии, импульса, спина, заряда и т.д. сохраняется для элементарных частиц в физике. Только когда модели цифровых частиц находятся в нашем распоряжении, с помощью которых члены могут быть представлены, становятся возможны сравнительные наблюдения с элементарными частицами в физике и их реакциями.

Это вопрос очевидного интереса, является ли сохранение различных величин, указанных в соответственно сконструированных цифровых частицах, связанным с соответствующим сохранением информации. Проблема становится еще более сложной, когда также рассматриваются поля.

Автор может только поставить вопрос без предложения ответа на него. Может быть, вопрос не столь ужасно важен. Так или иначе, вопрос сводится к проблеме «конфигурации», которая, как известно, крайне трудна для математической обработки.

Здесь мы приходим в прямой контакт с одной из трудностей теории информации. При передаче новостей, наибольшее возможное содержание информации получается, когда вероятность отдельных сигналов распределена как можно более равномерно. Эта ситуация упоминается как максимальная энтропия информации. Легко возможно рассмотреть это таким образом, что каждая возможность соотнесения ранее полученных новостей со следующим символом должна с необходимостью уменьшить содержание информации, что ограничивает посредством связанной избыточности свободу выбора символов (новостей, кон-
