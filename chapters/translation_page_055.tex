% Страница 55 - Глава 4: Общие рассмотрения

Однако удалось определить определённую зависимость радиоактивности от высоких температур, которая соответствует предположению о критических ситуациях, находящихся под влиянием окружающей среды.

Один результат важен в любом случае: предположение о действительности детерминизма только в положительном направлении времени никоим образом не влияется растворением физических законов в законы вероятности. Аналогично, увеличение энтропии не обязательно связано с этим вопросом. С точки зрения теории автоматов, каждый из этих вопросов принимает другой смысл. Энтропия может быть объяснена в цифровой модели, функция которой строго определена.

Рассмотрим классическую модель физики с этой точки зрения. Как уже упоминалось, действительность детерминизма, особенно в обоих направлениях времени, требует абсолютной точности отдельных процессов. Едва ли можно предположить, что серьёзные рассмотрения крайне важного значения этого допущения в отношении теории информации были предприняты. Такая модель требует бесконечно тонкой структуры пространственно-временных отношений. Бесконечное содержание информации требуется для неограниченного пространственно-временного элемента. Практически невозможно моделировать такую модель на компьютерах из-за необходимости бесконечного числа позиций. Источники ошибок соответственно велики из-за чрезвычайно большого числа столкновений между молекулами газа, и эти ошибки быстро приводят к отклонениям от теоретических процессов. Это означает, что чем лучше приближается правило причинности в обратном направлении времени, тем больше вычислений мы должны быть готовы произвести в нашей модели. Это приводит к результату, что моделирования универсальных систем с причинностью, функционирующей в обоих направлениях времени, относятся к категории <<неразрешимых>> проблем.

Конечно, можно сказать, что это справедливо только для вычислительных моделей моделирования. Но этот результат должен побудить нас пересмотреть этот вопрос. Имеем ли мы право предполагать модель природы, для которой не существует вычислимого моделирования?

С этой точки зрения представляется, что часто выдвигаемый аргумент о детерминизме в обоих направлениях времени должен быть фундаментально переэкзаменирован. Вопрос о временной симметрии физических законов часто обсуждается в связи с отражающими характеристиками пространства. Наблюдения из теории автоматов могут иметь значительную ценность в продвижении этого обсуждения.
