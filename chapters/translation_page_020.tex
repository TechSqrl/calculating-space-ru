% Страница 20 - продолжение раздела 2.7

лучше всего могут быть представлены как гибридные системы. Можно ли из этого
сделать выводы относительно природы? Следует ли поэтому саму природу
рассматривать как гибридную систему?

Мы ещё не избавились полностью от цифровых физических моделей. Если мы
полностью беспристрастны, представляется оправданным вопрос о том, имеют ли
бесконечно делимые величины (другими словами, действительно непрерывные
величины) какую-либо реальность в природе. Каковы были бы последствия,
например, если бы мы перешли к полному квантованию всех законов природы и
предположили бы в принципе, что каждая физическая величина подчиняется
некоторому виду квантования?

Прежде чем предпринять исследование реального вопроса, давайте сначала
рассмотрим классическую модель термодинамики, через которую соотношение
газов рассматривается моделью резиновых шаров, свободно движущихся через
пространство и сталкивающихся друг с другом. Если статическое поведение
этих шаров заменяется дифференциальным уравнением, оно справедливо только
для пространственных измерений, которые велики по сравнению со средним
расстоянием между отдельными частицами. В действительности, модель может
рассматриваться как аналоговая в большом масштабе, однако в деталях она
характеризуется корпускулярной природой материи.

\begin{wrapfigure}{r}{0.30\textwidth}
    \centering
    \includegraphics[width=0.28\textwidth]{images/page_020_img_01.png}
    \caption{Модель летящих и сталкивающихся частиц}
    \label{fig:12}
\end{wrapfigure}

Как выглядело бы вычисленное решение, если бы мы напрямую имитировали
модель летящих, сталкивающихся частиц? Конечно, отправная точка больше не
является дифференциальным уравнением; траектории полёта отдельных частиц
отслеживаются с помощью цифровых вычислений (Рис. 12, 13 и 14).

Для современных электронных компьютеров довольно просто составить программу
для этой цели. Мы не хотим вовлекаться в эти вычисления в ходе нашего
обсуждения (само вычисление относительно сложное и скучное), поскольку
необходимо большое количество частиц для того, чтобы результаты имели
статистическую ценность. Траектории полёта просты для вычисления, поскольку
они прямолинейны (эффекты гравитации не учитываются).
