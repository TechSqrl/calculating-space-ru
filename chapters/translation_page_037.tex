% Страница 37: Продолжение раздела 3.3
% Конрад Цузе "Rechnender Raum"

\begin{wrapfigure}{r}{0.40\textwidth}
    \centering
    \includegraphics[width=0.38\textwidth]{images/page_037_img_01.png}
    \caption*{Рис. 49}
\end{wrapfigure}

\begin{wrapfigure}{l}{0.40\textwidth}
    \centering
    \includegraphics[width=0.38\textwidth]{images/page_037_img_02.png}
    \caption*{Рис. 50}
\end{wrapfigure}

В случае взаимно ортогональных стрелок более длинная стрелка разделяется на две части; вклад одной части эквивалентен вкладу стрелки, ортогональной ей, и объединяется с первой, как на рис. 42. Остаток действует как изолированная стрелка (рис. 58).

Теперь мы способны конструировать частицы с различными направлениями распространения. Количество различных возможных направлений зависит от числа возможных значений вклада стрелки.

На рис. 59 показан пример с отношением стрелок 5~:~2. Направление движения соответствует отношению стрелок. Частицы проходят через различные фазы. Частица на рис. 59 имеет период $7\Delta t$. В течение одного периода частицы проходят через дискретную точку координаты $Q$ (нулевая точка фазы).

Частицы «исчезают» через определенные промежутки времени. Возможно построить линии одинаковой фазы (линии фаз $\tau_0 - \tau_6$).

На рис. 60 представлен пример ограничения возможных дискретных направлений движения. Необходимо подчеркнуть, что существует взаимная зависимость между скоростью распространения и направлением. Выбранное правило распространения не допускает различия в скорости частиц, движущихся в одном направлении.

На рис. 61--66 показана еще одна серия интересных случаев взаимодействия между такими частицами. Опять же процесс взаимодействия зависит от фазы. Реакция между двумя частицами всегда происходит, когда они находятся соответственно в нулевой точке на пересечении (например, рис. 61 и 62). Однако они могут вступать в реакцию и при других обстоятельствах, как показывают примеры на рис. 65 и 66. В этих случаях роль играют уже упомянутые линии фаз. Мы
