% Страница 22 - Конрад Цузе "Вычисляющее пространство"

Мы теперь понимаем, что современная физика заменила эту классическую картину. Процессы столкновения между отдельными частицами не поддаются точному определению согласно современной физике. Существуют только законы вероятности, которые соответствуют законам классической механики, рассматриваемым как статистическое среднее. Рассеяние обусловлено этим эффектом, с результатом, что даже для теоретически предполагаемых частных случаев порядок системы уменьшается со временем, и энтропия возрастает. Как это воспроизводится в компьютерной модели? До тех пор, пока мы не программируем специально этот эффект рассеяния в нашу модель, тщательно разработанный частный случай, о котором говорилось выше, не проявляет никакого эффекта рассеяния. Однако, как только система, через введение небольшого входного сигнала рассеяния, выходит из особенного упорядочения, ситуация становится похожей на ту, которая получается с моделями современной механики. Обычно нет необходимости уделять особое внимание эффектам рассеяния. Ошибки, присущие вычислениям, за исключением частных случаев, имеют тот же эффект (рис. 14).

Классическая модель требует абсолютной точности расчётов, требуя в компьютерной модели инструмента с бесконечным числом разрядов. Так как это невозможно на практике, ошибки вычислений проникают в процессы столкновений, которые имеют эффект --- подобно модели современной механики --- что отклонения от путей, предсказываемых теориями классической механики, появляются. Было бы возможно выразить эти отклонения статистическим законом. Однако существует значительное различие. В модели современной механики ошибки реальны; в вычислительной модели всё строго предопределено, не в смысле классической механики, а в смысле определённых входных параметров расчёта, которые могут только приближаться к классической модели. И то, и другое приводит к возрастанию энтропии.

Первоначально эквивалентный результат (то есть возрастание энтропии) возникает в обоих случаях из-за небольших отклонений от классической механики. В современных физических моделях эти отклонения определяются законами вероятности; в случае компьютерных моделей --- определёнными ошибками вычислений.

Это может казаться незначительным с первого взгляда. Однако если мы расширим этот процесс мышления несколько дальше, могут быть выведены весьма интересные следствия в отношении причинности, которые будут развиты в Главе 4.

Матричная механика также может рассматриваться в теории автоматов. В любом случае нам нужен автомат, в котором переход от одного состояния к следующему определяется законами вероятности. Матрицы переходов матричной механики соответствуют таблицам состояний автомата. Эта возможность автоматно-теоретических наблюдений не будет рассматриваться в большей степени. В следующей главе будут представлены несколько примеров дискретной обработки полей и задач о частицах.
