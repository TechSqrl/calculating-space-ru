% Страница 13 - продолжение раздела 2.3

\begin{wrapfigure}{r}{0.4\textwidth}
  \includegraphics[width=\linewidth]{images/page_013_img_01.png}
  \caption{Форма поверхности жидкости во вращающемся сосуде}
  \label{fig:10}
\end{wrapfigure}


В действительности мы имеем здесь выражение, справедливое для ситуации
только после установления равновесия. Для каждой равновесной ситуации
существует инициирующее действие. В эксперименте с вращающимся сосудом,
первоначально находящимся в покое, вращательное движение должно быть
передано жидкости через силы трения. Только после сложного волнового
взаимодействия, которое уменьшается со временем, установится равновесие.
По этой причине невозможно описать фактические процессы в этом переходе
с помощью нашего дифференциального уравнения. Процессы, происходящие в
течение этого периода, значительно более сложны, и их почти невозможно
описать математически. Мы также осознаём, что нет необходимости следить
за каждым из этих сложных процессов, когда нас интересует только конечное
состояние.

Соотношения очень похожи для многих уравнений в частных производных.
Эти уравнения используются для описания распределения напряжений в
равновесной ситуации в плоских и объёмных напряжённых состояниях.
Установление равновесия происходит в действительности через
высококомплексную последовательность шагов, в которой опять же торможение
этих процессов является условием для конечного установления равновесия.

Дифференциальные уравнения описывают только конечное состояние в случае
теории идеально несжимаемых жидкостей. Фактический процесс, ведущий к
установлению конечного состояния равновесия из состояния покоя, едва ли
мыслим без учёта сжимаемости и процессов торможения.

В случае этих дифференциальных уравнений вопрос не в фундаментальном
законе, который может быть описан в терминах теории автоматов как
функциональная переменная различных, последовательно происходящих
состояний. Это также влияет на возможные численные решения. Дифференциальные
уравнения, которые описывают допустимую последовательность состояний системы,
часто легче решить численно, чем те, которые представляют не более чем
управляющую функцию над конечным состоянием. Фактически, решения для таких
конечных состояний обычно должны быть найдены в пошаговом решении, часто
с помощью релаксационного процесса. Нет необходимости придавать значение
пошаговым приближениям конечного состояния для моделирования природных
или технических процессов; таким образом, возможно применять математически
более простые процессы в приближении.

Дифференциальное уравнение, которое описывает эволюционный процесс от
