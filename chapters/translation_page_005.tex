% Страница 5 - Глава 1: ВВЕДЕНИЕ

\chapter{Введение}
\label{ch:introduction}

Сегодня для нас очевидно, что численные расчёты могут успешно применяться
для прояснения физических взаимосвязей. Тем самым мы получаем
более или менее тесную взаимосвязь между математиками, физиками
и специалистами по обработке информации, соответствующую Рис.~1.

Математические системы служат для построения физических моделей,
численный расчёт которых сегодня выполняется с помощью электронного
оборудования для обработки данных.

\begin{wrapfigure}{r}{0.3\textwidth}
  \includegraphics[width=\linewidth]{images/page_005_img_01.png}
  \caption{Взаимосвязь между математиками, физиками и специалистами по обработке данных (Рис.~1)}
  \label{fig:1}
\end{wrapfigure}


Функция специалистов по обработке данных заключается прежде всего в том,
чтобы находить наиболее подходящие численные решения для моделей,
которые разработали математики и физики.
Обратное влияние обработки данных на сами модели и физические теории
выражается косвенно в предпочтительном использовании тех методов,
для которых численные решения особенно легко получить.

Тесное взаимодействие между математиками и физиками оказало особенно
благоприятное влияние на развитие моделей в теоретической физике.
Современная система квантовой теории в значительной степени является
чистой и прикладной математикой.

Поэтому представляется оправданным вопрос о том, может ли обработка данных
играть не просто вспомогательную роль в этом взаимодействии, или же она
также может быть источником плодотворных идей, которые сами влияют на
физические теории. Этот вопрос тем более оправдан, поскольку в тесном
сотрудничестве с обработкой данных развилась новая отрасль науки —
теория автоматов.
