% Перевод страницы 64 из "Rechnender Raum" Конрада Цузе
% Послесловие: страница 64

\begin{wrapfigure}{r}{0.4\textwidth}
  \includegraphics[width=\linewidth]{images/page_064_img_01.png}
  \caption{Реплика первого механического компьютера Конрада Цузе Z1 (Рис.~75). Двоичный механический калькулятор с электрическим приводом, использовавший булеву логику и числа с плавающей запятой. Фото H.~Zenil, Deutsches Technikmuseum, Берлин.}
  \label{fig:75}
\end{wrapfigure}


и/или вселенная). Hector Zenil (HZ) встретил проф. Horst Zuse (профессора Technische Universität Berlin) осенью 2006 года во время конференционного ужина в Берлине. Тема конференции была именно <<Является ли Вселенная компьютером?>> (Ist das Universum ein Computer?) и проводилась в Deutsches Technikmuseum и организована в честь Года информатики (Informatik Jahr) в Германии\textsuperscript{7}.

Конрад Цузе, однако, признал проблемы, которые, вероятно, возникнут при попытке согласовать цифровой взгляд на вселенную с теориями физики, предполагающими работу в пространствах континуума. Но согласно Конраду Цузе, законы физики могли быть объяснены в терминах законов переключателей или реле (неудивительно, учитывая его опыт преобразования его машин из механической в электронную форму через использование реле), и рассматривал физические законы как вычислительные приближения, захваченные математическими моделями. Ясно из Rechnender Raum, что Цузе знал, что дифференциальные уравнения могут быть решены цифровыми системами и считал этот факт свидетельством в пользу цифровой теории.

Годы до того, как John von Neumann объяснил преимущества архитектуры компьютера, в которой процессор отделён от памяти, Цузе уже пришёл к тому же выводу. Будучи строителем компьютеров в 1930-х годах, Цузе работал как любитель полностью вне математического сообщества, в своё собственное время, по вечерам и в выходные, в гостиной дома своих родителей. Однако он получил некоторую финансовую помощь от местного производителя счётных машин. Он также убедил Helmut Schreyer, бывшего университетского однокурсника, работать с ним. На совет своего друга Schreyer Цузе перешёл от механического к электромеханическому, телефонному релейному аппарату.

В своей автобиографии\textsuperscript{8} Цузе пишет, что в 1939 году, когда началась война, он был призван в пехоту для службы на передовой. Он никогда не видел боевых действий в качестве солдата. Его военная служба должна была продолжаться шесть месяцев, <<шесть месяцев, во время которых у меня было много времени, чтобы размышлять об идеях, разработанных и отражённых в моих дневниковых заметках 1937 и 1938 годов>>. Он был освобождён от активной службы и уволен, чтобы он мог выполнять работу, непосредственно связанную с разработкой оружия, как конструктор в Специальном отделе F компании Henschel Aircraft, где были разработаны дистанционно управляемые летающие бомбы.

В 1941 году, вскоре после завершения Z3, Цузе вернулся к работе конструктором в авиастроении в Henschel, дневной работе, в то время как он начал компанию~--- Zuse Apparatebau (Zuse Apparatus Construction), для производства его машин.

\footnotesize
\noindent
\textsuperscript{7}HZ написал пост в блоге об этом, доступный онлайн по адресу \url{http://www.mathrix.org/liquid/archives/is-the-universe-a-computer}.

\noindent
\textsuperscript{8}<<The Computer -- My Life>>, опубликовано на немецком языке издательством Springer-Verlag в 1993 году и переведено на английский в 2010 году, в юбилей рождения Цузе.
