% Полный перевод книги Конрада Цузе "Вычисляющее пространство"
% Rechnender Raum (1969) на русский язык (неофициальный перевод энтузиастов, 2025)

\documentclass[12pt,a4paper]{book}
\usepackage[utf8]{inputenc}
\usepackage[T2A]{fontenc}
\usepackage[russian]{babel}
\DeclareUnicodeCharacter{03BB}{$\lambda$}
\DeclareUnicodeCharacter{2228}{$\lor$}
\DeclareUnicodeCharacter{21D2}{$\Rightarrow$}
\DeclareUnicodeCharacter{03C0}{$\pi$}
\usepackage{amsmath}
\usepackage{amssymb}
\usepackage{graphicx}
\graphicspath{{../}}
\usepackage{wrapfig}
\usepackage{hyperref}
\usepackage{cite}
\usepackage{float}
\usepackage[section]{placeins}
\usepackage{geometry}
\geometry{
    left=3cm,
    right=2cm,
    top=2.5cm,
    bottom=2.5cm
}

% Настройки для корректной работы с русским языком
\selectlanguage{russian}

% Отключаем автоматическую генерацию гиперссылок для содержания
\hypersetup{
    colorlinks=true,
    linkcolor=black,
    filecolor=black,
    urlcolor=black,
    citecolor=black
}

\title{{\Huge\textbf{ВЫЧИСЛЯЮЩЕЕ ПРОСТРАНСТВО}}\\[0.5cm]
{\Large (Rechnender Raum)}}
\author{{\Large Конрад Цузе}\\[0.5cm]
{\small Перевод на русский язык (энтузиасты, 2025)}}
\date{1969}

\begin{document}

\maketitle

\tableofcontents

% Предисловие переводчика
\chapter*{Предисловие к русскому изданию}
\addcontentsline{toc}{chapter}{Предисловие к русскому изданию}

Данный перевод фундаментального труда Конрада Цузе «Rechnender Raum» (1969) представляет русскоязычному читателю одну из первых систематических попыток осмысления физической реальности через призму вычислительных процессов. Цузе, создатель первого программируемого компьютера, выдвинул революционную гипотезу: вселенная может быть понята как гигантский клеточный автомат, где фундаментальные законы физики возникают из простых вычислительных правил.

При переводе особое внимание уделялось точной передаче научных концепций, математических формул и структуры изложения, насколько это возможно при любительском переводе.

Перевод на русский язык выполнен группой энтузиастов в 2025 году и не является официальным изданием.

\clearpage

% Основное содержание - включаем все переведённые страницы
% Страницы 5-20
% Страница 5 - Глава 1: ВВЕДЕНИЕ

\chapter{Введение}
\label{ch:introduction}

Сегодня для нас очевидно, что численные расчёты могут успешно применяться
для прояснения физических взаимосвязей. Тем самым мы получаем
более или менее тесную взаимосвязь между математиками, физиками
и специалистами по обработке информации, соответствующую Рис.~1.

Математические системы служат для построения физических моделей,
численный расчёт которых сегодня выполняется с помощью электронного
оборудования для обработки данных.

\begin{wrapfigure}{r}{0.3\textwidth}
  \includegraphics[width=\linewidth]{images/page_005_img_01.png}
  \caption{Взаимосвязь между математиками, физиками и специалистами по обработке данных (Рис.~1)}
  \label{fig:1}
\end{wrapfigure}


Функция специалистов по обработке данных заключается прежде всего в том,
чтобы находить наиболее подходящие численные решения для моделей,
которые разработали математики и физики.
Обратное влияние обработки данных на сами модели и физические теории
выражается косвенно в предпочтительном использовании тех методов,
для которых численные решения особенно легко получить.

Тесное взаимодействие между математиками и физиками оказало особенно
благоприятное влияние на развитие моделей в теоретической физике.
Современная система квантовой теории в значительной степени является
чистой и прикладной математикой.

Поэтому представляется оправданным вопрос о том, может ли обработка данных
играть не просто вспомогательную роль в этом взаимодействии, или же она
также может быть источником плодотворных идей, которые сами влияют на
физические теории. Этот вопрос тем более оправдан, поскольку в тесном
сотрудничестве с обработкой данных развилась новая отрасль науки —
теория автоматов.

% Страница 6 - продолжение Главы 1

На последующих страницах будет развито несколько идей в этом направлении.
Мы не претендуем на полноту в рассмотрении предмета.

Такой процесс влияния может исходить из двух направлений:

\begin{enumerate}
\item Разработка и предоставление алгоритмических методов, которые могут
служить физику в качестве новых инструментов, с помощью которых он может
переводить свои теоретические знания в практические результаты. К ним
относятся прежде всего численные методы, которые всё ещё являются основным
инструментом при использовании электронных вычислительных машин. Идеи,
изложенные в последующих главах, могли бы особенно способствовать решению
проблемы численной устойчивости.

К ним относятся символьные вычисления, которые приобретают всё большее
значение сегодня. Под этим подразумевается не численный расчёт формулы,
а алгебраическая обработка самих формул в том виде, как они выражены
символами. Именно в квантовой механике обширная разработка формул
необходима прежде, чем может быть выполнен фактический численный расчёт.
Эта весьма интересная область не будет рассматриваться в последующем
материале.

\item Можно постулировать прямой процесс влияния, в частности, мыслительных
моделей теории автоматов на сами физические теории. Этот предмет, без
сомнения, более сложный, но также и более интересный.
\end{enumerate}

В этом заключается понятная трудность, состоящая в том, что различные области
знания должны быть приведены во взаимосвязь друг с другом. Уже сама область
физики разделяется на специализированные направления. Одни только математические
методы современной физики больше не знакомы каждому математику, и их понимание
требует многолетнего специального изучения.

Но даже теории и области знания, связанные с обработкой данных, уже разделяются
на различные специальные отрасли. В качестве примеров можно привести формальную
логику, теорию информации, теорию автоматов и теорию формальных языков. Идея
объединения этих областей (в той мере, в какой они релевантны) под термином
«кибернетика» ещё не получила широкого признания. Концепция кибернетики как
моста между науками весьма плодотворна, совершенно независимо от различных
определений самого термина.

Автор разработал несколько основных идей в этом направлении, которые он считает
ценными для представления на обсуждение. Некоторые из этих идей в их нынешней,
всё ещё незрелой форме могут быть несовместимы с проверенными концепциями
теоретической физики. Цель достигнута, если будет инициировано обсуждение
% Страница 7 - окончание Введения и начало Главы 2

и провоцирует стимулирование, которое однажды приведёт к решениям, приемлемым
также и для физиков.

Метод, применяемый ниже, в настоящее время всё ещё носит эвристический характер.
Автор считает, что условия ещё не созрели для формулировки точной теоретической
системы. Прежде всего, в Главе 2 существующие математические и физические модели
будут рассмотрены с точки зрения теории автоматов. В Главе 3 представлено
несколько примеров цифровых моделей и вводится выражение «цифровая частица».
В Главе 4 будет развито несколько общих мыслей и соображений, основанных на
результатах Глав 2 и 3, а в Главе 5 кратко рассматриваются перспективы
возможности дальнейших разработок.

\chapter{Вводные наблюдения}
\label{ch:introductory-observations}

\section{О теории автоматов}
\label{sec:automaton-theory}

Теория автоматов сегодня уже является широко развитой и в определённой степени
весьма абстрактной теорией, о которой написана обширная литература. Тем не менее,
автор хотел бы провести различие между собственно теорией автоматов и мыслительными
моделями, связанными с этой теорией, которые будут широко использоваться в
последующих главах. Глубокое понимание теории автоматов не является необходимым
для понимания последующих глав.

Теория автоматов появилась примерно одновременно с развитием современного
оборудования для обработки данных. Конструкция и принцип работы этих устройств
потребовали теоретических исследований, основанных на различных математических
методах, например, на методах математической логики. Первым полезным результатом
этого развития стала математика соединений, в которой особенно важную роль может
играть исчисление высказываний математической логики. Особое значение имеет
осознание того, что вся информация может быть разложена на значения да-нет (биты).
«Истинностные значения» исчисления высказываний принимают только две оценки
(истина и ложь). Поэтому связующие операции и правила исчисления высказываний
можно рассматривать как элементарные операции обработки информации. На Рис. 2
показаны элементарные соединения, соответствующие трём основным операциям
исчисления высказываний: конъюнкции, дизъюнкции и отрицанию.

Дальнейшие исследования привели к введению термина «состояние» автомата.
Кроме того, роль играют входные данные и выходные данные. Из входных данных
и начального состояния получаются новое состояние и выходные данные в соответствии
с алгоритмом, встроенным в автомат. На Рис. 3 показана принципиальная схема
% Страница 8 - продолжение раздела 2.1

\begin{wrapfigure}{r}{0.4\textwidth}
  \includegraphics[width=\linewidth]{images/page_008_img_01.png}
  \caption{Элементарные логические операции}
  \label{fig:2}
\end{wrapfigure}


\begin{wrapfigure}{l}{0.3\textwidth}
  \includegraphics[width=\linewidth]{images/page_008_img_02.png}
  \caption{Схема автомата для двухразрядного двоичного регистра}
  \label{fig:3}
\end{wrapfigure}


автомата для двухразрядного двоичного регистра. На рисунке $E_1$ и $E_0$ представляют
входы, на которые может быть подано двухразрядное двоичное число, а $A_2$, $A_1$
и $A_0$ представляют выходы, которые имеют значение трёхразрядного двоичного
числа. Двухразрядное двоичное число, образованное из разрядов $A_1$ и $A_0$,
передаётся обратно в автомат и представляет текущее состояние двоичного числа.
(В данном случае состояния символизируют число, уже введённое в процесс сложения,
к которому должно быть добавлено число $E_1$, $E_0$).

\begin{wrapfigure}{r}{0.3\textwidth}
  \includegraphics[width=\linewidth]{images/page_008_img_03.png}
  \caption{Таблица состояний для автомата на Рис. 3}
  \label{fig:4}
\end{wrapfigure}


Алгоритм, заданный автоматом, в простых случаях может быть представлен таблицами
состояний. Они имеют форму матрицы и для каждого состояния и каждой комбинации
входов дают результирующее состояние или комбинацию выходов. На Рис. 4 показана
таблица состояний для автомата на Рис. 3. В данном конкретном случае таблица
состояний представляет собой таблицу сложения. Теория автоматов исследует
различные возможные преобразования такого автомата и устанавливает ряд общих
правил, касающихся его метода работы. Для дальнейшего важно понимать термины
\textit{конечный}, \textit{автономный} и \textit{клеточный автомат}. Конечный автомат работает
с дискретным числом дискретных состояний; он приблизительно эквивалентен цифровой
машине обработки данных, которая состоит из ограниченного числа элементов, причём
каждый элемент способен принимать ограниченное число состояний (по крайней мере два),
в результате чего весь автомат может принимать только ограниченное число состояний.
Аналогичные условия справедливы для входов и выходов. Автономный автомат не может
принимать никаких входов (выходы также относительно несущественны). Он может

% Страница 9 - окончание раздела 2.1 и начало раздела 2.2

быть представлен, следовательно, машиной, которая работает независимо после
запуска. Его состояния следуют линейно в последовательности, как только запущена
начальная комбинация, и операционный процесс не может быть подвержен внешнему
влиянию из-за отсутствия входов.

Клеточный автомат представляет особую форму автомата, построенного из
взаимосвязанных, периодически повторяющихся ячеек. Этот тип автомата имеет
особое значение для последующих наблюдений. Позже он будет обсуждаться более
подробно.

Под термином «автоматно-теоретический способ мышления» мы понимаем способ
наблюдения, согласно которому техническая, математическая или физическая
модель рассматривается с точки зрения последовательности состояний, которые
следуют друг за другом согласно предопределённым правилам.

\section{О компьютерах}
\label{sec:about-computers}

Теория автоматов может использоваться как абстрактная математическая система,
однако эти мыслительные структуры также могут быть соотнесены с техническими
моделями, и аналогично теория автоматов может использоваться для описания
автоматов, особенно тех, которые подходят для обработки информации. В текущем
расширенном употреблении термин «вычислять» идентичен «обработке информации».
По аналогии термины «компьютер» и «машина обработки информации» могут
рассматриваться как идентичные.

Мы различаем два класса компьютеров: аналоговые компьютеры и цифровые
компьютеры. В аналоговом компьютере шаги вычисления выполняются в «аналоговой»
модели. Величины, представляющие численные значения, теоретически представлены
через непрерывные физические величины, такие как положения механических частей
(угол кручения), напряжение, скорости и тому подобное. Машина работает по
существу бесконечно. Представленные значения очевидно лежат ниже определённых
технических пределов. Они устанавливаются максимальными значениями и точностью
системы. Максимальные значения задаются чётко определённым верхним пределом,
который соответствует техническим ограничениям системы. Напротив, точность не
имеет чётко определённой величины, поскольку она зависит от изменений и от
внешних влияний (температура, влажность, присутствие возмущающих полей и т.д.).
Одним из хорошо известных аналоговых компьютеров является логарифмическая
линейка. На Рис. 5 показан механический суммирующий механизм в форме рычага,
который может быть заменён вращающим механизмом с шестернями, как на Рис. 6.
Этот механизм известен в технике под неподходящим термином «дифференциальный
механизм» и используется в задней оси каждого автомобиля.

Типичный конструктивный элемент аналоговых машин представлен интеграционным
механизмом, показанным на Рис. 7. Он работает с фрикционным диском A
% Страница 10 - продолжение раздела 2.2

\begin{wrapfigure}{r}{0.3\textwidth}
  \includegraphics[width=\linewidth]{images/page_010_img_01.png}
  \caption{Механический суммирующий механизм в форме рычага}
  \label{fig:5}
\end{wrapfigure}


\begin{wrapfigure}{l}{0.4\textwidth}
  \includegraphics[width=\linewidth]{images/page_010_img_02.png}
  \caption{Вращающий механизм с шестернями}
  \label{fig:6}
\end{wrapfigure}


находящимся в контакте с фрикционным колесом B. Расстояние $r$ фрикционного
колеса B от оси A может изменяться. Таким образом, механизм может использоваться
для интегрирования. В современных аналоговых приборах эти механические элементы
заменены электронными. Интегрирование может, например, выполняться путём
зарядки конденсатора.

\begin{wrapfigure}{r}{0.48\textwidth}
  \includegraphics[width=\linewidth]{images/page_010_img_03.png}
  \caption{Интеграционный механизм с фрикционным диском}
  \label{fig:7}
\end{wrapfigure}


Прерывистые процессы, как правило, не воспроизводимы с помощью аналоговых
приборов; другими словами, аналоговые компьютеры плохо спроектированы для
этих процессов.

В цифровых компьютерах все значения представлены числами. Поскольку цифровой
компьютер может содержать только определённую ограниченную сумму чисел, для
представления непрерывных значений доступен только ограниченный запас значений.
Это подразумевает значительное расхождение с математическими моделями.
Математические значения подчиняются концепции бесконечности в двух отношениях.

Во-первых, абсолютная величина чисел неограничена; кроме того, между любыми
двумя заданными значениями может предполагаться существование бесконечного
числа промежуточных значений. По этой причине компьютеры имеют (независимо
от используемого числового кода) максимальные значения, которые из технических
соображений (количество разрядов регистра и памяти) не могут быть превышены.
Кроме того, значения идут ступенчатым образом. Существуют соседние значения,
между которыми не могут быть вставлены дополнительные промежуточные значения. Это

% Страница 11 - продолжение раздела 2.2

приводит, среди прочих последствий, к ограниченной точности. В отличие от
аналоговых компьютеров, точность цифровых компьютеров строго определена и
не подвержена каким-либо случайным влияниям.

Дальнейший вывод состоит в том, что никакой цифровой компьютер не может
точно воспроизвести результаты процессов, определённых арифметическими
аксиомами. Так, например, математическая формула
\[
\frac{a \cdot b}{a} = b
\]
имеет общую применимость с единственным исключением, что $a$ не может быть
равно 0. Не существует конечного автомата, способного воспроизвести этот
факт точно и в общем виде. Тем не менее, возможно, увеличивая количество
разрядов до и после десятичной точки, для цифрового компьютера бесконечно
близко приблизиться к законам арифметики.

Мы в области математики уже настолько привыкли к концепции бесконечности,
что принимаем её, не рассматривая, что каждый бесконечный член связан с
разложением в ряд или с предельным процессом («для каждого числа существует
следующее за ним»). Соотнося этот процесс с теорией автоматов, мы получаем
вместо статического, предопределённого, конечного автомата ряд автоматов,
которые построены согласно определённому плану и отличаются друг от друга
только числом разрядов. Дан план построения автомата с $n$ разрядами;
кроме того, имеются инструкции для преобразования $n$-разрядного автомата
в автомат с $n + 1$ разрядами. С помощью предельного процесса $\lim_{n \to \infty}$
с использованием разложения в ряд получается правило автомата для
арифметических операций.

Цифровой компьютер, благодаря своей особой способности обрабатывать не
только числа, но и общую информацию (в отличие от аналогового компьютера),
открыл совершенно новые области, обсуждаемые ниже более подробно. В общем,
все вычислительные задачи могут быть решены на цифровом компьютере, тогда
как аналоговые компьютеры лучше подходят для специальных задач. Необходимо
подчеркнуть, что цифровые компьютеры работают строго детерминированным
образом. Используя один и тот же алгоритм (т.е. ту же программу) и вводя
одни и те же входные значения, всегда должны быть получены одинаковые
результаты. Ограниченная точность всегда приводит к одной и той же степени
неточности в результатах, когда операция выполняется несколько раз с
одними и теми же входными данными. Это в отличие от аналогового компьютера,
в котором ограниченная точность имеет различный эффект каждый раз при
выполнении программы и может быть выражена только в терминах статистической
вероятности.

В качестве дополнительных замечаний можно отметить, что были разработаны
гибридные системы, которые состоят из смеси принципов цифрового и
аналогового компьютера.
% Страница 12 - окончание раздела 2.2 и начало раздела 2.3

\begin{figure}[htbp]
  \centering
  \includegraphics[width=0.6\textwidth]{images/page_012_img_01.png}
  \caption{Гибридная система с цифро-аналоговым и аналого-цифровым преобразователями}
  \label{fig:8}
\end{figure}

Это может быть просто осуществлено через систему, в которой два компьютера
работают параллельно. Они соединены цифро-аналоговым преобразователем и
аналого-цифровым преобразователем (Рис. 8). В системах этого типа отдельные
части задачи разделены таким образом, что для каждого подраздела задачи
выбирается более подходящее устройство.

\begin{wrapfigure}{l}{0.4\textwidth}
  \includegraphics[width=\linewidth]{images/page_012_img_02.png}
  \caption{Представление величин через плотность импульсов}
  \label{fig:9}
\end{wrapfigure}


Объединение двух систем также может быть осуществлено через представление
самих значений. Так, например, величина может характеризоваться плотностью
импульсов (Рис. 9). Сами импульсы имеют цифровой характер, поскольку они
нормализованы по интенсивности и длительности; следовательно, они цифровые,
но их плотность (количество импульсов в единицу времени) может иметь любое
количество промежуточных значений, и поэтому она имеет аналоговый характер.
Сегодня распространено мнение, что человеческая нервная система работает
по этому принципу.

\begin{figure}[htbp]
  \centering
  \includegraphics[width=0.6\textwidth]{images/page_012_img_02.png}
  \caption{Представление величин через плотность импульсов}
  \label{fig:9}
\end{figure}

\section{Дифференциальные уравнения с точки зрения теории автоматов}
\label{sec:differential-equations}

Наблюдение за несколькими дифференциальными уравнениями показывает, что
этот способ мышления отнюдь не является самоочевидным для математиков и
физиков. В нашем распоряжении имеется ряд моделей физических данных,
которые могут быть представлены дифференциальными уравнениями. Например,
мы можем взять простое дифференциальное уравнение для представления формы
верхней поверхности жидкости во вращающемся сосуде, согласно которому в
каждой точке поверхности нормаль к поверхности определяется векторной
суммой гравитационного и центробежного ускорений (Рис. 10).

Это уравнение записывается:
\[
y' = \frac{r\omega^2}{g}
\]
где $\omega$ — угловая скорость контейнера.

Решение очень легко получить аналитически:
\[
y = \frac{\omega^2}{2g} \cdot r^2
\]

% Страница 13 - продолжение раздела 2.3

\begin{wrapfigure}{r}{0.4\textwidth}
  \includegraphics[width=\linewidth]{images/page_013_img_01.png}
  \caption{Форма поверхности жидкости во вращающемся сосуде}
  \label{fig:10}
\end{wrapfigure}


В действительности мы имеем здесь выражение, справедливое для ситуации
только после установления равновесия. Для каждой равновесной ситуации
существует инициирующее действие. В эксперименте с вращающимся сосудом,
первоначально находящимся в покое, вращательное движение должно быть
передано жидкости через силы трения. Только после сложного волнового
взаимодействия, которое уменьшается со временем, установится равновесие.
По этой причине невозможно описать фактические процессы в этом переходе
с помощью нашего дифференциального уравнения. Процессы, происходящие в
течение этого периода, значительно более сложны, и их почти невозможно
описать математически. Мы также осознаём, что нет необходимости следить
за каждым из этих сложных процессов, когда нас интересует только конечное
состояние.

Соотношения очень похожи для многих уравнений в частных производных.
Эти уравнения используются для описания распределения напряжений в
равновесной ситуации в плоских и объёмных напряжённых состояниях.
Установление равновесия происходит в действительности через
высококомплексную последовательность шагов, в которой опять же торможение
этих процессов является условием для конечного установления равновесия.

Дифференциальные уравнения описывают только конечное состояние в случае
теории идеально несжимаемых жидкостей. Фактический процесс, ведущий к
установлению конечного состояния равновесия из состояния покоя, едва ли
мыслим без учёта сжимаемости и процессов торможения.

В случае этих дифференциальных уравнений вопрос не в фундаментальном
законе, который может быть описан в терминах теории автоматов как
функциональная переменная различных, последовательно происходящих
состояний. Это также влияет на возможные численные решения. Дифференциальные
уравнения, которые описывают допустимую последовательность состояний системы,
часто легче решить численно, чем те, которые представляют не более чем
управляющую функцию над конечным состоянием. Фактически, решения для таких
конечных состояний обычно должны быть найдены в пошаговом решении, часто
с помощью релаксационного процесса. Нет необходимости придавать значение
пошаговым приближениям конечного состояния для моделирования природных
или технических процессов; таким образом, возможно применять математически
более простые процессы в приближении.

Дифференциальное уравнение, которое описывает эволюционный процесс от

% Страница 14 - продолжение 2.3 и начало 2.4

точки зрения теории автоматов может быть названо формой «вывода», поскольку
следующее состояние возникает из данного состояния через действие
дифференциала на данное состояние. В случае жидкостей и газов включение
члена сжатия сначала приводит к этой форме вывода. Состояние системы
задано распределением давления и скорости. Разности в давлении приводят
к силам, ведущим к новому распределению скоростей, которое само ведёт к
новому распределению плотности и, следовательно, давления через движение
масс. «Состояние» поля может быть описано, следовательно, скалярным полем
плотности $\gamma$ и полем скоростей $\mathbf{v}$. Уравнение может быть
выражено в форме вывода следующим образом:
\[
k \cdot \text{grad} \, \gamma \Rightarrow \frac{\partial \mathbf{v}}{\partial t}
\]
\[
-\text{div} \, \mathbf{v} \Rightarrow \frac{\partial \gamma}{\partial t}
\]
($k$ — фактор, который определяется физическими условиями). Алгоритмический
характер ещё более ясно выражен в следующей форме:
\[
\mathbf{v} + k(\text{grad} \, \gamma)dt \Rightarrow \mathbf{v}
\]
\[
\gamma - (\text{div} \, \mathbf{v})dt \Rightarrow \gamma
\]

В соответствии с обычными правилами языка программирования (алгоритмического
языка), одинаковые символы по обе стороны знака вывода относятся к различным
последовательным состояниям системы ($\mathbf{v}$, $\gamma$).

В случае несжимаемых жидкостей существует условие $\text{div} \, \gamma = 0$.
Это уравнение не имеет алгоритмического характера и в результате не может
быть преобразовано в форму вывода. Оно представляет лишь одно условие для
правильности решения, полученного другими средствами.

\section{Уравнения Максвелла}
\label{sec:maxwell-equations}

Уравнения Максвелла также могут быть изучены с этой точки зрения. Мы
ограничимся теми уравнениями, которые описывают распространение поля
в вакууме:
\[
\text{rot} \, \mathbf{H} = \frac{1}{c} \frac{\partial \mathbf{E}}{\partial t}
\]
\[
\text{div} \, \mathbf{E} = 0
\]
\[
\text{rot} \, \mathbf{E} = -\frac{1}{c} \frac{\partial \mathbf{H}}{\partial t}
\]
\[
\text{div} \, \mathbf{H} = 0
\]
% Страница 15 - продолжение раздела 2.4

Оба уравнения, которые содержат дифференциальный оператор rot, могут быть
легко преобразованы в форму вывода:
\[
\mathbf{E} + c(\text{rot} \, \mathbf{H})dt \Rightarrow \mathbf{E}
\]
\[
\mathbf{H} - c(\text{rot} \, \mathbf{E})dt \Rightarrow \mathbf{H}
\]
(ротор $\mathbf{H}$ даёт приращение $\mathbf{E}$; ротор $\mathbf{E}$ даёт приращение $\mathbf{H}$).

Оба уравнения дивергенции, с другой стороны, не имеют формы вывода. Если
принять во внимание волновую область поля, мы получаем:
\[
\text{div} \, \mathbf{E} = 4\pi\rho
\]

Это уравнение недостаточно для алгоритмического описания закона
распространения волн. Являются ли поэтому уравнения Максвелла неполными?
Они используются для описания распространения поперечных, но не продольных
волн. Причина того, что уравнения Максвелла в их обычной форме достаточны
для описания всех процессов, происходящих в электромагнитных полях,
основывается на том факте, что в природе не существует растущих, вновь
появляющихся или исчезающих волн. Происходят только смещения заряда. При
такого рода смещении уравнения Максвелла достаточны для описания изменений
в полях, связанных со смещениями. Автор не смог найти точного математического
доказательства этого ни в одном тексте, но это должно предполагаться.
Интересное замечание в этом отношении находится в «Beckersauter» (страница 186),
где развивается поле для равномерно движущегося заряда. Это приводит,
что достаточно интересно, к эллиптической деформации ранее сферически-симметричного
поля. Эта деформация соответствует гипотезе сокращения Лоренца.
Возможно переформулировать утверждение, что «уравнения Максвелла инвариантны
относительно специальной теории относительности»: «В результате использования
природой трюка бокового расширения (ротор) в расширяющемся поле, система
специальной теории относительности логически обоснована».
% Страница 16 - продолжение раздела 2.4

\begin{wrapfigure}{r}{0.3\textwidth}
  \includegraphics[width=\linewidth]{images/page_016_img_01.png}
  \caption{Распределение поля между двумя противоположными зарядами}
  \label{fig:11}
\end{wrapfigure}


Мы можем представить функциональную природу этого бокового расширения
следующим образом: предположим, что мы хотим вычислить поле между двумя
противоположными зарядами $+e$ и $-e$, допустим, что мы не знаем распределение
поля само по себе, хорошо известное и также легко выводимое. Мы начинаем,
как показано на Рис. 11, с распределения, заведомо являющегося ложным,
просто соединяя $+e$ и $-e$ линейно-постоянной силой от начала до конца.

Применение уравнений Максвелла к этому распределению поля приводит к
многошаговому асимптотическому приближению к определяемому полю.

В этом процессе также демонстрируется, что мы получаем результаты без
использования уравнения
\[
-\text{div} \, \mathbf{E} \Rightarrow \frac{\partial \gamma}{\partial t}
\]
при рассмотрении электромагнитных полей, хотя, как мы видели, это уравнение
необходимо для рассмотрения сжимаемых жидкостей. Нам даже не нужно вводить
плотность электрического поля $\gamma$. Тот факт, что результаты получаются
без этого члена, не является доказательством того, что природа работает без
обращения к плотности поля. Предполагая, однако, что такое условие
существовало, тем не менее было бы почти невозможно продемонстрировать его
существование, поскольку оба уравнения «ротора» устанавливают сами по себе
такое распределение поля, что
\[
\text{div} \, \mathbf{E} = 0
\]
удовлетворяется в общем случае. В результате дивергенция не вносит вклада
в распределение поля. Поскольку невозможно создавать или уничтожать заряды,
у нас нет экспериментальных средств для проверки справедливости закона
продольного расширения в природе.

Каково же тогда обоснование для исследования этого закона? Вопрос
интересен в связи с концепцией численной устойчивости, и он будет
рассмотрен снова ниже.

% Страница 17 - разделы 2.5 и 2.6

\section{Идея о гравитации}
\label{sec:gravitation-idea}

В этом отношении вводится краткое рассмотрение гравитации. Если мы принимаем
справедливость уравнений Максвелла в их переданном смысле также и для
гравитации, то простое объяснение распространения гравитационных полей
движущимися массами и инвариантность законов небесной механики, основанная
на этом распределении, применяется к специальной теории относительности.
Поскольку относительные скорости небесных тел в пределах нашего диапазона
наблюдения лежат в порядке величины 1/10\,000 от скорости света,
гравитационные магнитные поля были просто настолько слабыми, что они были
неизмеримыми. Разумеется, должно рассматриваться небольшое затухание
планетарных движений. Автор был бы весьма благодарен за критическое
рассмотрение этих мыслей физиками.

\section{Дифференциальные уравнения и разностные уравнения, дискретизация}
\label{sec:digitalization}

Если дифференциальные уравнения выражены в форме «вывода» согласно теории
автоматов, то они могут быть смоделированы технической моделью (автоматом)
и решены. Сам по себе аналоговый компьютер является идеальным автоматом.
Он работает, по крайней мере в теории, с непрерывными значениями и работает
постоянно; другими словами, мы имеем непрерывный поток состояний, последнее
из которых всегда определяется тем, что ему предшествует. На практике
аналоговые компьютеры используются главным образом для вычисления
дифференциальных уравнений. Тем не менее, существует довольно узкий предел
возможностей аналогового компьютера. Для уравнений в частных производных
аналогичные технические модели доступны только при особых обстоятельствах.

Решение дифференциальных уравнений с помощью цифрового автомата немедленно
осложняется ранее упомянутыми трудностями: дифференциальные уравнения
оперируют непрерывными значениями и бесконечными плотностями поля. Цифровые
инструменты оперируют прерывистыми значениями. Бесконечная плотность поля
потребовала бы бесконечной ёмкости памяти и бесконечного времени вычисления.
Поэтому необходимо достигать компромиссов в обоих отношениях.

Обычно переходят от дифференциальных уравнений к разностным уравнениям,
когда ищутся численные решения. В этом процессе полученные значения всё ещё
рассматриваются как непрерывные. Фактически, переход от дифференциальных
уравнений к разностным уравнениям включает два граничных перехода:
(1) $\Delta x \to dx$, и (2) увеличение числа разрядов включённых величин.

Первый граничный переход постоянно ведёт к предельному значению, которое
предвосхищает второй переход; другими словами, построение разностных

% Страница 18 - продолжение раздела 2.6 и начало 2.7

частных имеет смысл только если градации между значениями много меньше,
чем выбранное значение $\Delta$. Этот факт имеет определённое влияние на
численную устойчивость вычисления.

Если переходы выполняются таким образом, что значения остаются
приблизительно того же порядка величины, что и шаговые значения, сохраняется
ступенчатая форма кривой, и невозможно построить дифференциальное частное.

В последующих наблюдениях это расстояние будет использоваться намеренно,
конкретно через последовательное дальнейшее развитие мыслей о дискретизации.

Систематическое сужение числа разрядов рассматриваемых величин приводит к
ограничению переменных теми, которые охватываются элементарной логикой;
например, значения да-нет или троично-переменные значения. Как мы обнаружим
позже, тройные значения и троичная система счисления, основанная на этих
значениях, имеет определённые преимущества, поскольку округление вверх и
округление вниз легче выполнять, и деление на 6, необходимое при делении
области поля на 6 соседних ячеек, также легче вычислить. Присваивая
значения +1, 0 и -1 числам, это соответствует возможным электрическим
частицам +e, 0, -e.

Непрерывная плотность поля должна быть разделена на отдельные значения для
численного решения — процесс, который проще всего выполнить с помощью сетки.
Простейшей сеткой, несомненно, является ортогональная. Существуют другие
возможные выборы: треугольные и шестиугольные сетки в двух измерениях,
например, и сетка в трёх измерениях, соответствующая наиболее плотной
упаковке сфер.

Если в вычислении возникает несколько различных значений поля (например,
векторы скорости и плотности), не обязательно, чтобы эти значения были
локализованы в одной и той же точке сетки. Нет необходимости локализовать
три компоненты пространственного вектора. В этом случае также возможно
разделение. Нет дальнейшей необходимости в построении цифровой структуры
пространства приближать законы евклидова пространства. Ряд общих наблюдений
о представлении физических проблем был представлен ранее с точки зрения
теории автоматов.

\section{Наблюдения физических теорий с точки зрения теории автоматов}
\label{sec:automaton-observations}

До этого момента мы рассматривали только проблему использования компьютеров
для приближения физических моделей и численного отслеживания физических
процессов. В этом контексте было бы возможно предложить фундаментально
отличный вопрос: в какой степени реализации, полученные из изучения
вычислимых решений, полезны при применении непосредственно к физическим
моделям? Является ли природа

% Страница 19 - продолжение раздела 2.7

цифровой, аналоговой или гибридной? И есть ли по существу какое-либо
оправдание для постановки такого вопроса?

Классические модели физики, несомненно, аналоговы по природе. Напряжённость
поля различных потенциалов, как и сила гравитации, не подчинены
«партикуляризации». Не существует таких пределов, как «пороговые значения»
(минимальный размер), предельные значения (максимальные значения) или пределы
плотности самого поля. Даже расширение классических законов теорией
относительности полностью находится в рамках концепции континуума. Только
для скорости предполагается существование абсолютного верхнего предела
(скорости света), и эта концепция полностью согласуется с «аналоговым»
мышлением.

Только с введением особой природы материи через её подразделение на молекулы,
атомы и элементарные частицы несколько величин приобрели дискретный характер,
но это не обязательно должно приравниваться к «цифровой» интерпретации
законов природы. Классическая проблема многих тел была аналоговой по природе,
даже когда каждое из отдельных тел обладало индивидуальными характеристиками
с дискретными свойствами (массами).

Квантовая физика первой отклоняется в нескольких отношениях от концепции
бесконечных величин, в той мере, в какой она предполагает только дискретные
значения для определённых физических величин. Наиболее известно соотношение
между частотой и энергией светового кванта, которое определяется формулой
$E = h \cdot \nu$, где $h$ — универсальная константа природы. Разумеется,
сама энергия не квантована, а только частное $E/\nu$. Это несколько
отличается от случая, когда энергия может иметь только дискретное число
значений из-за ограниченного числа разрядов в вычислителе цифрового
компьютера.

Постулаты квантовой теории имеют далеко идущие последствия в отношении
квантования различных физических величин. Концепции классического
пространственного континуума действительно отвергаются, но не через замену
континуума сеткой дискретных значений, а скорее через процесс, посредством
которого переходят к фундаментально отличным исходным точкам, подобным
конфигурационному пространству более высоких измерений, в котором определены
вероятностные значения (например, вероятность того, что частица находится
в определённом месте в определённое время). Даже в этой концепции идея
континуума не отвергается, поскольку дифференциальные уравнения квантовой
механики не управляются никакими ограничениями в отношении величин полей.

Модели современной физики, следовательно, имеют дело как с непрерывными,
так и с дискретными значениями. Казалось бы уместным рассмотреть гибридную
систему. Будет чрезвычайно трудно найти техническую модель гибридного
компьютера, которая ведёт себя согласно законам квантовой физики.

Мы признали предварительный вывод, что наши физические модели
% Страница 20 - продолжение раздела 2.7

лучше всего могут быть представлены как гибридные системы. Можно ли из этого
сделать выводы относительно природы? Следует ли поэтому саму природу
рассматривать как гибридную систему?

Мы ещё не избавились полностью от цифровых физических моделей. Если мы
полностью беспристрастны, представляется оправданным вопрос о том, имеют ли
бесконечно делимые величины (другими словами, действительно непрерывные
величины) какую-либо реальность в природе. Каковы были бы последствия,
например, если бы мы перешли к полному квантованию всех законов природы и
предположили бы в принципе, что каждая физическая величина подчиняется
некоторому виду квантования?

Прежде чем предпринять исследование реального вопроса, давайте сначала
рассмотрим классическую модель термодинамики, через которую соотношение
газов рассматривается моделью резиновых шаров, свободно движущихся через
пространство и сталкивающихся друг с другом. Если статическое поведение
этих шаров заменяется дифференциальным уравнением, оно справедливо только
для пространственных измерений, которые велики по сравнению со средним
расстоянием между отдельными частицами. В действительности, модель может
рассматриваться как аналоговая в большом масштабе, однако в деталях она
характеризуется корпускулярной природой материи.

\begin{wrapfigure}{r}{0.30\textwidth}
    \centering
    \includegraphics[width=0.28\textwidth]{images/page_020_img_01.png}
    \caption{Модель летящих и сталкивающихся частиц}
    \label{fig:12}
\end{wrapfigure}

Как выглядело бы вычисленное решение, если бы мы напрямую имитировали
модель летящих, сталкивающихся частиц? Конечно, отправная точка больше не
является дифференциальным уравнением; траектории полёта отдельных частиц
отслеживаются с помощью цифровых вычислений (Рис. 12, 13 и 14).

Для современных электронных компьютеров довольно просто составить программу
для этой цели. Мы не хотим вовлекаться в эти вычисления в ходе нашего
обсуждения (само вычисление относительно сложное и скучное), поскольку
необходимо большое количество частиц для того, чтобы результаты имели
статистическую ценность. Траектории полёта просты для вычисления, поскольку
они прямолинейны (эффекты гравитации не учитываются).


% Страницы 21-40
% Страница 21 - Конрад Цузе "Вычисляющее пространство"

\begin{wrapfigure}{r}{0.30\textwidth}
    \centering
    \includegraphics[width=0.28\textwidth]{images/page_021_img_01.png}
    \caption{Модель столкновения частиц}
    \label{fig:13}
\end{wrapfigure}

Процессы столкновения являются наиболее интересной частью. Предполагается равенство массы и упругости частиц. Сначала рассмотрим случай, когда частицы встречаются точно; то есть, во-первых, траектории лежат в одной плоскости и взаимно пересекаются, и во-вторых, центры обеих частиц одновременно встречаются в точке пересечения. Этот случай неинтересен, так как упругое столкновение не отличается значительно от ситуации, в которой обе частицы продолжают двигаться беспрепятственно по своим путям, если каждую частицу рассматривать отдельно. Кроме того, в общих ситуациях вероятность возникновения такого случая стремится к нулю по мере повышения точности расчётов. Поэтому интерес представляют только те случаи, в которых траектории не пересекаются точно, или в которых центры подходят к приблизительному пересечению только примерно одновременно.

\begin{wrapfigure}{l}{0.30\textwidth}
    \centering
    \includegraphics[width=0.28\textwidth]{images/page_021_img_02.png}
    \caption{Траектории частиц после столкновения}
    \label{fig:14}
\end{wrapfigure}

В этом случае частицы имеют различные траектории после столкновения, чем до него. Здесь нет необходимости останавливаться и прочно устанавливать закон столкновения. Поведение зависит от размера частиц и закона упругости. Крупные частицы сталкиваются чаще, чем маленькие. Жёсткие частицы ведут себя иначе, чем мягкие. Статистический результат поведения большого числа частиц остаётся одним и тем же.

Если сравнить такую вычислительную модель с физической моделью, возникают следующие интересные аспекты. В обоих случаях можно увидеть, что в целом упорядоченные состояния приводят к неупорядоченным состояниям, то есть энтропия возрастает. В любом случае можно создать определённые исключительные случаи, в которых данная энтропия остаётся постоянной. Возьмём, например, сосуд с точно параллельными стенками и серию частиц, траектории которых точно перпендикулярны этим стенкам и достаточно далеко отстоят друг от друга, так что взаимодействия частиц между собой нет. В этом случае траектории остаются неизменными в смысле классической механики. То же самое справедливо и для компьютерной модели, если система координат, на которой основаны расчёты, установлена параллельно или ортогонально стенкам. Конечно, существуют и другие интересные частные случаи, в которых происходят процессы столкновения между частицами, но тем не менее остаётся определённый порядок (рис. 14).

% Страница 22 - Конрад Цузе "Вычисляющее пространство"

Мы теперь понимаем, что современная физика заменила эту классическую картину. Процессы столкновения между отдельными частицами не поддаются точному определению согласно современной физике. Существуют только законы вероятности, которые соответствуют законам классической механики, рассматриваемым как статистическое среднее. Рассеяние обусловлено этим эффектом, с результатом, что даже для теоретически предполагаемых частных случаев порядок системы уменьшается со временем, и энтропия возрастает. Как это воспроизводится в компьютерной модели? До тех пор, пока мы не программируем специально этот эффект рассеяния в нашу модель, тщательно разработанный частный случай, о котором говорилось выше, не проявляет никакого эффекта рассеяния. Однако, как только система, через введение небольшого входного сигнала рассеяния, выходит из особенного упорядочения, ситуация становится похожей на ту, которая получается с моделями современной механики. Обычно нет необходимости уделять особое внимание эффектам рассеяния. Ошибки, присущие вычислениям, за исключением частных случаев, имеют тот же эффект (рис. 14).

Классическая модель требует абсолютной точности расчётов, требуя в компьютерной модели инструмента с бесконечным числом разрядов. Так как это невозможно на практике, ошибки вычислений проникают в процессы столкновений, которые имеют эффект --- подобно модели современной механики --- что отклонения от путей, предсказываемых теориями классической механики, появляются. Было бы возможно выразить эти отклонения статистическим законом. Однако существует значительное различие. В модели современной механики ошибки реальны; в вычислительной модели всё строго предопределено, не в смысле классической механики, а в смысле определённых входных параметров расчёта, которые могут только приближаться к классической модели. И то, и другое приводит к возрастанию энтропии.

Первоначально эквивалентный результат (то есть возрастание энтропии) возникает в обоих случаях из-за небольших отклонений от классической механики. В современных физических моделях эти отклонения определяются законами вероятности; в случае компьютерных моделей --- определёнными ошибками вычислений.

Это может казаться незначительным с первого взгляда. Однако если мы расширим этот процесс мышления несколько дальше, могут быть выведены весьма интересные следствия в отношении причинности, которые будут развиты в Главе 4.

Матричная механика также может рассматриваться в теории автоматов. В любом случае нам нужен автомат, в котором переход от одного состояния к следующему определяется законами вероятности. Матрицы переходов матричной механики соответствуют таблицам состояний автомата. Эта возможность автоматно-теоретических наблюдений не будет рассматриваться в большей степени. В следующей главе будут представлены несколько примеров дискретной обработки полей и задач о частицах.

% Страница 23 - Конрад Цузе "Вычисляющее пространство"

\chapter{Примеры дискретной обработки полей и частиц}
\label{ch:digital-fields}

\section{Выражение <<Цифровая частица>>}
\label{sec:digital-particle}

Рассмотрим сначала одномерное пространство. В этом отношении мы можем привести пример из гидромеханики и один из инженерии приборов. Рассмотрим поведение газов без трения в прямом цилиндре. После исключения и сбора слагаемых, которые для наших целей неуместны (плотность и т.д.), мы можем получить несколько упрощённое соотношение реальных физических сил.

У нас есть две величины: $p$ (давление), которое мы фиксируем в дискретных точках 1, 2 и 3, и $v$ (скорость), которую мы выражаем в промежуточных точках 1', 2' и 3'.

\[
\begin{array}{cccccc}
p & 1 & 2 & 3 & 4 & 5 \\
v & 1' & 2' & 3' & 4' & 5'
\end{array}
\]

где $\Delta_s p$ и $\Delta_s v$ представляют разность значений $p$ и $v$ между соседними точками, $\Delta_t p$ и $\Delta_t v$ соответствуют разностям между $p$ и $v$ в последовательных временных интервалах.

Следующие дифференциальные уравнения затем имеют место:

\[
k_0 \Delta_s p \Rightarrow \Delta_t v
\]

\[
k_1 \Delta_s v \Rightarrow \Delta_t p
\]

Выраженное словами: изменение скорости пропорционально изменению давления, и разность давления пропорциональна изменению скорости. Во втором уравнении член $\Delta_t p$ переписан так, чтобы указать, что он относится к $\Delta p$ после того, что в первом уравнении. Два коэффициента $k_0$ и $k_1$, содержащие физические характеристики $\Delta x$ (пространственный компонент) и $\Delta t$ (временной компонент), могут быть для наших целей объединены в один коэффициент $k$. Мы затем получаем:

\[
- \Delta_s p \Rightarrow \Delta_t v
\]

\[
- k \Delta_s v \Rightarrow \Delta_t p
\]

Символ $\Rightarrow$ используется, чтобы указать, что $\Delta p$ во втором уравнении не идентичен тому, что в первом уравнении.

Ясно, что эти уравнения могут быть преобразованы из дифференциальных уравнений в разностные уравнения, когда $\Delta x$ и $\Delta t$ стремятся к нулю.

% Страница 24 - Конрад Цузе "Вычисляющее пространство"

Совершенно противоположное условие представляет интерес для нас. Хотя математики и программисты обычно пытаются установить разностные уравнения таким образом, чтобы дифференциальное уравнение, лежащее в основе разностного уравнения, приближалось как можно ближе, мы можем разрешить вопрос, используя наиболее общую дискретизацию.

Мы теперь способны преобразовать физический закон импульса в инженерный закон приборов. Если мы позволим величинам $p$ и $v$ и соответствующим значениям $\Delta p$ и $\Delta v$ принимать только целые значения, мы должны выбрать целое числовое значение $k$, чтобы разностное уравнение давало целые результаты. Если мы сначала примем $k = 1$, мы получаем уравнения:

\[
- \Delta_s p \Rightarrow \Delta_t v
\]

\[
- \Delta_s v \Rightarrow \Delta_t p
\]

Мы далее пытаемся отнести $p$ и $v$ наименьшим возможным значениям, то есть $-1$, 0 и $+1$, и изучить поведение системы, которая удовлетворяет этим условиям. Мы получаем в результате следующее арифметическое соотношение:

\[
v - \Delta_s p \Rightarrow v
\]

\[
p - \Delta_s v \Rightarrow p
\]

На рис. 15 показана простая схема вычислений для этого правила. Мы имеем четыре значения $v$, $-\Delta v$, $p$ и $-\Delta p$ за единицу времени. Пространственные сектора противоположны друг другу. Нулевые значения не записываются для простоты. Четыре стабильные элементарные формы представлены [(1), (2), (3) и (4)], которые мы будем считать взаимно независимыми "цифровыми частицами". Есть два временных интервала, $t_1$ и $t_2$, соответственно для значений $v$, $-\Delta v$, $p$ и $-\Delta p$; $v$ и $p$ предполагаются для $t_1$. Отсюда следует, что $-\Delta v$ и $-\Delta p$ соответствуют временному интервалу $t_2$, и, следуя приведённому выше уравнению, значения $v$ и $p$ соответствуют следующему временному интервалу $t_2$.

Уравнения относятся к распространению простого импульса. Частицы стабильны только при этой скорости. В то же время эта скорость является наивысшей возможной для системы. Система не допускает никаких других скоростей. На рис. 16 показана графическая версия этого импульса.

С точки зрения теории автоматов, мы имеем дело с линейно-расширенным бесконечным автоматом, который периодически повторяется в автомате (клеточный автомат). Значения $v$ и $p$ представляют состояния автомата; $\Delta v$ и $\Delta p$ выводятся из них. Приведённое выше уравнение устанавливает функцию, согласно которой последующее состояние вытекает из предыдущего.
% Страница 25 - Конрад Цузе "Вычисляющее пространство"

\begin{figure}[htbp]
\centering
\includegraphics[width=0.6\textwidth]{images/page_025_img_01.png}
\caption{Расширение импульса давления}
\label{fig:15}
\end{figure}

\begin{figure}[htbp]
\centering
\includegraphics[width=0.6\textwidth]{images/page_025_img_02.png}
\caption{Расширение импульса с учётом скорости}
\label{fig:16}
\end{figure}

\begin{figure}[htbp]
\centering
\includegraphics[width=0.6\textwidth]{images/page_025_img_03.png}
\caption{Неустойчивая форма расширения импульса}
\label{fig:17}
\end{figure}

На рисунках 17 и 18 показана неустойчивая форма расширения изолированного импульса давления, с которым не связан никакой импульс скорости (как это было в рисунках 15 и 16). На рисунке 17 значения $\Delta$ опущены по причинам обобщения.

\begin{figure}[htbp]
\centering
\includegraphics[width=0.6\textwidth]{images/page_025_img_04.png}
\caption{Форма расширения импульса с сохранением импульса}
\label{fig:18}
\end{figure}

Эта форма расширения импульса противоречит нашему пониманию расширения первоначально изолированной ячейки давления в газонаполненном цилиндре. Из этой модели мы вывели разностное уравнение. Дискретизация была проведена столь общим образом, что отклонения от дифференциального уравнения приводят к отклонениям от физических законов. Сохранение импульса, а не энергии, является ключом к вычислениям, лежащим в основе разностного уравнения. Графическое представление рис. 18 показывает, что среднее $(p = 1)$ остаётся постоянным, а среднее значение $v$ постоянно равно 0. С другой стороны, расширение чередующихся положительных и отрицательных $p$-значений в графическом представлении указывает на очевидное постоянное увеличение потенциальной энергии. Соответствующее верно для значений кинетической энергии, представленных значениями $v$.

Было бы интересно в этой точке выяснить, является ли такое отклонение обязательно связано с грубой дискретизацией или можно ли построить грубые цифровые модели, которые подчиняются всем условиям исходного дифференциального уравнения, в данном случае особенно сохранению энергии. Конечно, такая упрощённая модель требует точного определения термина <<энергия>>. Это просто отмечается без дальнейшего рассмотрения здесь.

Интересно то, что пара изолированных импульсов даёт устойчивую систему: испускание двух расходящихся цифровых частиц. Очевидно, только определённые конфигурации возможны, тогда как другие исключены или не обеспечивают стабильные результаты. Это имеет определённое сходство с некоторыми ситуациями в квантовой механике.

Так как наше выбранное вычислительное правило имеет чисто аддитивный характер, применяется принцип суперпозиции; то есть отдельные формы могут рассматриваться независимо друг от друга, в результате чего естественно, что появляются значения больше 1.

% Страница 26 - Конрад Цузе "Вычисляющее пространство"

Это означает, что две противоположно движущиеся частицы не влияют друг на друга, но проходят мимо или проходят друг через друга без изменения формы. В системе, строго описываемой принципом суперпозиции, невозможны результаты, которые соответствуют реакциям между элементарными частицами, известными в физике. Это свидетельствует о том, что нет необходимости встраивать линейные элементы в наши модели. Наиболее простой и грубый вид --- это общее ограничение значений сверху и снизу. Это может быть продемонстрировано примерами на рис. 19.

\begin{figure}[htbp]
\centering
\includegraphics[width=0.6\textwidth]{images/page_026_img_01.png}
\caption{Примеры цифровых частиц с различными результатами суперпозиции при ограничении значений}
\label{fig:19}
\end{figure}

Здесь у нас есть две приближающиеся цифровые частицы, а именно в примерах (1) и (2) слева, соответствующие предыдущей реакции согласно принципу суперпозиции. Мы видим, что в примере (1) возникают значения $+2$ и $-2$. В примере (3) частицы проходят друг через друга без возникновения значений больше $+1$ и $-1$.

В этой ситуации может быть наблюдён интересный результат грубой дискретизации. Ход процесса столкновения отличается от фазового состояния расстояния между двумя частицами. Это не видно внешне. На рис. 19 показан пример (1) с законом ограничения, соответствующим рис. 20.

\begin{figure}[htbp]
\centering
\includegraphics[width=0.6\textwidth]{images/page_026_img_02.png}
\caption{Система вычисления с нелинейным правилом, где $1 + 1$ дает значение 1}
\label{fig:20}
\end{figure}

Здесь есть только три значения: $-$, 0 и $+$. На рис. 20 показана соответствующая система расчётов. Она построена таким образом, что $1 + 1$ даёт значение 1. Мы видим, что несмотря на это ограничение, частицы свободно могут пересекаться друг с другом, результат, который сам по себе был бы неожиданным с первого взгляда, так как были сделаны грубые сокращения вычислительного правила. Применение вычислительного правила (рис. 20) к примеру (2) не даёт ничего нового, конечно, потому что в примере не встречаются значения $-2$ и $+2$.

Интересно то, что несмотря на это, может быть отмечен определённый процесс реакции при взаимодействии частиц. Если рассмотреть примеры (2) и (3), например, можно увидеть, что в случае (3), в отличие от (2), может быть наблюдена определённая замедленность процесса. В (2) частицы пересекаются и удаляются друг от друга беспрепятственно. В (3) можно утверждать, что частицы сначала реагируют друг с другом и что в результате этой реакции испускаются две новые цифровые частицы. Вопрос о том, происходит ли (2) или (3), снова зависит от фазового состояния расстояния и внешне является вопросом случайности.

% Страница 27 - Конрад Цузе "Вычисляющее пространство"

Без знания точной пространственной структуры можно только определить, что в нашем примере возможны два фундаментальных случая взаимодействия частиц, для каждого из которых вероятность возникновения равна 1/2.

\begin{figure}[htbp]
\centering
\includegraphics[width=0.6\textwidth]{images/page_027_img_01.png}
\caption{Сводка восьми возможных случаев взаимодействия цифровых частиц}
\label{fig:21}
\end{figure}

\begin{figure}[htbp]
\centering
\includegraphics[width=0.6\textwidth]{images/page_027_img_02.png}
\caption{Схематические идеализированные траектории частиц для двух различных паттернов взаимодействия a и b}
\label{fig:22}
\end{figure}

На рис. 21 показана сводка восьми возможных случаев взаимодействия частиц; рис. 22 представляет схематические, идеализированные траектории частиц для двух различных паттернов взаимодействия a и b. Необходимо явно подчеркнуть, что траектории являются идеализированными траекториями частиц. В действительности наша модель представляет не непрерывное движение, а процесс пошагового продвижения.

Интересно отметить, что в нелинейном вычислительном правиле (рис. 20) изолированная точка давления приводит к испусканию двух частиц. Установление предельных значений очевидно ограничивает свободные процессы суперпозиции. В случае неограниченных значений частицы, соответствующие рис. 15, также теоретически суперпозиционируемы. Это означает, что мы можем построить гору давления любой высоты с её сопутствующим распределением скорости, которое удовлетворяет правилу пошагового расширения; то есть, которое остаётся стабильным. Эти стабильные "крупные" частицы всегда делимы на элементарные частицы. Это уже не верно, когда применяется правило, соответствующее рис. 20 (система вычисления с нелинейным правилом).

Наша исходная позиция, в которой мы выбрали коэффициент 1 относительно значения $\Delta$, соответствует очень жёсткой среде в назначенном физическом паттерне газонаполненного цилиндра. Более гибкая ситуация получается, когда коэффициент уменьшается. В этом случае в более точных расчётах возникают нецелые числа. Если мы хотим продолжить работу с целыми числами или вводить только незначительные градации, необходимо вводить округление вверх и округление вниз.

% Страница 28 - Конрад Цузе "Вычисляющее пространство"

В этом отношении тернарная система превосходит двоичную. Значение 1/2 лежит точно посередине между 0 и 1. Значения 1/3 и 2/3 также могут быть точно вставлены между значениями 0 и 1.

Отсюда мы хотим начать со следующего:

\[
v - \frac{\Delta p}{3} \Rightarrow v
\]

\[
p - \frac{\Delta v}{3} \Rightarrow p
\]

Значения $\Delta p/3$ и $\Delta v/3$ округляются вверх или вниз до целых чисел.

\begin{figure}[htbp]
\centering
\includegraphics[width=0.6\textwidth]{images/page_028_img_01.png}
\caption{Расширение импульса давления в тернарной системе}
\label{fig:23}
\end{figure}

\begin{figure}[htbp]
\centering
\includegraphics[width=0.6\textwidth]{images/page_028_img_02.png}
\caption{Система вычислений с коэффициентом 1/3}
\label{fig:24}
\end{figure}

На рис. 25 (1) показана стабильная частица в этой системе с периодом $3\Delta t$. Скорость распространения составляет 1/3 от скорости частицы в соответствующем рисунке (рис. 15). Это также соответствует физической модели, в которой мягкая среда имеет более медленную скорость звука. Здесь мы имеем ситуацию, при которой "скорость переключения" между соседними частицами значительно выше (в примере в три раза больше), чем скорость частицы.

\begin{figure}[htbp]
\centering
\includegraphics[width=0.6\textwidth]{images/page_028_img_03.png}
\caption{Стабильная частица в тернарной системе: (1) частица с периодом $3\Delta t$; (2) детальная система расчётов}
\label{fig:25}
\end{figure}

В более сложных моделях "вычисляющего пространства" было бы понятно, что скорости света, соответствующие максимальным скоростям частиц, которые значительно медленнее скорости переключения, существуют. Однако это не означает, что в такой модели возможны "скорости сигналов", превышающие скорость света (в модели). Скорость переключения имеет чисто локальное значение.

Интересно то, что цифровая частица принимает различные конфигурации в течение периода. Импульс давления появляется частично только с значением $+2$, частично как пара со значениями $+1$ и $+1$. Положение частицы может быть определено для следующего периода, но не без дополнительной информации для отдельных фаз данного периода. Разве это не аналогично квантовой теории, которая связывает положение и импульс через принцип неопределённости? В любом случае компьютерная модель, несмотря на кажущуюся ошибку, характеризуется строго предопределёнными событиями.

% Страница 29 - Конрад Цузе "Вычисляющее пространство"

\begin{wrapfigure}{r}{0.30\textwidth}
    \centering
    \includegraphics[width=0.28\textwidth]{images/page_029_img_01.png}
    \caption{Выдержка из системы расчётов с представлением процесса взаимодействия двух частиц}
    \label{fig:26}
\end{wrapfigure}

На рис. 26 показан процесс взаимодействия двух таких частиц. Рисунок демонстрирует, что частицы не просто проходят мимо друг друга, а что они реагируют, на этот раз с сокращением времени взаимодействия (в отличие от рис. 22). Процесс также может быть представлен как отталкивание. Здесь можно видеть в режиме просмотра рисунков, что такие термины, как "прохождение сквозь" и "отталкивание", теряют значение, когда применяются к реакции цифровых частиц. Квантовая теория дала соответствующие результаты, хотя и не в цифровой форме.

При взаимодействии частиц, соответствующем рис. 26, конечно, существует значительно больше различаемых случаев, очевидных из систематического исследования, в сравнении с примером на рис. 23. Мы должны сначала исследовать, какие частицы возможны в этой системе. Влияние фаз разделения также должно быть учтено, и наконец возможности взаимодействия частиц в различных фазах должны быть рассмотрены.

Целью этого документа не является проведение исчерпывающего обследования. Предыдущее наблюдение нескольких простых примеров стимулирует целый ряд интересных концепций.
% Страница 30 - Конрад Цузе "Вычисляющее пространство"

\begin{wrapfigure}{r}{0.30\textwidth}
    \centering
    \includegraphics[width=0.28\textwidth]{images/page_030_img_01.png}
    \caption{Процесс взаимодействия двух цифровых частиц (Рис.~27)}
    \label{fig:27}
\end{wrapfigure}

\begin{wrapfigure}{l}{0.30\textwidth}
    \centering
    \includegraphics[width=0.28\textwidth]{images/page_030_img_02.png}
    \caption{Идеализированные траектории частиц для взаимодействия по схеме (Рис.~28)}
    \label{fig:28}
\end{wrapfigure}

\begin{wrapfigure}{r}{0.30\textwidth}
    \centering
    \includegraphics[width=0.28\textwidth]{images/page_030_img_03.png}
    \caption{Блок-диаграмма для вычисляющего пространства}
    \label{fig:29}
\end{wrapfigure}

На рис. 29 показана блок-диаграмма для вычисляющего пространства, соответствующего ранее введённому правилу расчётов. Квадраты $v$ и $p$ представляют регистры, к которым могут быть добавлены числа. Сдвигающие части системы, которые служат для выполнения вычитания, представлены кругами, отмеченными $\Delta$. Вертикальная линия на выходе членов $\Delta$ означает отрицание. Блок-диаграмма может, конечно, быть подразделена на её отдельные сдвигающие элементы. Используемые в настоящее время символы сводят сдвиг к его отдельным элементам, которые соответствуют основным операциям булевой алгебры (конъюнкция, дизъюнкция и отрицание). Используемые здесь элементы информации с трёхзначной логикой должны были быть преобразованы в двоичные элементы посредством двух булевых переменных (2 бита). Из 4 возможных комбинаций этих двух значений используются только три. По этой причине более подробное представление опущено. Чтобы сделать блок-диаграмму на рис. 29 работоспособной, необходима чистая импульсная работа. Поэтому импульсные такты представлены на рис. 29 буквами I и II. При этом предполагается, что члены чистого сложения работают без временной задержки для построения значений $\Delta$, в то время как регистры передают свою информацию дальше только с добавлением следующего импульса. Эта импульсная работа соответствует тонкой структуре временного измерения.

\section{Двумерные системы}
\label{sec:2d-systems}

Рассмотрим кратко двумерную систему. Наиболее простая структура --- это сетка, соответствующая ортогональной системе координат. Система обладает двумя определёнными осями, которые входят даже в простое распространение импульса. Начнём с простого правила, где каждая точка сетки может иметь состояния 0 и 1. В каждом временном интервале такая 1 передаётся каждой соседней точке сетки. Комбинация импульсов, возникающих из различных соседних точек, выполняется в соответствии с правилом дизъюнкции. Если состояние точки сетки $(x, y)$ равно $\varphi_{x,y}$, мы получаем следующее уравнение:

% Страница 31 - Конрад Цузе "Вычисляющее пространство"

\[
\varphi_{x-1,y} \vee \varphi_{x+1,y} \vee \varphi_{x,y-1} \vee \varphi_{x,y+1} \Rightarrow \varphi_{x,y}
\]

\begin{wrapfigure}{r}{0.50\textwidth}
    \centering
    \includegraphics[width=0.48\textwidth]{images/page_031_img_01.png}
    \caption{Расширение импульса в двумерной сетке}
    \label{fig:30}
\end{wrapfigure}

Расширение вдоль осей координат происходит быстрее, чем вдоль диагоналей. С таким правилом можно развить мало, так как через короткое время оно приводит к состоянию, при котором все пространственные точки достигают состояния "1" и тем самым никакие конфигурации, частицы и т.д. невозможны (рис. 30).

Далее мы рассмотрим аналогичное правило, в котором, тем не менее, допускаются многоместные значения, а комбинация происходит путём сложения. При передаче между точками сетки значения умножаются на коэффициент $k$. Мы получаем формулу для этого правила:

\[
K(\varphi_{x-1,y} + \varphi_{x+1,y} + \varphi_{x,y-1} + \varphi_{x,y+1}) \Rightarrow \varphi_{x,y}
\]
На рисунках 32 и 33 даны два примера для коэффициентов 1/4 и 1/2. По причинам симметрии необходимо рассмотреть только 45° сечение. Как на рис. 32, значения введены только для фронта импульса. Римские цифры соответствуют отдельным временным фазам с временным интервалом разделения $\Delta t$. Из примеров видно, что фронт движется так, как представлено на рис. 31; то есть, с его пиком вдоль оси координат, хотя значения вдоль диагоналей больше. Устремляющаяся вперёд точка очень скоро достигает своего пика.

% Страница 32 - Конрад Цузе "Вычисляющее пространство"

Поскольку мы не можем предположить бесконечное число малых значений в цифровом пространстве, минимальное значение вскоре достигается; то есть пик затухает. Было бы интересно проследить прогресс такого расширения с помощью вычислительной машины. Вопрос особого интереса состоит в том, возможно ли и как быстро значения сходятся в круговой схеме расширения.

Одно ясно: невозможно построить цифровые частицы из такого правила. Мы должны найти другие правила.

\begin{wrapfigure}{r}{0.40\textwidth}
    \centering
    \includegraphics[width=0.38\textwidth]{images/page_032_img_01.png}
    \caption{Расположение значений $v$ и $p$ в шахматном порядке}
    \label{fig:31}
\end{wrapfigure}

\begin{wrapfigure}{l}{0.30\textwidth}
    \centering
    \includegraphics[width=0.28\textwidth]{images/page_032_img_02.png}
    \caption{Отдельные появляющиеся значения}
    \label{fig:32}
\end{wrapfigure}

\begin{wrapfigure}{r}{0.40\textwidth}
    \centering
    \includegraphics[width=0.38\textwidth]{images/page_032_img_03.png}
    \caption{Волновой фронт параллельно одной из осей координат}
    \label{fig:33}
\end{wrapfigure}

Можно взять правила для линейного пространства, которые порождают стабильные частицы, и применить их к двумерному пространству. Конечно, тогда нам нужна взаимосвязь между двумя измерениями, так как без неё отдельные ортогональные точки сетки имели бы независимое существование. На рис. 31 показана одна возможность расположения значений $v$ и $p$ в шахматном порядке. На рис. 32 показаны отдельные появляющиеся значения. Два компонента, $v_x$ и $v_y$, должны быть рассмотрены для $v$. Одного значения достаточно для $p$.

% Страница 33 - Конрад Цузе "Вычисляющее пространство"

Две оси связаны через $p$. Теперь мы можем сформулировать следующее правило:

\[
v_x - \Delta p_x \Rightarrow v_x
\]

\[
v_y - \Delta p_y \Rightarrow v_y
\]

\[
p - (\Delta v_x + \Delta v_y) \Rightarrow p
\]

\begin{figure}[htbp]
    \centering
    \includegraphics[width=0.6\textwidth]{images/page_033_img_01.png}
    \caption{Размещение значений $v$ и $p$ в шахматной решётке (Рис.~34)}
    \label{fig:34}
\end{figure}

\begin{figure}[htbp]
    \centering
    \includegraphics[width=0.6\textwidth]{images/page_033_img_02.png}
    \caption{Получающиеся значения компонентов скорости $v_x$, $v_y$ и давления $p$ (Рис.~35)}
    \label{fig:35}
\end{figure}

\begin{figure}[htbp]
    \centering
    \includegraphics[width=0.6\textwidth]{images/page_033_img_03.png}
    \caption{Волновой фронт параллельно одной из осей координат}
    \label{fig:36}
\end{figure}

\begin{figure}[htbp]
    \centering
    \includegraphics[width=0.6\textwidth]{images/page_033_img_04.png}
    \caption{Волна, движущаяся по диагонали}
    \label{fig:37}
\end{figure}

Из-за связи через $p$ отдельные импульсы, соответствующие рисункам 15 и 16, исчезают. Стабильные, хотя и не бесконечно параллельные волновые фронты могут быть построены. На рис. 36 показан такой волновой фронт параллельно одной из осей координат, а рис. 37 показывает волну, движущуюся по диагонали. Скорости распространения являются функциями направления.

Было бы интересно рассмотреть различные следствия более или менее грубой дискретизации в этом случае. Так как правила связаны с уравнениями разреженной газовой динамики и гидродинамики, интересно, можно ли (например) гидродинамически стабильную структуру вихря грубо дискретизировать и можно ли построить "цифровые элементы". Это исследование может быть проведено только с помощью вычислительных машин.

Чтобы построить стабильные частицы в двумерном пространстве, мы сначала рассмотрим другой способ.

\section{Цифровые частицы в двумерном пространстве}
\label{sec:2d-particles}

Предположим ортогональную сетчатую структуру, соответствующую рис. 38. Мы больше не делаем различие между $v$-точками и $p$-точками, но позволяем каждой точке значения $p_x$, $p_y$. Для простоты сначала предположим, что значения $p$ могут принимать значения $-$, 0, $+$. Мы можем тогда говорить о $p$-стрелках или о коротких стрелках. Сначала мы устанавливаем, что изолированная стрелка (стрелка, у которой нет перпендикулярной стрелки, возникающей в той же точке сетки) непосредственно передаётся в следующую точку сетки. Четыре возможных примера такого одиночного изолированного импульса приводятся ниже. Он может быть передан вперёд только в ортогональном направлении.

Мы можем сначала определить, что есть два случая взаимодействия между двумя стрелками, приближающимися в одной и той же ортогонали. Оба они описаны ниже. В одном случае стрелки продолжают удаляться друг от друга; в другом они сокращаются. Какой случай происходит, зависит от фазы разделения. Нам ещё нужно правило для случая пересекающихся стрелок. Две пересекающиеся стрелки существуют в точке Z в момент времени I. Согласно нашим предыдущим правилам, они были бы распространены вперёд, каждая в своём собственном направлении, независимо от другой.

% Страница 34 - Конрад Цузе "Вычисляющее пространство"

\begin{wrapfigure}{r}{0.40\textwidth}
    \centering
    \includegraphics[width=0.38\textwidth]{images/page_034_img_01.png}
    \caption{Ортогональная сетчатая структура}
    \label{fig:38}
\end{wrapfigure}

Теперь мы устанавливаем, что две стрелки в действительности распространяются вперёд в своих соответствующих направлениях в сторону точек B и C, и в точках B и C они обмениваются направлением. Мы получаем таким образом стабильную частицу с периодом $2\Delta t$, которая распространяется диагонально вперёд.

Интересно отметить, что из этого правила возникают карманы, которые закреплены на 4 соседних точках сетки; они имеют период $2\Delta t$. Также возможен двойной стабильный карман с периодом $\Delta t$. Как видно из рисунков и логического развития, которое можно заполнить из примеров на предыдущих страницах, из такого правила может быть создана целая серия частиц. Понимание всех деталей этого правила и его многочисленных следствий требует углубленного анализа, который здесь не может быть полностью проведён.

Однако можно видеть, что цифровые частицы, которые могут быть построены из таких правил, имеют много свойств, которые соответствуют некоторым свойствам физических частиц. Чтобы полностью понять этот важный факт, необходимо ещё много исследований.

% Страница 35 - Конрад Цузе "Вычисляющее пространство"

\begin{wrapfigure}{r}{0.40\textwidth}
    \centering
    \includegraphics[width=0.38\textwidth]{images/page_035_img_01.png}
\caption{Ортогональная сетчатая структура (Рис.~39)}
    \label{fig:39}
\end{wrapfigure}

\begin{wrapfigure}{l}{0.40\textwidth}
    \centering
    \includegraphics[width=0.38\textwidth]{images/page_035_img_02.png}
\caption{Четыре возможных примера одиночного изолированного импульса (Рис.~40)}
    \label{fig:40}
\end{wrapfigure}

\begin{wrapfigure}{r}{0.40\textwidth}
    \centering
    \includegraphics[width=0.38\textwidth]{images/page_035_img_03.png}
\caption{Два случая взаимодействия между двумя стрелками (Рис.~41)}
    \label{fig:41}
\end{wrapfigure}

\begin{wrapfigure}{l}{0.40\textwidth}
    \centering
    \includegraphics[width=0.38\textwidth]{images/page_035_img_04.png}
\caption{Правило для случая пересекающихся стрелок (Рис.~42)}
    \label{fig:42}
\end{wrapfigure}

\begin{wrapfigure}{r}{0.40\textwidth}
    \centering
    \includegraphics[width=0.38\textwidth]{images/page_035_img_05.png}
\caption{Стабильная частица с периодом $2\Delta t$ (Рис.~43)}
    \label{fig:43}
\end{wrapfigure}

\begin{wrapfigure}{l}{0.40\textwidth}
    \centering
    \includegraphics[width=0.38\textwidth]{images/page_035_img_06.png}
\caption{Карманы, закреплённые на четырёх соседних точках решётки (Рис.~44)}
    \label{fig:44}
\end{wrapfigure}

\begin{wrapfigure}{r}{0.30\textwidth}
    \centering
    \includegraphics[width=0.28\textwidth]{images/page_035_img_07.png}
\caption{Двойной стабильный карман с периодом $\Delta t$ (Рис.~45)}
    \label{fig:45}
\end{wrapfigure}

% Страница 36: Продолжение раздела 3.3
% Конрад Цузе "Rechnender Raum"

дополнительные примеры, эти карманы не могут быть уничтожены.

Теперь у нас имеются частицы, которые могут распространяться в восьми дискретных направлениях в плоскости, а также стоячие карманы. На рис. 46--57 представлена серия интересных примеров взаимодействия таких частиц.

\begin{figure}[htbp]
    \centering
    \includegraphics[width=0.6\textwidth]{images/page_036_img_01.png}
    \caption*{Рис. 46}
\end{figure}

Сначала мы сохраняем условие, что стрелки могут иметь только значения $-$, $0$, $+$. Две противоположно направленные стрелки взаимно уничтожают друг друга в одной точке решетки, а две стрелки одной ориентации действуют как единая изолированная стрелка.

\begin{figure}[htbp]
    \centering
    \includegraphics[width=0.6\textwidth]{images/page_036_img_02.png}
    \caption*{Рис. 47}
\end{figure}

Видно, что ход различных взаимодействий зависит как от времени, так и от разности фаз. Частицы могут проходить друг через друга, взаимно уничтожаться или образовывать новые частицы. Карманы коварны, так как они могут уничтожать частицы, не исчезая сами. С другой стороны, карманы могут возникать из определенных форм взаимодействия (рис. 55 и 57). В модели космоса, функционирующей в соответствии с этим правилом, все частицы в конечном итоге были бы преобразованы в твердые карманы. Поэтому эта модель имеет малую практическую ценность.

\begin{figure}[htbp]
    \centering
    \includegraphics[width=0.6\textwidth]{images/page_036_img_03.png}
    \caption*{Рис. 48}
\end{figure}

При взаимодействии весьма существенно, лежит ли точка пересечения траекторий частиц в дискретно определенной точке системы координат. В этом случае происходит реакция (например, рис. 52 и 53).

Возможности этой системы могут быть исследованы путем введения стрелок различной абсолютной длины. Для стрелок, указывающих в одном направлении, мы используем правило сложения. Гораздо сложнее расширить правило рис. 42, включив в него две пересекающиеся стрелки различных длин. Мы можем принять следующее соглашение.

% Страница 37: Продолжение раздела 3.3
% Конрад Цузе "Rechnender Raum"

\begin{wrapfigure}{r}{0.40\textwidth}
    \centering
    \includegraphics[width=0.38\textwidth]{images/page_037_img_01.png}
    \caption*{Рис. 49}
\end{wrapfigure}

\begin{wrapfigure}{l}{0.40\textwidth}
    \centering
    \includegraphics[width=0.38\textwidth]{images/page_037_img_02.png}
    \caption*{Рис. 50}
\end{wrapfigure}

В случае взаимно ортогональных стрелок более длинная стрелка разделяется на две части; вклад одной части эквивалентен вкладу стрелки, ортогональной ей, и объединяется с первой, как на рис. 42. Остаток действует как изолированная стрелка (рис. 58).

Теперь мы способны конструировать частицы с различными направлениями распространения. Количество различных возможных направлений зависит от числа возможных значений вклада стрелки.

На рис. 59 показан пример с отношением стрелок 5~:~2. Направление движения соответствует отношению стрелок. Частицы проходят через различные фазы. Частица на рис. 59 имеет период $7\Delta t$. В течение одного периода частицы проходят через дискретную точку координаты $Q$ (нулевая точка фазы).

Частицы «исчезают» через определенные промежутки времени. Возможно построить линии одинаковой фазы (линии фаз $\tau_0 - \tau_6$).

На рис. 60 представлен пример ограничения возможных дискретных направлений движения. Необходимо подчеркнуть, что существует взаимная зависимость между скоростью распространения и направлением. Выбранное правило распространения не допускает различия в скорости частиц, движущихся в одном направлении.

На рис. 61--66 показана еще одна серия интересных случаев взаимодействия между такими частицами. Опять же процесс взаимодействия зависит от фазы. Реакция между двумя частицами всегда происходит, когда они находятся соответственно в нулевой точке на пересечении (например, рис. 61 и 62). Однако они могут вступать в реакцию и при других обстоятельствах, как показывают примеры на рис. 65 и 66. В этих случаях роль играют уже упомянутые линии фаз. Мы

% Страница 38: Завершение раздела 3.3 и начало 3.4
% Конрад Цузе "Rechnender Raum"

\begin{wrapfigure}{r}{0.40\textwidth}
    \centering
    \includegraphics[width=0.38\textwidth]{images/page_038_img_01.png}
    \caption*{Рис. 51}
\end{wrapfigure}

\begin{wrapfigure}{l}{0.40\textwidth}
    \centering
    \includegraphics[width=0.38\textwidth]{images/page_038_img_02.png}
    \caption*{Рис. 52}
\end{wrapfigure}

\begin{wrapfigure}{r}{0.40\textwidth}
    \centering
    \includegraphics[width=0.38\textwidth]{images/page_038_img_03.png}
    \caption*{Рис. 53}
\end{wrapfigure}

\begin{wrapfigure}{l}{0.40\textwidth}
    \centering
    \includegraphics[width=0.38\textwidth]{images/page_038_img_04.png}
    \caption*{Рис. 54}
\end{wrapfigure}

могли бы построить временную линию фазы $R$, которая представляет обе частицы. Если она проходит через точку пересечения траекторий частиц $S$, то реакция возможна (рис. 65 и 66).

Разумеется, эти примеры очень просты и примитивны. Однако даже эти простые формы дают обилие подсказок; они показывают, что избранный основной метод дискретизации представляет наибольший интерес и что развитие правил принесет дополнительные концепции.

\section{О трехмерных системах}
\label{sec:three-dimensional-systems}

Концепции, развитые в разделах 3.2 и 3.3, могут быть также применены к трехмерным системам. Исследования автора в этой области еще не завершены и должны быть отложены для дальнейшего изучения.

% Страница 39: Коллекция иллюстраций (Рис. 55-60)
% Конрад Цузе "Rechnender Raum"

\begin{figure}[p]
\centering
\begin{tabular}{cc}
\includegraphics[width=0.43\textwidth]{images/page_039_img_01.png} &
\includegraphics[width=0.43\textwidth]{images/page_039_img_02.png} \\
\textit{Рис. 55} & \textit{Рис. 56} \\[1em]
\includegraphics[width=0.43\textwidth]{images/page_039_img_03.png} &
\includegraphics[width=0.43\textwidth]{images/page_039_img_04.png} \\
\textit{Рис. 57} & \textit{Рис. 58} \\[1em]
\includegraphics[width=0.43\textwidth]{images/page_039_img_05.png} &
\includegraphics[width=0.43\textwidth]{images/page_039_img_06.png} \\
\textit{Рис. 59} & \textit{Рис. 60}
\end{tabular}
\end{figure}

% Страница 40: Коллекция иллюстраций (Рис. 61-66)
% Конрад Цузе "Rechnender Raum"

\begin{figure}[p]
\centering
\begin{tabular}{cc}
\includegraphics[width=0.43\textwidth]{images/page_040_img_01.png} &
\includegraphics[width=0.43\textwidth]{images/page_040_img_02.png} \\
\textit{Рис. 61} & \textit{Рис. 62} \\[1em]
\includegraphics[width=0.43\textwidth]{images/page_040_img_03.png} &
\includegraphics[width=0.43\textwidth]{images/page_040_img_04.png} \\
\textit{Рис. 63} & \textit{Рис. 64} \\[1em]
\includegraphics[width=0.43\textwidth]{images/page_040_img_05.png} &
\includegraphics[width=0.43\textwidth]{images/page_040_img_06.png} \\
\textit{Рис. 65} & \textit{Рис. 66}
\end{tabular}
\end{figure}


% Страницы 41-50
% Страница 41: Раздел 4 - Общие рассмотрения, 4.1 Клеточные автоматы
% Конрад Цузе "Rechnender Raum"

\chapter{Общие рассмотрения}
\label{ch:general-considerations}

\section{Клеточные автоматы}
\label{sec:cellular-automata}

Примеры дискретизации полей и частиц, которые были представлены, в своем настоящем незавершенном виде все еще далеки от возможности служить в формулировке физических правил. Тем не менее, они дают грубое впечатление о возможностях использования инструментов теории автоматов для ответа на физические вопросы.

Примеры в основном рассматривали точечные решетки. Отдельный клеточный автомат состоит, следовательно, из точечной решетки, которая связана с соседними точками посредством обмена информацией. В случаях, показанных на рис. 34 и 35, решетки представляют собой шахматные доски двух различных значений $p$ и $v$ в решеточной форме. Существуют различные возможности их комбинации, так что разделение на отдельные автоматы не специфично. Это не влияет на поведение всей системы.

В целом разделение континуума на дискретные клеточные автоматы имеет различные последствия в зависимости от точного способа разделения. Идея структуры пространственной решетки уже рассматривается в различных контекстах физиками, хотя не применительно к теории автоматов. Вообще говоря, идея того, что космос действительно может быть подразделен на такие ячейки, физиками резко отвергается. Мы согласны с тем, что пространство не может рассматриваться как континуум даже в бесконечно малых сечениях. Концепция наименьшей длины уже широко принята сегодня, хотя не в отношении идеи подразделения на точечную решетку, а скорее как принципиальный предел в различении двух различных частиц. Сомнения, касающиеся структуры решетки, существенно следующие:

\noindent\textbf{(a)} Структура решетки устранила бы изотропию пространства.

Ясно, что регулярная решётчатая структура задаёт привилегированные направления. Это имеет влияние, например, на расширение полей (рис. 31, 38) и на дискретные возможные направления, в которых цифровая частица может двигаться (рис. 60). Мы не знаем физических экспериментов, которые дали бы ключ к предпочтительным направлениям такого типа, но область не была систематически изучена на предмет этого эффекта. Трезвое размышление, тем не менее, показывает, что стоит рассмотреть правила для подобной решеточно-пространственной структуры, которые не позволяют структуре решетки становиться видимой в областях меньшей и промежуточной энергии и частот. Постоянная решетки должна быть значительно меньше, чем элементарная наименьшая длина приблизительно $10^{-13}$ см (Боппа предполагает даже $10^{-56}$ см). Область обычной оптики, например, работает с длинами волн экстраординарной величины в сравнении с этими длинами. Едва ли возможно представить себе эксперимент, который смог бы

% Страница 42: Продолжение раздела 4.1 - Клеточные автоматы
% Конрад Цузе "Rechnender Raum"

определить возможное дискретное направление распространения фотонов, если мы предположим точность такого изменения направления (в круговой мере) того же порядка величины, что и то, что мы способны различить между частотами, а именно $10^{-12}$ (эффект Мёссбауэра).

Результаты такого рода можно ожидать только в очень высоких энергетических диапазонах, когда длина волны и период длины подходят к постоянной решетки. Только сегодня мы имеем способность проводить такие эксперименты. Автор должен оставить физикам решение, могут ли эти явления быть наблюдаемы с помощью имеющихся в настоящее время экспериментальных техник и в каких пределах.

\noindent\textbf{(b)} Искривленные объемы, как они предполагаются общей теорией относительности, трудно представить со структурой решетки пространства.

Боппа выбрал способ, предполагая декартово пространство, в котором три пространственные координаты каждая сходятся на себе. Это можно себе представить в двумерном пространстве, предположив тор.

Конечно, существуют многочисленные возможные отклонения от этих последствий. Весь предмет еще слишком молод для того, чтобы можно было сделать окончательные положительные или отрицательные выводы. Следующие возможности можно упомянуть:

\noindent\textbf{($\alpha$)} Предположение о фиксированных схемах в форме клеточных автоматов не единственная логическая возможность для определения логических связей между дискретными значениями в пространстве. Если мы введем изменение схем как функцию результатов предыдущего процесса, переменные схемы могут быть правильно разработаны.

\noindent\textbf{($\beta$)} Концепция растущего автомата тесно связана с правильной вариабельностью схем.

Обе возможности требуют вначале очень хорошо подготовленной теории. Так как теория автоматов является молодой областью, возможности которой ни в коем случае не исчерпаны, мы можем ожидать дальнейших развитий в рассматриваемом направлении.

\noindent\textbf{($\gamma$)} Предположение о решетке неявно предполагает наличие инерциальной системы, что противоречиво со строгой интерпретацией теории относительности. Это будет рассмотрено более подробно.

В этом свете использование ортогональной сети является наиболее удобным способом начала исследований. Результаты, полученные таким образом, будут, конечно, столь же справедливы, когда со временем теория автоматов даст новые методы для использования.

% Страница 43: Раздел 4.2 - Цифровые частицы и клеточные автоматы
% Конрад Цузе "Rechnender Raum"

\section{Цифровые частицы и клеточные автоматы}
\label{sec:particles-and-automata}

Цифровые частицы можно рассматривать как возмущения в нормальных условиях клеточного автомата. Это возмущение имеет отчетливый паттерн, который подвергается периодическим изменениям. Согласно теории автоматов, каждое состояние развивается из предшествующего; тем не менее, весь паттерн может колебаться в процессе. До некоторой степени мы имеем дело с «текущими состояниями». В соответствии с этим, цифровые частицы можно рассматривать как «самовоспроизводящиеся системы». Данный паттерн генерируется в соседней области клеточного автомата.

В примерах в Главе 3 цифровые поля и цифровые частицы рассматриваются раздельно. Современная теория поля стремится объяснить даже элементарные частицы через сингулярности и специальные формы полей. Теория автоматов, понятно, хорошо подходит для дискретизации таких интерпретаций и их подчинения правилам теории автоматов. Автор надеется иметь возможность рассмотреть этот предмет более глубоко в другом вкладе.

\section{О теории относительности}
\label{sec:relativity-theory}

Вопрос об изотропии пространства, очевидно, требует борьбы с теорией относительности. Преобразования Лоренца, столь важные для специальной теории относительности,, очевидно, могут быть бесконечно приближены посредством численных оценок. Тем не менее, крайне сложно имитировать в цифровом виде согласованную форму модели теории относительности. Наш физический опыт немедленно говорит нам, что не может быть доказана никакая превосходная система координат, и что мы оправданы в рассмотрении каждой системы координат столь же справедливой, как и любая другая, в этом случае преобразования Лоренца формулируют отношения между этими инерциальными системами. Строгая интерпретация специальной теории относительности приводит, однако, к заключению, что в действительности превосходная система координат не существует, и что бесполезно искать такую систему экспериментально. В любом представлении космоса как клеточных автоматов почти невозможно избежать допущения превосходной системы движения. Мы можем конструировать структуру клеточных автоматов таким образом, чтобы было доступно большее число, хотя все еще конечное количество, превосходных систем координат. Постоянство скорости света во всех инерциальных системах представлено посредством цифровой имитации преобразований Лоренца и связанного с этим сокращения тел.

В любом случае, соотношение между скоростью света и скоростью передачи между отдельными ячейками клеточного автомата должно следовать из такой модели. Они не обязательно должны быть идентичны. Напротив, можно предположить, что скорость передачи от ячейки к ячейке должна быть больше

% Страница 44: Продолжение раздела 4.3 - О теории относительности
% Конрад Цузе "Rechnender Raum"

чем скорость распространения сигнала, полученного из этой передачи. Эта более высокая скорость передачи имеет только локальное значение. Из-за анизотропии вычисляющего пространства она различна в различных направлениях. В любом случае «цифровая» модель, по сравнению с аналоговой моделью теории относительности, дает значительное различие: чем ближе скорость инерциальной системы приближается к стандарту скорости света, тем более критичной становится цифровая имитация процессов. В случае богатых энергией частиц мы приходим к процессам, которые можно охарактеризовать (по крайней мере до некоторой степени) как «ошибка вычисления» вычисляющего пространства. Таким образом, существенно различное поведение частиц очень высокой энергии (высокая скорость, высокая частота) может быть объяснено.

Строгая интерпретация специальной теории относительности имеет следствием, что для каждой инерциальной системы может быть воображена другая, которая движется с начальной скоростью меньшей, чем $c$. Физические правила столь же справедливы во второй системе, как и в первой. Этот процесс может быть повторен столько раз, сколько желательно, по крайней мере в принципе. Полная чудовищность этой мысли только смутно понятна. Здесь должно быть сказано опять, что всякое представление бесконечности предполагает ограничивающий процесс. Здесь мы имеем дело с бесконечно частым повторением реакции другой инерциальной системы, которая движется относительно предыдущей. Этот процесс имеет немного последствий, если применяются наблюдения информационно-теоретического характера, как мы будем рассматривать далее.

Следующее утверждение также представляет интерес.

\begin{figure}[htbp]
  \centering
  \includegraphics[width=0.6\textwidth]{images/page_044_img_01.png}
  \caption*{Рис. 67}
\end{figure}

Мы сначала введем термин «объем сдвига». Это равно числу сдвигающихся частей, вовлеченных в процесс, умноженному на число сдвигающихся импульсов, которые участвуют в данном процессе, например в периоде цифровой частицы.

На рис. 67 показано упрощенное представление, в котором можно предположить, что возмущение, представляющее цифровую частицу, простирается на расстояние $P_0 - P_1$. Предполагается, что частица неподвижна в инерциальной системе $x, t$. В этом случае пространство $P_0, P_1, P_2, P_3$ равно объему сдвига одного периода. Если эта частица движется относительно системы $x, t$, мы можем говорить о второй инерциальной системе

% Страница 45: Продолжение раздела 4.3 и начало 4.4
% Конрад Цузе "Rechnender Raum"

\begin{figure}[h]
\caption*{Рис. 67}
\end{figure}

$x', t$ согласно специальной теории относительности, относительно которой движущаяся частица неподвижна. Инверсия, соответствующая преобразованиям Лоренца, дает объем сдвига $P_0, P'_1, P'_2, P'_3$.

Это равно по площади объему сдвига $P_0, P_1, P_2, P_3$. Мы можем, следовательно, говорить об инвариантности объема сдвига.

\section{Рассмотрения теории информации}
\label{sec:information-theory-considerations}

Термин информация получает значительное значение в ходе этих различных рассмотрений. Теория информации сформулировала термин «содержание информации» с ясностью в отношении систем передачи новостей. По этой причине мы склонны рассматривать теорию информации как теорию обработки информации.

Это, однако, не совсем правильно. Легко совершаемое применение терминов из теории информации в соседней области передачи новостей, к сожалению, приводит к частой путанице. Даже в настоящем рассмотрении нам необходимо ясно представлять, что понимается под содержанием информации. Сложно говорить о физических процессах в терминах передачи новостей. Это было бы интересно само по себе только поскольку мы могли бы включить людей в наше рассмотрение. Если мы предположим бесконечно тонкое распространение наших новостей, передаваемых посредством электромагнитных волн, она должна быть бесконечно сохранена, пока границы не будут установлены для них временной конечностью вселенной. Метафорически мы можем также рассмотреть лучи из вселенной, приближающиеся к нам от других звезд, как новости для людей, в этом случае вопрос о содержании информации этих новостей имеет смысл.

Такое отношение между человеком и природой находится в современном утверждении квантовой теории, которая стремится соотнести все измеримые величины в математической системе. Информация, которую мы получаем от природы о структуре атомных оболочек, состоит во многом из частот испущенного света кванта. В этом случае использование термина «содержание информации» является значимым. Вопрос не будет далее исследоваться здесь.

Если мы пренебрегаем этим определением информации как средства передачи новостей, все еще невозможно говорить о содержании информации необитаемых систем, если мы рассматриваем ширину вариации возможных форм объекта, паттерна или тому подобного. Таким образом, карточка с перфорацией может содержать, благодаря своей вариабельности, определенное содержание информации, измеренное в битах.

Технические характеристики самой карточки с перфорацией, включая сопровождающие системы пробивания и считывания, устанавливают верхние пределы количества информации, которая может быть введена, что определяется как емкость информации. В передаче новостей эта емкость не нуждается быть полностью использована, так что информация, передаваемая от отправителя к получателю на карточке с перфорацией

% Страница 46: Продолжение раздела 4.4 - Рассмотрения теории информации
% Конрад Цузе "Rechnender Raum"

может быть ниже емкости.

Также возможно говорить о максимальной возможной информационной емкости конечного автомата, если мы рассматриваем число его возможных состояний как меру. Если это равно $n$, то содержание информации равно $\log_2(n)$ (логарифм по основанию два). Запрограммированная вычислительная машина представляет этот тип автомата, как мы знаем. Если такой инструмент имеет $m$ элементов, для каждого из которых есть две возможные позиции (например, триггеры, ферритовые сердечники в памяти и т.д.), то число возможных состояний равно $2^m$, а информационная емкость равна $m$. В этом процессе не делаются различия между отдельными возможными состояниями. Из всего $2^m$ возможных состояний каждое состояние, в котором каждый регистр и запоминающее устройство удален (т.е. установлен в ноль), считается так же, как состояния, в результате которых решение очень сложного дифференциального уравнения хранится в памяти. Эмоционально мы естественно склонны предполагать, что оборудование не содержит информации в его нулевом состоянии, хотя во втором упомянутом состоянии доступны чрезвычайно интересные научные результаты для использования математиками. Этот пример показывает необходимость большой осторожности в определении терминов в теории информации.

Различие в этой ситуации состоит в том, что для получателя два состояния имеют принципиально различное значение. Состояние «всё удалено» является только расширением знания получателя о том, что машина в данный момент находится в основном состоянии, в то время как во втором случае знание получателя увеличивается в отношении значительных результатов.

Если не учитывать эти отдельные значения информации для получателя, то можно сделать вывод, что содержание информации конечного автомата не может быть увеличено во время выполнения вычисления. Потому что вычисление производится полностью автоматически после введения программы и входных значений, результаты установлены с самого начала. Результаты имеют большую ценность для лица, использующего оборудование: для чего бы он позволил компьютеру выполнить вычисление, если не для увеличения своего знания, что возможно только, если конечное состояние автомата имеет большее содержание информации, чем начальное состояние.

Первый результат рассмотрения космоса как клеточного автомата состоит в том, что отдельные ячейки представляют конечный автомат. Вопрос о том, в какой степени возможно рассматривать всю вселенную как конечный автомат, зависит от допущения, которое мы делаем в отношении ее размеров. Если мы возьмем тор более высокого порядка, как уже предложено Боппой, мы имеем дело с конечным автоматом в целом. Это изначально справедливо, что отдельные ячейки могут принять ограниченное число состояний и, следовательно, имеют только ограниченное содержание информации. Это равно справедливо и для всего космоса, если мы делаем подходящие допущения о его границах.

% Страница 47: Продолжение раздела 4.4 - Рассмотрения теории информации
% Конрад Цузе "Rechnender Raum"

Теория автоматов демонстрирует, что различные характеристические ходы выполнения возможны для конечного автомата, несколько из которых будут рассмотрены.

Для каждого заданного состояния существует последующее состояние. Поэтому возможно выразить отношение «состояние $A$ растворяет состояние $B$» как отношение $F(A, B)$ и представить его в форме диаграммы стрелок. Такая диаграмма стрелок часто называется «графом». На рис. 68a--d показаны различные типы диаграмм стрелок. Важно помнить, что каждое состояние может иметь только одно последующее состояние, хотя существует несколько предшествующих состояний, которые могут его растворить. Диаграммы процессов показывают, что автономный автомат должен в каждом случае завершиться периодическим циклом, который при определенных условиях также может вырождаться в единственное конечное состояние.

\begin{figure}[htbp]
  \centering
  \includegraphics[width=0.6\textwidth]{images/page_047_img_01.png}
  \caption*{Рис. 68}
\end{figure}

Это знание не может быть перенесено на отдельные ячейки клеточного автомата, так как они связаны с соседними ячейками посредством обмена информацией и, следовательно, не приводят к автономному конечному автомату.

При допущениях границ космоса во вселенной мы имеем дело с конечным автономным автоматом, как только мы исключаем любого рода влияния более крупного внешнего мира. Первый результат является несколько разочаровывающим следствием того, что космический процесс с необходимостью должен закончиться периодическим циклом. Это осознание, в себе логически неоспоримое, имеет другие значения при количественном исследовании.

Размеры вселенной предполагаются некоторыми физиками быть на порядке величины $10^{41}$ элементарных длин ($10^{-13}$ см) приблизительно (приблизительно 10 миллионов световых лет). Мы имеем дело, следовательно, с объемом приблизительно $10^{123}$ элементарных кубиков элементарной длины на стороне.

Если отдельный бит содержания информации назначен каждому из этих элементарных кубиков, то у нас уже есть $2^{10^{123}}$ различных состояний вселенной, которые должны рассматриваться. Это число представляет только нижний предел. На самом деле, должна быть предположена гораздо более тонкая решетка, для которой еще неизвестно, сколько вариаций возможны в каждой точке решетки. Далее должно рассматриваться, что пространство вычисляет чрезвычайно точно. Соотношение электростатических взаимодействий к взаимодействиям гравитационных полей составляет приблизительно $10^{40} : 1$. Взаимодействие ядерных сил снова на порядки сильнее. Более высокое из двух значений представляет в действительности только нижний предел, который наиболее вероятно на много порядков слишком мал.

Если мы предположим, что число временных импульсов приближается к порядку величины пространственного расширения, в действительности $10^{41}$, то получается результат, что несмотря на это долгое время, только исчезающе малая часть возможных состояний космоса может существовать. Существуют $2^{10^{82}}$ типов путей реакции возможны, каждый из которых независим от любого другого. Это также означает, что число отклонений и ветвлений не поддается пониманию велико. Ранее рассмотренные наблюдения теории автоматов, относящиеся к рис. 68, теряют всю предсказательную ценность. Какова ценность осознания того, что эволюция вселенной следует периодическому циклу, когда даже в уже очень большом диапазоне времени, рассматриваемом одиночный период в лучшем случае может пройти, и в большинстве случаев вообще нет?

% Страница 48: Продолжение раздела 4.4 - Рассмотрения теории информации
% Конрад Цузе "Rechnender Raum"

Рассмотрение замкнутых процессов, т.е. сдвигающихся процессов, включающих цифровую частицу, представляется более плодотворным. Мы уже наблюдали, что цифровая частица состоит из серии периодически повторяющихся паттернов в клеточном автомате и что они не закреплены на месте, а могут двигаться в пространстве отдельных ячеек, подобно движущейся пишущей машине. Термин «текущее состояние» уже был введен.

Вопрос о содержании информации цифровой частицы можно рассматривать с нескольких точек зрения. Сначала цифровая частица занимает установленную позицию в пространстве в определенный момент времени. Содержание информации цифровой частицы не может быть больше, чем информационная емкость этой позиции в пространстве, которая определяется суммой возможных состояний этой области. Крайне маловероятно, что каждое изменение состояния такой ограниченной области соответствует цифровой частице. Гораздо более вероятно, что ограниченное выделение растворяет отдельные стабильные периодические паттерны.

Мы можем запросить, полностью независимо от пространства, связанного с цифровой частицей, сколько вариаций паттерна, представляющих фазы цифровой частицы, на самом деле возможны? Выгодно классифицировать паттерны вдоль различных линий:

\begin{enumerate}
\item тип;
\item направление и скорость (импульс);
\item состояние фазы;
\item позиция частицы.
\end{enumerate}

Ответ на вопрос 1 предполагает наличие модели, которая допускает различные типы цифровых частиц, как мы имеем в природе с фотонами и электронами и т.д.

Ответ на вопрос 2 требует, чтобы наша модель допускала различные скорости и направления распространения периодического паттерна.

Последовательность фаз результирует из последовательности периодического паттерна, связанной со специальным типом частицы и импульса.

Вопрос 4 имеет смысл только когда рассматривается взаимоотношение частиц. Конечно, невозможно, чтобы замкнутая область пространства держала информацию о собственном состоянии.

Примеры на рис. 42--66 из Главы 3 удовлетворяют этим условиям только в ограниченной мере. Во-первых, модель допускает представление только одного типа частиц. Далее только направление может быть варьировано, но не скорость. Длина периодов отдельных частиц не постоянна, но это не представляет интереса для нашего рассмотрения. Содержание информации этого типа частиц зависит от точности представления длины стрелки или от числа мест, с которым она цифрово представлена. Если мы предположим абсолютные длины компоненты для примера 4, то мы получаем 9 различных длин стрелки, включая нулевое значение, для этого компоненты; в двумерном пространстве есть 81 различные вариации импульса. На основе этих возможных вариаций в частицах, даже в пределах данных ограничений, возможно определить содержание информации частицы. Каждая из этих частиц имеет серию связанных состояний фазы, так что число возможных паттернов цифровых частиц все еще больше. Частица на рис. 59 имеет, например, 7 различных состояний фазы ($\tau_0 - \tau_6$).

Вопрос о сохранении информации при реакции между цифровыми частицами является интересным. В примерах, приведенных в Главе 3, импульсные стрелки складываются в ходе реакции. Это означает, что число

% Страница 49: Продолжение раздела 4.4 - Рассмотрения теории информации
% Конрад Цузе "Rechnender Raum"

мест в импульсной стрелке новой результирующей частицы должно быть больше, чем число мест в реагирующей частице. Если мы исключим длину стрелки 0 для простоты и предположим, что стрелка реагирующей частицы может быть представлена двоичными разрядами, то стрелка результирующей частицы должна быть представлена 4 двоичными разрядами. До реакции у нас есть 2 частицы, каждая из которых имеет содержание информации $2 \times 3$ бит (всего 12 бит). После реакции у нас есть частица с информационной емкостью только $2 \times 4 = 8$ бит. Во время реакции мы потеряли 4 бита информации.

В этом процессе мы позволили стрелке результирующей частицы быть представленной большим числом мест. Это уже само по себе означает допущение нового типа частицы. Если это не допускается, должно быть найдено правило, которое вступает в действие всякий раз, когда допустимое число мест превышается в процессе сложения. Если мы просто предположим, что максимальное значение не может быть превышено, то последовательные реакции приводят после определенного периода времени к результату того, что нам остаются частицы с абсолютными максимальными импульсными стрелками.

Выбранные здесь примеры для цифровых частиц все еще гораздо слишком просты, чтобы быть строго связанными с физическими процессами. На самом деле, мы никогда не сталкиваемся в природе с ситуацией, в которой частицы одного типа реагируют друг с другом, не говоря уже о результате, что две такие частицы реагируют, чтобы дать частицу более высокого типа. Сохранение энергии, импульса, спина, заряда и т.д. сохраняется для элементарных частиц в физике. Только когда модели цифровых частиц находятся в нашем распоряжении, с помощью которых члены могут быть представлены, становятся возможны сравнительные наблюдения с элементарными частицами в физике и их реакциями.

Это вопрос очевидного интереса, является ли сохранение различных величин, указанных в соответственно сконструированных цифровых частицах, связанным с соответствующим сохранением информации. Проблема становится еще более сложной, когда также рассматриваются поля.

Автор может только поставить вопрос без предложения ответа на него. Может быть, вопрос не столь ужасно важен. Так или иначе, вопрос сводится к проблеме «конфигурации», которая, как известно, крайне трудна для математической обработки.

Здесь мы приходим в прямой контакт с одной из трудностей теории информации. При передаче новостей, наибольшее возможное содержание информации получается, когда вероятность отдельных сигналов распределена как можно более равномерно. Эта ситуация упоминается как максимальная энтропия информации. Легко возможно рассмотреть это таким образом, что каждая возможность соотнесения ранее полученных новостей со следующим символом должна с необходимостью уменьшить содержание информации, что ограничивает посредством связанной избыточности свободу выбора символов (новостей, кон-

% Страница 50: Завершение раздела 4.4 - Рассмотрения теории информации
% Конрад Цузе "Rechnender Raum"

фигурации).

Этот конфликт между требованием максимального содержания информации и одновременным требованием надежности передачи (минимум ошибок) затрагивает самую сердцевину теории информации.

Если физические правила должны быть предполагаемы детерминированными, то все процессы в физическом мире должны быть полностью определены начальными условиями. Это означает, что информационное содержание начального состояния кодирует информацию всех будущих состояний. В отношении информации, это является источником беспокойства. Это означает, что для вселенной, постигаемой целиком, информационное содержание начального состояния должно быть чрезвычайно велико.

На практике мы имеем только неполное знание начального состояния, и в результате процесс становится для нас вероятностным. Что здесь происходит, однако, это несколько философское по своей природе, и это касается границы, на которой детерминированная система становится вероятностной для нас, потому что нам не хватает информации. В области обработки информации в живых организмах, этот вопрос имеет значение, хотя не в контексте нашего настоящего рассмотрения.

Если мы полностью приняли концепцию, что космос может рассматриваться как клеточный автомат, то это имело бы следствия в отношении самого фундаментального отношения между энергией и материей. В инженерии информационной обработки, энергия используется для целей информационной обработки. Однако энергия может быть рассмотрена как носитель информации. Сам процесс передачи энергии может быть рассмотрен как процесс передачи информации. В этом свете материал и энергия представляют собой различные проявления одного основного явления - информации. Естественно, вещество кажется проявлением информационного содержания через механизмы изоляции и интеграции. Это является спекулятивным, безусловно, но такого рода спекуляция кажется оправданной в этом контексте.

Таким образом, результаты, которые автор попытался представить, вы видите как заключение, как попытку связать понятия современной физики с понятиями из области теории вычислений и автоматов. Концепция дискретизации представляет определённо новый подход к формулировке физических процессов. Какое практическое значение, если таковое имеется, могут иметь эти соображения, остаётся открытым вопросом. Однако по крайней мере можно указать, что соответствующие результаты показывают определённое логическое соответствие и не являются категорически исключёнными современными физическими теориями.


% Страницы 51-60
% Перевод страницы 51 "Rechnender Raum" Конрада Цузе
% Раздел 4.4 и начало раздела 4.5

содержание которой можно уже предсказать, не имеет информационного содержания). Всякая конфигурация по необходимости представляет через свои правила ограничение возможных средств представления и тем самым уменьшает информационное содержание. Сохранение информации и сохранение конфигурации в определённой степени противоречивы.

Вопрос о том, можно ли интерпретировать проверенные в физике термины (энергия, действующий квант, элементарный заряд, масса и т.\,д.) терминами информационной теории или обработки информации, пока ещё не может быть решен.

В модели клеточного автомата, построенной так, чтобы в нём происходили процессы, которые можно связать с перечисленными физическими величинами, эти величины должны быть представлены конструкцией схем; то есть величинами, представленными в схемах.

Ещё более важным, чем понятие информационного содержания, является понятие информационного обмена. Из принципов схемы вытекает нечто динамическое, а не нечто статическое. Возможно, это можно было бы назвать сохранением событий или усложнением событий (доктор Реше предложил идею <<сохранения сложности>>, хотя в другом контексте). Рассматриваемый таким образом, процесс сдвига приобретает дополнительное значение. Если действующему кванту приписать размерность <<процесс сдвига>>, то для энергии мы получим размерность <<процесс сдвига в единицу времени>>. Принцип сохранения энергии можно тогда интерпретировать как принцип сохранения событий. Термин <<действующий квант>> уже указывает на тесную связь с эффектами сдвигового типа, а именно с процессом сдвига. Представление энергии как <<события>> делает связь между энергией и частотой более легко понятной. Эти мысли пока что только простые предположения. Их цель --- стимулировать применение средств наблюдения автоматной теории в физике.

Рассмотрение принципа неопределённости Гейзенберга с точки зрения информационной теории следует далее. Если имеется запоминающая ёмкость в $m$ бит для цифрового представления двух величин $A$ и $B$, то мы имеем свободу распределить две величины различным числом мест и даже различающимися точностями по числу мест. Если величине $A$ отведено $n$ мест, то $B$ имеет $m - n$ мест. Ошибка в $A$ имеет порядок величины $2^{-n}$, ошибка в $B$ --- порядок величины $2^{-(m-n)}$. Произведение обеих ошибок даёт постоянную $2^{-m}$.

Возможно предположить, что обе сопряжённые величины $A$ и $B$ представляются не непосредственно паттерном цифровых частиц, а представляют выведённые величины, которые появляются только в определённых процессах. Ограничения на информационное содержание цифровых частиц не позволяют обеим величинам быть представленными с максимально возможной точностью. В случае цифровых частиц, даже если одна из величин полностью неопределена, другая

% Страница 52 - Глава 4: Общие рассмотрения

\section{О детерминизме и причинности}
\label{sec:determinism-causality}

Вопрос о детерминизме и причинности тесно связан с наблюдениями из теории информации и теории автоматов. Выражение <<причинность>> строго не применяется в литературе. В дальнейшем оно всегда используется для обозначения того, что обычно называют <<детерминизмом>>, а именно определение следующего состояния замкнутой системы как функции предыдущего состояния. Всю вселенную можно рассматривать как замкнутую систему, в той мере, в какой учитываются необходимые следствия этого допущения.

Теория автоматов оперирует концепцией состояния автомата. Конечные автоматы могут иметь ограниченное число состояний. При отсутствии входного сигнала результирующее состояние вытекает из предыдущего из-за алгоритма, лежащего в основе автомата. Поскольку теория автоматов оперирует абстрактными понятиями, переход из одного состояния в другое происходит в теории без промежуточных шагов. Теория автоматов не ставит вопрос о том, как именно этот переход происходит в функционирующем автомате. Она занимается исключительно тем фактом, что, например, триггер переходит из одного состояния в другое в течение определённого времени — периода импульса. Технологический анализ процесса переключения, который возможен, выходит за пределы рассмотрения теории автоматов, если только он не относится к пониманию таких деталей.

Некоторые физики, например Артур Марх, полагают, что прямой переход атома из одного устойчивого состояния в другое трудно согласовать с принципом причинности. Он понимает идею причинности таким образом, что переход из одной замкнутой системы в другую требует непрерывного процесса. Это толкование едва ли может устоять перед теоретико-автоматным рассмотрением физических процессов. Нельзя предположить, что эта идея основана на действительности. Мышление в целых числах и в дискретных состояниях требует мыслительного процесса с неконтинуальными переходами, в котором закон причинности формулируется в алгоритмах. Работа с дискретными состояниями и квантификацией как таковой не обязательно требует отказа от причинного способа наблюдения.

% Страница 53 - Глава 4: Общие рассмотрения

Это непрерывный переход в смысле теории автоматов должен быть отличен от мысли о непрерывном переходе между отдельными устойчивыми состояниями атома. Поскольку мы не способны анализировать такой процесс переходов экспериментально, все теории по этому вопросу относятся к области спекулятивного. В теоретико-автоматном смысле естественная цель состоит в создании моделей, которые позволяют отследить эти переходы поотдельности и позволяют объяснить излучение или поглощение фотонов в соответствующем процессе. Мы не можем предсказать, будет ли эта цель когда-либо достигнута. Часто высказываемое мнение о том, что такие переходы принципиально не поддаются анализу и такие эксперименты следует подчинить более плодотворным начинаниям, может, однако, быть опровергнуто. Квантовая физика предоставляет статистические законы для таких процессов, посредством которых отдельные детерминированные определения замещаются статистическими определениями. Этот вопрос будет рассмотрен дальше в связи с обсуждением вероятности.

Важно выяснить, действительна ли детерминированность в обоих направлениях времени; иными словами, являются ли более поздние состояния системы ясными функциями предыдущих состояний и наоборот. Классическая модель механики идеально удовлетворяет требованию симметрии относительно времени. Статистическая квантовая механика вводит идею вероятности и наблюдает отклонение от временной симметрии в увеличении энтропии. В общем, конечные автоматы следуют законам, определённым только в положительном направлении. Алгоритм устанавливает только то, какое состояние вытекает из данного, но не обратное. Возможно построить автоматы, в которых предыдущее состояние определяется следующим за ним, но это не обязательно подразумевает симметрию во времени. Рассмотрение компьютеров может это прояснить. Компьютер, при условии безупречной работы, детерминирован в положительном направлении времени. В общем, вычислительные процессы необратимы, что видно из рассмотрения основных операций, на которых основаны все высшие вычисления, и которые необратимы (например, a ∨b ⇒ c). Вычислитель является одним примером вычислительной машины, которая эффективно детерминирована в обоих направлениях, поскольку она считает вперёд в одном направлении времени и назад в другом, в той мере, в какой мы рассматриваем только таблицы состояний и не анализируем процессы поотдельности.

Различные характерные типы работы автономного автомата уже были рассмотрены в разделе 4.4 в связи с Рис.~68. Тип 68б соответствовал бы автомату, детерминированному в обоих направлениях, как упомянутый вычислитель.

% Страница 54 - Глава 4: Общие рассмотрения

Тем не менее, остаётся различие: в положительном направлении времени правило, согласно которому следующее состояние связано с предыдущим, явно дано алгоритмом. В отрицательном направлении времени существует единственная корреляция, конечно, но эта корреляция дана только неявно; то есть, она не может быть непосредственно вычислена без дополнительного знания. Это различие не ясно видно в диаграммах, соответствующих Рис.~68, и в таблице состояний, соответствующей Рис.~4. В любом случае, такой тип представления возможен только для очень простых автоматов и служит больше для начальных экспериментов, чем для практических определений процесса работы автомата. Фактическое правило образования следующего состояния из предыдущего дано схемами автомата. Мы можем сказать, что автономный автомат детерминирован в положительном направлении времени и что в специальных случаях отрицательного направления времени существует <<псевдодетерминированность>>.

Взаимоотношения цифровых частиц подобны в случаях, рассмотренных в Главе 4.4. Пока такая частица следует своему пути независимо от внешних влияний, происходит единственная последовательность состояний. Как только мы рассмотрим последовательность двух частиц, условия немедленно становятся другими. В этом случае примеры из Главы 3, Рис.~42-66 относятся к необратимым процессам. Основное правило сдвига регулирует процессы взаимодействия частиц. Нет никакого стимула для частицы разделиться на две частицы в какой-то момент времени. Это утверждение делает только одно утверждение о моделях, использованных в Главе 3. Вопрос о том, возможно ли построить пригодные модели цифровых частиц, которые не имели бы такой характеристики, трудно ответить. Это то же самое проблем, с которой сталкивается физик при распаде элементарных частиц или атомных ядер. Современное состояние теоретической физики таково, что мы можем дать только вероятностные законы для таких процессов. В модели, которая следует заранее определённому процессу и исключает рабочие элементы, в соответствии с вероятностными законами, есть только два способа решения:

(a) цифровая модель построена таким образом, что она содержит своего рода часы, которые разрешают процесс, когда достигнуто определённое состояние;

(b) влияние окружающей среды (например, полей, через которые движется цифровая частица) принимается во внимание. В процессе прохождения через свои различные фазы частица может пройти через критические состояния, в которых влияние окружающей среды (частота и т. д.) вызывает разделение частицы.

Современное состояние физических теорий не позволяет делать выводы о физических законах из этих возможностей цифровых моделей. То, что уже было сказано о переходе из одного атомного состояния в другое, одинаково актуально и здесь: ни один эксперимент не позволяет заглянуть за кулисы, и все теории по сути дела спекулятивны.

% Страница 55 - Глава 4: Общие рассмотрения

Однако удалось определить определённую зависимость радиоактивности от высоких температур, которая соответствует предположению о критических ситуациях, находящихся под влиянием окружающей среды.

Один результат важен в любом случае: предположение о действительности детерминизма только в положительном направлении времени никоим образом не влияется растворением физических законов в законы вероятности. Аналогично, увеличение энтропии не обязательно связано с этим вопросом. С точки зрения теории автоматов, каждый из этих вопросов принимает другой смысл. Энтропия может быть объяснена в цифровой модели, функция которой строго определена.

Рассмотрим классическую модель физики с этой точки зрения. Как уже упоминалось, действительность детерминизма, особенно в обоих направлениях времени, требует абсолютной точности отдельных процессов. Едва ли можно предположить, что серьёзные рассмотрения крайне важного значения этого допущения в отношении теории информации были предприняты. Такая модель требует бесконечно тонкой структуры пространственно-временных отношений. Бесконечное содержание информации требуется для неограниченного пространственно-временного элемента. Практически невозможно моделировать такую модель на компьютерах из-за необходимости бесконечного числа позиций. Источники ошибок соответственно велики из-за чрезвычайно большого числа столкновений между молекулами газа, и эти ошибки быстро приводят к отклонениям от теоретических процессов. Это означает, что чем лучше приближается правило причинности в обратном направлении времени, тем больше вычислений мы должны быть готовы произвести в нашей модели. Это приводит к результату, что моделирования универсальных систем с причинностью, функционирующей в обоих направлениях времени, относятся к категории <<неразрешимых>> проблем.

Конечно, можно сказать, что это справедливо только для вычислительных моделей моделирования. Но этот результат должен побудить нас пересмотреть этот вопрос. Имеем ли мы право предполагать модель природы, для которой не существует вычислимого моделирования?

С этой точки зрения представляется, что часто выдвигаемый аргумент о детерминизме в обоих направлениях времени должен быть фундаментально переэкзаменирован. Вопрос о временной симметрии физических законов часто обсуждается в связи с отражающими характеристиками пространства. Наблюдения из теории автоматов могут иметь значительную ценность в продвижении этого обсуждения.

% Страница 56 - Глава 4: Общие рассмотрения

\subsection{О вероятности}

Проблема детерминизма в современной физике тесно связана с законами вероятности. Наблюдение из теории автоматов может быть вставлено здесь. Конечно, возможно строить математические системы, такие как матричная механика и волновая механика, в которых значения вероятности играют значительную роль. Теоретик автоматов может ввести идею вероятности в свои теории и может установить последовательное состояние, зависящее от значений вероятности. До этого момента процесс является простой математической игрой на бумаге. Это становится критичным, когда мы пытаемся построить готовые формы таких механизмов, которые работают в соответствии с законами вероятности. Такие расчёты выполнялись в наших вычислительных автоматах с значительным успехом уже некоторое время (метод Монте-Карло). Элемент случайности вводится в расчёт в виде <<значений случайности>>. Генерация этих значений случайности — решающая проблема. Есть два способа это осуществить.

(a) Значения генерируются путём моделирования метода костей и тех типов числовых рядов, в которых не существует никакой зависимости между числами. Такой числовой ряд может быть разработан из расчёта иррациональных чисел (например, π). В действительности, этот процесс строго детерминирован. Тем не менее, мы говорим о значениях псевдослучайности. Этот процесс совершенно достаточен, когда правило генерации для таких значений случайности тщательно выбрано.

(b) Механизм берётся из природы, который либо настолько сложен, что его нельзя показать регулярным, либо о котором можно сказать, что в соответствии с действительными законами физики он предоставляет <<реальные>> значения вероятности. Механизм костей принадлежит к первому сорту, где причинные правила играют роль, но в случае достаточно тщательно построенной кости можно показать равную вероятность для каждого случая. То же самое верно для всех игр случая (рулетка и т. д.). В другом случае мы полагаемся на тот факт, что, например, радиоактивность определённого материала подчиняется строгим законам вероятности. Значительно ли, что вероятностный процесс в действительности детерминирован в этих атомах, не имеет значения, так как опыт показывает, что в любом случае законы вероятности могут быть предположены без ведения к неправильным результатам. В этом случае вычисляющий автомат рассматривает значения вероятности в известной степени как внешние входные значения. Остаётся верным, однако, что реальные значения вероятности едва ли возможны в технических автоматах.

% Страница 57 - Глава 4: Общие рассмотрения

Необходимо также помнить, что выбор алгоритма для создания значений псевдослучайности имеет высокое значение в случае (a). Это означает, что возможны только те выборы из диапазона основных числовых рядов, которые следуют друг за другом максимально нерегулярно и которые имеют максимально равномерно возможное распределение вероятности. Это означает, что более длинные ряды одного и того же числа и ряды чисел с одинаковым разделением (1, 2, 3) должны быть исключены, хотя эти ряды в реальных рядах значений случайности так же вероятны или маловероятны, как и любой другой числовой ряд.

Конечно, мы можем задать чисто спекулятивный вопрос, допустимы ли истинные вероятностные законы к теоретико-автоматному наблюдению физических процессов. Этот вопрос является философским и отмечен здесь только без ответа.

\subsection{Представление интенсивности}

Представление интенсивности напряженности полей и других численных величин в клеточных автоматах должно быть специально рассмотрено. По этой причине здесь рассматриваются несколько основных возможностей.

\begin{wrapfigure}{r}{0.3\textwidth}
  \includegraphics[width=\linewidth]{images/page_057_img_01.png}
  \caption{Представление интенсивности в двумерной сетке (Рис.~69)}
  \label{fig:69}
\end{wrapfigure}


На Рис.~\ref{fig:69} показана двумерная сетка, в которой отдельные узлы сетки заняты элементарными логическими значениями; например, значения да-нет. Если мы приписываем этим значениям числа 0 и 1, то статистическое распределение значений 1 представляет шкалу для напряженности поля. Такой тип представления может достичь немного, конечно, если необходимо учитывать много порядков величины плотности. Как уже упоминалось, отношение электростатических взаимодействий к гравитационным взаимодействиям находится в порядке 10^40 : 1. Если бы мы захотели представить эти различия интенсивности в трёхмерном пространстве, соответствующему Рис.~\ref{fig:69}, используя значения да-нет, потребовался бы куб с длиной стороны приблизительно 10^13 сеточных единиц. Это представляет только нижний предел, так как в действительности напряженности полей могут отличаться на ещё большие порядки величины. Если мы возьмём сетку с элементарной длиной 10^{-13} см, принятой физиками, это означало бы, что согласно этим расчётам потребовалось бы пространство в несколько кубических сантиметров, чтобы представить напряженность поля.

Этот тип модели не может быть очень полезным, совершенно независимо от того, что чрезвычайно трудно установить законы для стабильных цифровых частиц с таким типом статистического распределения.

% Страница 58 - Глава 4: Общие рассмотрения

Гораздо более рациональный метод предлагается принципом значений мест. Это не приводит к идее построить вычисляющие автоматы в соответствии с принципом Рис.~69. На Рис.~70 показано идеальное устройство суммирующей машины, состоящей из соседних ячеек, среди которых виден иерархический порядок. Отдельные ячейки координированы с числами разного значения. Это отражается в односторонней конструкции процесса передачи u_0 -- u_6.

На Рис.~71 показана передача этого мыслительного процесса линейному клеточному автомату. Каждая ячейка объединена с полной суммирующей машиной. Каждая ячейка C_i подразделяется на отдельные этапы сложения A_{0...5}. При конструировании такой системы сдвига необходимо помнить, что передачи между уровнями внутри ячейки должны быть координированы по времени с передачей информации между отдельными ячейками.

\begin{figure}[htbp]
  \centering
  \includegraphics[width=0.6\textwidth]{images/page_058_img_01.png}
  \caption{Устройство суммирующей машины с иерархическим порядком}
  \label{fig:70}
\end{figure}

\begin{figure}[htbp]
  \centering
  \includegraphics[width=0.6\textwidth]{images/page_058_img_02.png}
  \caption{Передача принципа к линейному клеточному автомату}
  \label{fig:71}
\end{figure}


Этот принцип относительно легко применить на практике для одномерных и двумерных клеточных автоматов. Теоретически его можно применять к трёхмерным и более высокомерным автоматам без каких-либо изменений. В дополнение к измерениям, которые соответствуют топологическому устройству соседних ячеек (пространственное измерение), существует также измерение уровня. Это воображаемо только в трёхмерном пространстве и должно быть конструктивно встроено (спроецировано) в трёхмерное пространство.

Можно задать дальнейший вопрос: может ли в симметрично построенном клеточном автомате быть введён иерархический порядок путём способа занятости? На Рис.~72 демонстрируется принцип. Отдельные ячейки могут содержать, например, отдельные этапы сложения и не могут принимать многозначные числа. Они распределены между несколькими соседними ячейками в соответствии с принципом значения мест. Трудность возникает в том, что такое устройство имеет природу занятости. Если концепция применяется к многомерному автомату, легко видно, что возникают крупные осложнения.

% Страница 59 - Глава 4: Общие рассмотрения

Клеточные автоматы предоставляют элегантное решение, когда каждая ячейка содержит полную вычислительную систему, как символически представлено на Рис.~73. Эти отдельные вычислительные системы содержат как элементы обработки информации, так и элементы хранения информации.

\begin{wrapfigure}{r}{0.4\textwidth}
  \includegraphics[width=\linewidth]{images/page_059_img_01.png}
  \caption{Клеточный автомат с полной вычислительной системой в каждой ячейке}
  \label{fig:72}
\end{wrapfigure}


Сетевой автомат, представленный на Рис.~74, является дальнейшим развитием клеточного автомата, соответствующего Рис.~73. Отдельные ячейки здесь отвечают только за обработку информации. Ветвящиеся линии B соединяют отдельные ячейки и служат как для передачи информации, так и для хранения информации. Отдельные ячейки могут состоять из одноместных суммирующих блоков в соответствии с принципом серии, действительным для вычислительных машин. Предварительные исследования автора показали, что этот тип автомата весьма успешен, в частности при решении численных задач, а также при моделировании физических процессов. Более специальное рассмотрение будет предметом другой статьи.

\begin{wrapfigure}{l}{0.4\textwidth}
  \includegraphics[width=\linewidth]{images/page_059_img_02.png}
  \caption{Клеточный автомат с полной вычислительной системой}
  \label{fig:73}
\end{wrapfigure}


\begin{wrapfigure}{r}{0.4\textwidth}
  \includegraphics[width=\linewidth]{images/page_059_img_03.png}
  \caption{Сетевой автомат с элементами обработки информации}
  \label{fig:74}
\end{wrapfigure}


% Страница 60 - Глава 5: ЗАКЛЮЧЕНИЯ

\chapter{Заключения}

Хотя эти наблюдения не приводят к новым легко понимаемым решениям, тем не менее может быть продемонстрировано, что предложенные методы открыли несколько новых перспектив, достойных продолжения. Включение концепций информации и теории автоматов в физические наблюдения станет ещё более критичным, так как всё более будут использоваться целые числа, дискретные состояния и тому подобное.

В следующей таблице предпринимается попытка связать различные возможные концептуализации:

\begin{center}
\begin{tabular}{|l|l|l|}
\hline
\textbf{КЛАССИЧЕСКАЯ ФИЗИКА} & \textbf{КВАНТОВАЯ МЕХАНИКА} & \textbf{ВЫЧИСЛЯЮЩЕЕ ПРОСТРАНСТВО} \\
\hline
Механика точки & Волновая механика & Теория автоматов \\
\hline
Алгебра счётчика & Частицы & Волна-частица, состояние счётчика, цифровая частица \\
\hline
Аналоговая & Гибридная & Цифровая \\
\hline
Анализ & Дифференциальные уравнения & Разностные уравнения и логические операции \\
\hline
Все значения непрерывны & Ряд значений квантован & Все значения имеют только дискретные значения \\
\hline
Без предельных значений & За исключением скорости света, нет предельных значений & Минимальные и максимальные значения для каждой возможной величины \\
\hline
Бесконечно точны & Отношение вероятности & Ограничения точности вычисления \\
\hline
Причинность в обоих направлениях времени & Только статичная причинность, разделение на вероятности & Причинность только в положительном направлении времени; введение вероятностных членов возможно, но не необходимо \\
\hline
Классическая механика статистически аппроксимируется & Объяснимы ли пределы вероятности квантовой физики определёнными структурами вычисляющего пространства? & \\
\hline
Основаны на формулах & Основаны на счётчиках & \\
\hline
\end{tabular}
\end{center}

Ввиду перечисленных выше возможностей ясно, что есть несколько различных точек зрения возможны:

(1) <<Идеи вычисляющего пространства противоречат некоторым признанным концепциям современной физики (например, изотропии пространства); поэтому фундаментальный базис должен быть ложным>>.

(2) <<Законы вычисляющего пространства должны быть пересмотрены с целью устранения существующих противоречий>>.


% Страницы 61-69 (послесловие и заключительные разделы)
% Перевод страницы 61 из "Rechnender Raum" Конрада Цузе
% Послесловие: страница 61

\section*{Послесловие к <<Rechnender Raum>> Конрада Цузе}

\noindent
\textit{Adrian German\textsuperscript{1} и Hector Zenil\textsuperscript{2}}

\noindent
\textsuperscript{1}Школа информатики и вычислений,\\
Университет Индианы, Блумингтон, США

\noindent
\textsuperscript{2}Отделение компьютерных наук,\\
Университет Шеффилда, Великобритания

\vspace{0.3cm}

Существует много параллелей между интересами Цузе и Тьюринга. В середине 1930-х годов некоторые исследователи были заняты тем, что по сути являлось исследованием природы вычисления и попыткой выяснить, было ли бы возможно построить вычислительную машину. Отчасти это было следствием программы Гильберта, но несомненно также было обусловлено определённой цепью исторических событий. Как указал Рауль Рохас\textsuperscript{1}, люди начали думать о компьютерах именно тогда, когда настало время их строить. Конечно, были Шёнфинкель (SKI комбинаторы), Чёрч ($\lambda$-исчисление), Пост (системы тегов), Клини (рекурсивные функции), Тьюринг (a-машины), и некоторые другие.

Возможно, основное различие между всеми остальными подходами и подходом Цузе заключается в том, что Цузе был инженером-строителем, стремящимся решать конкретные проблемы, и поэтому его подход был по сути чисто практическим. Таким образом, цель Цузе с самого начала была в построении конкретной, механической реализации вычисления. Подход Тьюринга находился на полпути между чистой абстракцией и практической реализацией. Этот факт один может объяснить, почему работа Тьюринга в итоге была более видна, чем работы других. Подход Цузе, будучи ответом инженера на вопрос о природе вычисления, принял форму действительной машины\textsuperscript{2}.

Цузе, возможно, не понимал, что существует фундаментальная концепция, лежащая в основе вопроса, который все эти люди задавали и в итоге пытались ответить (Цузе работал в относительной изоляции, в отличие от других, которые в основном знали друг о друге). Тьюринг в конечном счёте предоставил наиболее близкий ответ на вопрос через свою концепцию универсальности вычисления, основополагающее понятие информатики. Парадоксально, но сегодняшние цифровые компьютеры в некоторых отношениях могут быть более подобны машинам Цузе, чем идеализации Тьюринга, особенно

{\footnotesize
\noindent
\textsuperscript{1}На недавнем докладе \textit{Zuse and Turing in Context} в Кембридже, Великобритания, 18 февраля 2012 года.

\noindent
\textsuperscript{2}Самый полный источник информации~--- Интернет-архив Конрада Цузе, хранимый Раулем Рохасом, доступный онлайн по адресу \url{http://www.zib.de/zuse/home.php} (доступ апрель 2012). Его сын, Хорст Цузе, ведёт домашнюю страницу своего отца по адресу \url{http://www.horst-zuse.homepage.t-online.de/konrad-zuse.html} (доступ апрель 2012). Юрген Шмидхубер также ведёт веб-сайт, посвящённый Цузе, доступный по адресу \url{http://www.idsia.ch/~juergen/zuse.html} (доступ апрель 2012).
}

% Перевод страницы 62 из "Rechnender Raum" Конрада Цузе
% Послесловие: страница 62

потому, что Цузе должен был иметь дело с мельчайшими деталями действительного построения физической машины (например, стандарт IEEE для кодирования чисел с плавающей запятой почти идентичен представлению, использованному в Z1 и Z3 Цузе). Цузе никогда не думал об универсальности, как Тьюринг, но, как доказал Рохас, не без определённого творчества, Z1 и Z3 случайным образом (потому что это никогда не было целью Цузе, и он даже не сформулировал вопрос) оказались способны к универсальному вычислению\textsuperscript{4}. Цузе никогда не думал о том, как машина могла бы перейти в неограниченное вычисление (необходимое для универсальности), например, и если бы это произошло, как заставить её остановиться (Рохас предполагает, что потребовался бы механический/электрический трюк для произвольной остановки машин, с требуемым вычислением завершённым и каким-то образом закодированным среди других вычислений в выходе, если бы неограниченное вычисление было разрешено, например, путём зацикливания перфокарты).

После выпуска в 1935 году Цузе стал анализатором напряжений в авиастроительной компании Henschel, где работал над проблемами вибрации летательных аппаратов. Анализ напряжений требовал грозных расчётов, которые в то время могли быть выполнены только с большим трудом с помощью команд людей-вычислителей, оснащённых портативными счётными машинами\textsuperscript{5}. Цузе считал, что многие расчёты, которые он выполнял, могли бы быть просто автоматизированы. Получив грант на исследования в 1936 году от Reichsluftfahrtministerium (немецкого министерства авиации), он случайным образом построил свою первую вычислительную машину между 1936 и 1938 годами, а в 1938 году он строил свою вторую, используя телефонные реле в отличие от первой, которая была механической. Его Z3 был завершён в 1941 году, был полностью функционален и мог выполнять вычисления\textsuperscript{6}. Его Z1 уже был программируемым, несмотря на механическую конструкцию, используя перфоленту.

Его основная мотивация переключиться с механической на электронную форму была обусловлена заботой о надёжности~--- он хотел построить устойчивые и отказоустойчивые машины~--- но Z3, построенный с электронными реле, был логически эквивалентен Z1. Z1 и Z3 могли быть запрограммированы и могли выполнять все арифметические вычисления, могли загружать и сохранять информацию в двоичном коде и были способны к вычислениям с плавающей запятой (тогда как Mark I и ENIAC в США всё ещё представляли данные в десятичном коде, несмотря на то, что оба работали

\footnotesize
\noindent
\textsuperscript{4}См. <<The Architecture of Konrad Zuse's Early Computing Machines>> Рауля Рохаса в <<The First Computers -- History and Architecture>>, MIT Press, 2000, pp. 237-262, под редакцией R. Rojas и Ulf Hashagen.

\noindent
\textsuperscript{5}\url{http://www.independent.co.uk/news/people/obituary--konrad-zuse-1526795.html} (доступ апрель 2012).

\noindent
\textsuperscript{6}Онлайн видео, сделанное в Deutsches Museum München, показывает, как работал Z3, на примерах арифметического деления и вычисления квадратных корней: \url{http://www.youtube.com/watch?v=J98KVfeC8fU} (доступ апрель 2012)

% Перевод страницы 63 из "Rechnender Raum" Конрада Цузе
% Послесловие: страница 63

с двоичными логическими элементами и были неспособны к вычислениям с плавающей запятой).

Цузе решил использовать двоичную систему и металлические пластины, которые могли двигаться только в одном направлении, то есть могли только менять положение, точно так же, как это делают современные цифровые компьютеры на своём самом низком уровне работы (Цузе, казалось, верил, что механические устройства и вычисления на цифровой основе были более надёжны по сравнению, например, с вакуумными трубками, как предполагал его друг Helmut Schreyer).

\vspace{0.3cm}

\textbf{Рисунок 75:} Реплика первого механического компьютера, разработанного Конрадом Цузе~--- Z1, завершённого в 1938 году. Это был двоичный электромеханический калькулятор, который использовал логику Буля и двоичные числа с плавающей запятой. Фотография сделана H. Zenil в Deutsches Technikmuseum (<<Немецком музее технологии>>), Берлин.

\vspace{0.3cm}

Цузе и Тьюринг никогда не встречались, но они познакомились с работами друг друга. Цузе упоминает работу Тьюринга в своей автобиографии, и известно, что Тьюринг входил в программный комитет/комитет рецензентов по крайней мере одного коллоквиума, который посещал Цузе~--- но не Тьюринг~--- в Max-Planck-Gesellschaft в Гёттингене в 1947 году. Если бы Тьюринг присутствовал, они действительно встретились бы.

Но если Цузе не открыл концепцию универсального вычисления, он был заинтересован в другом очень глубоком вопросе~--- вопросе о природе природы: <<Является ли природа цифровой?>> Он склонялся к утвердительному ответу, и его идеи были опубликованы, согласно Хорсту Цузе (старший сын Конрада), в Nova Acta Leopoldina. Хорст родился именно тогда, когда Конрад впервые думал о Rechnender Raum (общий перевод на английский~--- <<Calculating Space>>, но фраза на его родном немецком языке несёт намного больше когнитивного веса, чем её простой английский эквивалент, в свете идей, рассматриваемых в работе Цузе: вычисление, вычисление природы, пространство

% Перевод страницы 64 из "Rechnender Raum" Конрада Цузе
% Послесловие: страница 64

\begin{wrapfigure}{r}{0.4\textwidth}
  \includegraphics[width=\linewidth]{images/page_064_img_01.png}
  \caption{Реплика первого механического компьютера Конрада Цузе Z1 (Рис.~75). Двоичный механический калькулятор с электрическим приводом, использовавший булеву логику и числа с плавающей запятой. Фото H.~Zenil, Deutsches Technikmuseum, Берлин.}
  \label{fig:75}
\end{wrapfigure}


и/или вселенная). Hector Zenil (HZ) встретил проф. Horst Zuse (профессора Technische Universität Berlin) осенью 2006 года во время конференционного ужина в Берлине. Тема конференции была именно <<Является ли Вселенная компьютером?>> (Ist das Universum ein Computer?) и проводилась в Deutsches Technikmuseum и организована в честь Года информатики (Informatik Jahr) в Германии\textsuperscript{7}.

Конрад Цузе, однако, признал проблемы, которые, вероятно, возникнут при попытке согласовать цифровой взгляд на вселенную с теориями физики, предполагающими работу в пространствах континуума. Но согласно Конраду Цузе, законы физики могли быть объяснены в терминах законов переключателей или реле (неудивительно, учитывая его опыт преобразования его машин из механической в электронную форму через использование реле), и рассматривал физические законы как вычислительные приближения, захваченные математическими моделями. Ясно из Rechnender Raum, что Цузе знал, что дифференциальные уравнения могут быть решены цифровыми системами и считал этот факт свидетельством в пользу цифровой теории.

Годы до того, как John von Neumann объяснил преимущества архитектуры компьютера, в которой процессор отделён от памяти, Цузе уже пришёл к тому же выводу. Будучи строителем компьютеров в 1930-х годах, Цузе работал как любитель полностью вне математического сообщества, в своё собственное время, по вечерам и в выходные, в гостиной дома своих родителей. Однако он получил некоторую финансовую помощь от местного производителя счётных машин. Он также убедил Helmut Schreyer, бывшего университетского однокурсника, работать с ним. На совет своего друга Schreyer Цузе перешёл от механического к электромеханическому, телефонному релейному аппарату.

В своей автобиографии\textsuperscript{8} Цузе пишет, что в 1939 году, когда началась война, он был призван в пехоту для службы на передовой. Он никогда не видел боевых действий в качестве солдата. Его военная служба должна была продолжаться шесть месяцев, <<шесть месяцев, во время которых у меня было много времени, чтобы размышлять об идеях, разработанных и отражённых в моих дневниковых заметках 1937 и 1938 годов>>. Он был освобождён от активной службы и уволен, чтобы он мог выполнять работу, непосредственно связанную с разработкой оружия, как конструктор в Специальном отделе F компании Henschel Aircraft, где были разработаны дистанционно управляемые летающие бомбы.

В 1941 году, вскоре после завершения Z3, Цузе вернулся к работе конструктором в авиастроении в Henschel, дневной работе, в то время как он начал компанию~--- Zuse Apparatebau (Zuse Apparatus Construction), для производства его машин.

\footnotesize
\noindent
\textsuperscript{7}HZ написал пост в блоге об этом, доступный онлайн по адресу \url{http://www.mathrix.org/liquid/archives/is-the-universe-a-computer}.

\noindent
\textsuperscript{8}<<The Computer -- My Life>>, опубликовано на немецком языке издательством Springer-Verlag в 1993 году и переведено на английский в 2010 году, в юбилей рождения Цузе.

% Перевод страницы 65 из "Rechnender Raum" Конрада Цузе
% Послесловие: страница 65

Когда Z3 стал работоспособным, это была первая в мире практическая автоматическая вычислительная машина, и в течение двух лет она оставалась единственной. Вторая машина, Z4, была быстро введена в эксплуатацию. Во время войны Z3 была продемонстрирована перед несколькими ведомствами, но она так и не была введена в повседневную эксплуатацию. В 1944 году Z3 была уничтожена в авиационном налёте, но была реконструирована в 1960 году и установлена в Deutsches Museum в Мюнхене.

Цузе и Schreyer, однако, должны были покинуть здание, где была расположена их компьютерная машина. По мере окончания войны Цузе отступил в Hinterstein, деревню на юго-востоке Германии, где родился его старший сын (Horst). Там он восстановил свой компьютер Z4 в сарае, и он стал первой в мире коммерческой операционной вычислительной машиной, сданной в аренду ETH Zürich (одному из двух университетов Swiss Federal Institutes of Technology).

Затем он начал работать в области, которая не требовала физических ресурсов~--- программирование компьютеров. Он разработал язык~--- Plankalkül (означающий <<формальная система планирования>> или <<исчисление программ>>; <<универсальный язык>> согласно Цузе, который сравнивал его с <<искусственным мозгом>>)~--- который предвосхитил некоторые концепции программирования, которые появились позже, и может быть рассмотрен как первый язык высокого уровня, хотя для него никогда не был написан ни компилятор, ни интерпретатор. В 1945 году, возможно с той же мотивацией, которая привела Тьюринга к шахматам, а именно тем фактом, что игра считалась олицетворением человеческого интеллекта и в то же время казалась весьма алгоритмичной, Цузе работал над алгоритмами игры в шахматы, сформулированными как подпрограммы на его Plankalkül. Годом ранее, в 1944 году, он организовал свою работу в диссертацию\textsuperscript{9}, которая никогда не была защищена формально. Название, которое он выбрал для своей работы, было <<Beginnings of a Theory of General Computing>> (<<Начала теории общего вычисления>>), пытаясь установить основания того, что сегодня обычно понимается как обработка информации: <<Вычисление (Rechnen)>>, писал он, <<означает в общем случае формирование новых данных из заданных данных согласно некоторому правилу>>. Концепция алгоритма позже заменила его концепцию Vorschrift (или правила). Его язык программирования, как и логика, проектирование и построение его вычислительных машин, были полностью его собственной работой, выполненной в изоляции от развития, происходившего в других местах.

{\footnotesize
\noindent
\textsuperscript{9}См. <<The Plankalkül of Konrad Zuse -- Revisited>> Friedrich L. Bauer в <<The First Computers -- History and Architecture>>, упомянуто ранее.
}

% Перевод страницы 66 из "Rechnender Raum" Конрада Цузе
% Послесловие: страница 66

Пока он был в Hinterstein, он написал трактат, озаглавленный <<Freedom and Causality in the Light of the Computing Machine>> (<<Свобода и Причинность в свете вычислительной машины>>). В своей автобиографии он пишет: <<Я думаю, что большинство исследователей, вовлечённых в развитие компьютера, в какой-то момент своей жизни так или иначе рассматривали вопрос об отношении между свободной волей человека и причинностью>>. Это должно было стать основным импульсом для работы, которая привела к переводу, представленному в этом томе:

\begin{quote}
<<Рассматривая причинность, мне внезапно пришло в голову, что вселенная могла бы быть представлена как гигантская вычислительная машина. Я имел в виду релейный калькулятор: релейные калькуляторы содержат цепочки реле. Когда реле срабатывает, импульс распространяется через всю цепь. Мне в голову пришла мысль, что это также должна быть форма распространения кванта света. Мысль прочно осела; на протяжении многих лет я развивал её в концепцию Rechnender Raum, или <<вычисляющей вселенной>>. Однако потребовалось ещё тридцать лет, прежде чем я смог правильно сформулировать эту идею.>>
\end{quote}

В 1967 году Цузе предположил, что сама вселенная работает на клеточном автомате или подобной вычислительной структуре, метафизической позиции, известной сегодня как цифровая физика, предметом которой занимался сам Ed Fredkin до того, как познакомился с работами Цузе. Возбужденный открытием этой работы, Fredkin пригласил Цузе в Cambridge, MA. Перевод Rechnender Raum, воспроизведённый здесь, из немецкой (опубликованной) версии идей Цузе, был фактически заказан во время работы Ed Fredkin в качестве директора Project MAC\textsuperscript{10} в MIT (лаборатории AI, которая была предшественницей нынешних лабораторий AI в MIT).

Более чем через двадцать лет после своего Rechnender Raum в автобиографии Цузе он написал:

\begin{quote}
<<В конечном анализе, концепция вычисляющей вселенной требует переосмысления идей, для которых физики ещё не готовы. Однако ясно, что прежние концепции достигли пределов своих возможностей; но никто не осмеливается перейти на принципиально новый путь. Однако при квантизации уже были предприняты предварительные шаги в направлении цифровизации физики; но только несколько физиков попытались думать вдоль линий этих новых категорий компьютерной науки. [\ldots] Это было иллюстрировано совершенно ясно на конференции по Physics of Computation, проведённой 6--8 мая 1981 года [в MIT]. Что было типично на этой конференции, так это то, что, хотя связь между
\end{quote}

{\footnotesize
\noindent
\textsuperscript{10}Ed Fredkin также является автором, внёсшим вклад в \textit{A Computable Universe: Understanding \& Exploring Nature as Computation}.
}

% Перевод страницы 67 из "Rechnender Raum" Конрада Цузе
% Послесловие: страница 67

\begin{quote}
физикой и информатикой и/или компьютерным оборудованием рассматривалась в деталях, физические возможности и пределы компьютерного оборудования всё ещё доминировали в обсуждениях. Более глубокий вопрос о том, в какой степени процессы в физике могут быть объяснены как компьютерные процессы, рассматривался только в незначительной степени на этой в остальном очень передовой конференции.>>
\end{quote}

Оригинал Rechnender Raum, похоже, был потерян. Насколько нам известно, перевод, заказанный Project MAC (предшественником нынешней MIT Computer Science and Artificial Intelligence Laboratory или CSAIL), никогда не был опубликован в журнале\textsuperscript{11}. Он воспроизведён здесь переведённым на современный \LaTeX, что потребовало значительной работы, несмотря на использование сначала методик OCR с Mathematica, чтобы избежать полного начала с нуля. Он публикуется в этом томе без изменений, за исключением, возможно, нескольких исправленных опечаток и перераспределения текста и изображений в соответствии с форматом книги. Материал одновременно устарелый и удивительно современный: <<Я предлагаю, что в информационно-теоретическом анализе объекты и элементарные размерности физики не должны быть дополнены концепцией информации, а скорее должны быть объяснены ею>>. Цузе всегда был осведомлён о гипотетической природе своего тезиса: <<Концепция вычисляющей вселенной всё ещё лишь гипотеза; ничего не было доказано. Однако я уверен, что эта идея может помочь раскрыть тайны природы>>.

Цузе ссылается на людей, более скептически настроенных к нам, на цитату из Freeman Dyson (<<Innovation in Physics>>, опубликовано в \textit{Scientific American}, Vol. 199, No. 3, (сентябрь 1958), pp. 74--82.): <<Несколько месяцев назад Werner Heisenberg и Wolfgang Pauli верили, что они сделали существенный шаг вперёд в направлении теории элементарных частиц. Pauli проходил через Нью-Йорк, и его убедили прочитать лекцию, объясняющую новые идеи аудитории, которая включала Niels Bohr. Pauli говорил час, а затем была общая дискуссия, во время которой его довольно резко критиковало младшее поколение. Наконец, Bohr был приглашен произнести речь, подытоживающую аргумент. <<Мы все согласны>>, сказал он, <<что ваша теория безумна. Вопрос, который разделяет нас, заключается в том, безумна ли она настолько, чтобы иметь шанс быть правильной. Моё собственное чувство состоит в том, что она недостаточно безумна.>>'

<<Воображение>>, говорил Цузе, <<это ключ ко всему прогрессу>>.

\vspace{0.3cm}

A. German и H. Zenil

Bloomington, IN, США и Sheffield, Великобритания

{\footnotesize
\noindent
\textsuperscript{11}Отсканированные копии краткой немецкой версии и перевода на английский, сопровождаемые дополнительным контекстным материалом, доступны онлайн на веб-сайте Schmidhuber <<Zuse's thesis>> по адресу \url{http://www.idsia.ch/~juergen/digitalphysics.html}. Немецкая версия также находится по адресу \url{http://www.zib.de/zuse/Inhalt/Texte/Chrono/60er/Pdf/76scan.pdf} (ссылки доступны апрель 2012)
}

% Перевод страницы 68 из "Rechnender Raum" Конрада Цузе
% Послесловие: страница 68

\begin{wrapfigure}{r}{0.4\textwidth}
  \includegraphics[width=\linewidth]{images/page_068_img_01.png}
  \caption{(Как) Природа вычисляет? Панельная дискуссия (Рис.~76)}
  \label{fig:76}
\end{wrapfigure}


Панельная дискуссия, показанная на Рис.~\ref{fig:76}, была организована A. German и H. Zenil в последний день конференции 2008 NKS Midwest Conference, с участием (в порядке): Greg Chaitin, Ed Fredkin, Rob de Ruyter, Anthony Leggett, Cristian Calude, Tommaso Toffoli и Stephen Wolfram, модерировали (слева направо) Gerardo Ortiz, George Johnson и Hector Zenil, в Университете Индианы, Блумингтон.

См. \url{http://www.cs.indiana.edu/~dgerman/2008midwestNKSconference/}.

\vspace{0.3cm}

физикой и компьютерной наукой и/или компьютерным оборудованием была рассмотрена в деталях, физические возможности и пределы компьютерного оборудования всё ещё доминировали в обсуждениях. Более глубокий вопрос~--- в какой степени процессы в физике могут быть объяснены как компьютерные процессы~--- рассматривался только в маргинальной степени на этой иначе очень передовой конференции.

\input{../chapters/translation_page_069.tex}

% Глоссарий терминов
\chapter*{Глоссарий основных терминов}
\addcontentsline{toc}{chapter}{Глоссарий основных терминов}

\begin{description}
\item[Cellular automaton] --- Клеточный автомат
\item[Computing space] --- Вычисляющее пространство
\item[Digital particles] --- Цифровые частицы
\item[Automaton theory] --- Теория автоматов
\item[Information processing] --- Обработка информации
\item[Differential equations] --- Дифференциальные уравнения
\item[Difference equations] --- Разностные уравнения
\item[Digital physics] --- Цифровая физика
\item[Plankalkül] --- Планкалкюль (первый язык программирования высокого уровня)
\item[Universal computation] --- Универсальность вычисления
\item[Digitalization] --- Дискретизация (термин оригинала: digitalization)
\item[Hybrid system] --- Гибридная система
\item[Field density] --- Плотность поля
\item[Quantization] --- Квантование
\item[Reversibility] --- Обратимость
\item[Entropy] --- Энтропия
\item[Determinism] --- Детерминизм
\item[Causality] --- Причинность
\end{description}

% Библиография
\begin{thebibliography}{99}
\bibitem{zuse1969} Zuse, K. (1969). \textit{Rechnender Raum}. Schriften zur Datenverarbeitung, Band 1. Braunschweig: Friedrich Vieweg \& Sohn.

\bibitem{fredkin1990} Fredkin, E. (1990). Digital Mechanics. \textit{Physica D: Nonlinear Phenomena}, 45(1-3), 254--270.

\bibitem{wolfram2002} Wolfram, S. (2002). \textit{A New Kind of Science}. Champaign, IL: Wolfram Media.

\bibitem{lloyd2006} Lloyd, S. (2006). \textit{Programming the Universe: A Quantum Computer Scientist Takes on the Cosmos}. New York: Alfred A. Knopf.

\bibitem{tegmark2014} Tegmark, M. (2014). \textit{Our Mathematical Universe: My Quest for the Ultimate Nature of Reality}. New York: Alfred A. Knopf.

\bibitem{deutsch1997} Deutsch, D. (1997). \textit{The Fabric of Reality}. London: Penguin Books.

\bibitem{zuse1970} Zuse, K. (1970). \textit{Calculating Space}. MIT Technical Translation AZT-70-164-GEMIT.

\bibitem{petri1962} Petri, C.A. (1962). \textit{Kommunikation mit Automaten}. Schriften des IIM Nr. 2, Institut für Instrumentelle Mathematik, Bonn.
\end{thebibliography}

\end{document}
