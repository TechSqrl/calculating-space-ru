% Глоссарий терминов перевода "Rechnender Raum" (Вычисляющее пространство)
% Конрад Цузе - Страницы 21-35

\documentclass[11pt]{article}
\usepackage[utf-8]{inputenc}
\usepackage[russian]{babel}
\usepackage{hyperref}
\usepackage{array}
\usepackage{booktabs}

\title{Глоссарий терминов \\ "Вычисляющее пространство" Конрада Цузе}
\author{Перевод страниц 21-35}
\date{22 ноября 2025}

\begin{document}

\maketitle

\tableofcontents

\newpage

\section{Основные понятия}

\subsection{Вычисляющее пространство (Computing Space)}

Фундаментальное понятие Цузе, обозначающее дискретную модель физического пространства, в которой:
\begin{itemize}
    \item Пространство состоит из элементов, организованных в правильную сетку
    \item Каждый элемент может находиться в одном из конечного числа состояний
    \item Состояния меняются дискретными временными шагами
    \item Переходы определяются локальными правилами взаимодействия
\end{itemize}

\textbf{Русский эквивалент}: вычисляющее пространство

\textbf{Альтернативные названия}: computationally space, cellular space

\subsection{Цифровая частица (Digital Particle)}

Стабильная, повторяющаяся конфигурация состояний в вычисляющем пространстве, которая:
\begin{itemize}
    \item Сохраняет свою форму при распространении
    \item Движется с постоянной скоростью
    \item Может взаимодействовать с другими частицами
    \item Обладает свойствами, аналогичными физическим частицам
\end{itemize}

\textbf{Русский эквивалент}: цифровая частица

\textbf{Примеры}: импульсы давления, волны в одномерных и двумерных системах

\subsection{Клеточный автомат (Cellular Automaton)}

Математическая модель дискретной системы, где:
\begin{itemize}
    \item Пространство разделено на регулярную сетку клеток
    \item Каждая клетка имеет конечный набор состояний
    \item Состояние клетки в следующий момент зависит от её текущего состояния и состояний соседних клеток
    \item Все клетки обновляются одновременно
\end{itemize}

\textbf{Русский эквивалент}: клеточный автомат

\textbf{Исторический контекст}: Цузе использовал эту концепцию за 20 лет до Конвея и его "Game of Life"

\section{Физико-математические термины}

\subsection{Дифференциальное уравнение (Differential Equation)}

Уравнение, описывающее изменение величины во времени и пространстве:
\[
k_0 \Delta_s p \Rightarrow \Delta_t v
\]

\textbf{Русский эквивалент}: дифференциальное уравнение

\textbf{В контексте Цузе}: Дифференциальные уравнения физики (например, уравнения гидродинамики) преобразуются в разностные для компьютерной реализации.

\subsection{Разностное уравнение (Difference Equation)}

Дискретный аналог дифференциального уравнения, использующий конечные разности:
\[
v - \Delta_s p \Rightarrow v
\]

\textbf{Русский эквивалент}: разностное уравнение

\textbf{Особенность}: Цузе подчёркивает, что разностные уравнения могут значительно отличаться от исходных дифференциальных уравнений при грубой дискретизации.

\subsection{Дискретизация (Digitalization)}

Процесс преобразования непрерывной системы в дискретную:
\begin{enumerate}
    \item Выбор шага дискретизации по пространству ($\Delta x$) и времени ($\Delta t$)
    \item Ограничение значений переменных дискретным набором (например, $-1, 0, +1$)
    \item Замена дифференциальных операторов разностными
\end{enumerate}

\textbf{Русский эквивалент}: дискретизация (в оригинале используется термин \emph{digitalization})

\textbf{Критическое замечание}: Цузе показывает, что грубая дискретизация может привести к качественно новому поведению (например, испусканию частиц вместо простого рассеяния).

\section{Понятия теории автоматов}

\subsection{Теория автоматов (Automaton Theory)}

Раздел информатики, изучающий абстрактные механические устройства и их вычислительные возможности.

\textbf{Русский эквивалент}: теория автоматов

\textbf{Применение у Цузе}: Вычисляющее пространство можно рассматривать как бесконечный клеточный автомат.

\subsection{Состояние (State)}

Конфигурация значений во всех элементах системы в данный момент времени.

\textbf{Русский эквивалент}: состояние

\textbf{В контексте}: Переход от одного состояния к следующему определяется детерминированным правилом.

\subsection{Таблица переходов (State Table / Transition Table)}

Описание того, как система переходит из одного состояния в другое.

\textbf{Русский эквивалент}: таблица переходов, таблица состояний

\textbf{Связь с матричной механикой}: Цузе отмечает аналогию между таблицами переходов и матрицами переходов квантовой механики.

\section{Физические и динамические понятия}

\subsection{Энтропия (Entropy)}

Мера беспорядка в системе.

\textbf{Русский эквивалент}: энтропия

\textbf{Ключевое наблюдение Цузе}:
\begin{itemize}
    \item В классической модели упорядоченное состояние приводит к беспорядку (возрастание энтропии)
    \item В вычислительной модели то же самое происходит благодаря ошибкам округления
    \item Это демонстрирует глубокую связь между детерминизмом и хаосом
\end{itemize}

\subsection{Импульс (Pulse / Wave Pulse)}

Возмущение, распространяющееся через пространство с определённой скоростью.

\textbf{Русский эквивалент}: импульс

\textbf{В модели Цузе}: Импульсы давления и скорости в газовом цилиндре служат примерами цифровых частиц.

\subsection{Волновой фронт (Wave Front)}

Геометрическое место точек с одинаковой фазой волны.

\textbf{Русский эквивалент}: волновой фронт

\textbf{Особенность в двумерной системе}: Скорость распространения зависит от направления.

\subsection{Процесс столкновения (Collision Process)}

Взаимодействие двух частиц при сближении.

\textбf{Русский эквивалент}: процесс столкновения

\textbf{Результаты в цифровой модели}:
\begin{itemize}
    \item При определённых фазовых состояниях частицы проходят друг через друга
    \item При других фазовых состояниях происходит упругое отталкивание
    \item Без точного знания внутренней структуры различить эти случаи невозможно
\end{itemize}

\section{Параметры и переменные}

\subsection{Давление (Pressure)}

Обозначается $p$, физическая величина в гидродинамике.

\textbf{В модели Цузе}: Фиксируется в дискретных точках сетки.

\subsection{Скорость (Velocity)}

Обозначается $v$ (или $v_x$, $v_y$ в двумерном случае).

\textбф{В модели Цузе}: Выражается в промежуточных точках между точками давления.

\subsection{Пространственная разность (Spatial Difference)}

Обозначается $\Delta_s p$ или просто $\Delta p$.

\textbf{Определение}: Разность значения в соседних пространственных точках.

\subsection{Временная разность (Temporal Difference)}

Обозначается $\Delta_t p$ или аналогично.

\textбf{Определение}: Разность значения в последовательные моменты времени.

\section{Системы и геометрии}

\subsection{Одномерное пространство (One-Dimensional Space)}

Простейший случай, когда действие происходит вдоль прямой линии.

\textbf{Примеры}:
\begin{itemize}
    \item Газ в цилиндре
    \item Волны на струне
\end{itemize}

\subsection{Двумерное пространство (Two-Dimensional Space)}

Действие происходит на плоскости, организованной в ортогональную сетку.

\textбф{Новые явления}:
\begin{itemize}
    \item Диагональное распространение
    \item Более сложные типы взаимодействия частиц
    \item Зависимость скорости от направления
\end{itemize}

\subsection{Ортогональная сетка (Orthogonal Grid)}

Прямоугольная сетка, состоящая из горизонтальных и вертикальных линий.

\textбф{В модели Цузе}: Каждой точке сетки $(x, y)$ присваиваются состояния.

\subsection{Гиперболические системы координат (Hyperbolic Coordinate Systems)}

Системы, применяемые при работе с искривлённым пространством.

\textбф{Перспектива}: Цузе предполагает, что свойства искривлённого пространства общей теории относительности могут быть приблизительно воспроизведены цифровыми методами.

\section{Операции и правила}

\subsection{Суперпозиция (Superposition)}

Принцип, согласно которому результат взаимодействия нескольких воздействий равен сумме результатов каждого воздействия отдельно.

\textбф{Русский эквивалент}: суперпозиция, принцип суперпозиции

\textбф{Следствие в цифровой модели}: При чистой суперпозиции невозможны реакции, соответствующие взаимодействиям элементарных частиц.

\subsection{Дизъюнкция (Disjunction)}

Логическая операция "ИЛИ", обозначаемая $\vee$.

\textбф{Русский эквивалент}: дизъюнкция

\textбф{В булевой алгебре}: Соответствует операции логического сложения.

\subsection{Булева алгебра (Boolean Algebra)}

Система логических операций (конъюнкция, дизъюнкция, отрицание).

\textбф{Русский эквивалент}: булева алгебра

\textбф{Применение}: Трёхзначные системы в модели Цузе преобразуются в двоичные операции.

\subsection{Ограничение значений (Value Limiting)}

Установление верхнего и нижнего пределов для переменных.

\textбф{Пример}: Ограничение значений только величинами $-1, 0, +1$.

\textбф{Эффект}: Вводит нелинейность и позволяет моделировать взаимодействия частиц.

\subsection{Округление (Rounding)}

Преобразование нецелого числа в целое.

\textбф{Варианты}:
\begin{itemize}
    \item Округление вверх (ceiling)
    \item Округление вниз (floor)
\end{itemize}

\textбф{В контексте}: Необходимо при использовании дробных коэффициентов (например, $\Delta p/3$).

\section{Технические понятия}

\subsection{Блок-диаграмма (Block Diagram)}

Схематическое представление системы, показывающее её компоненты и связи.

\textбф{Русский эквивалент}: блок-диаграмма

\textбф{На странице 30}: Показана блок-диаграмма вычисляющего пространства с регистрами и сдвигающими элементами.

\subsection{Регистр (Register)}

Устройство для хранения информации и выполнения операций сложения.

\textбф{Русский эквивалент}: регистр

\subsection{Импульсная работа (Pulse Timing)}

Синхронизация операций с помощью импульсных сигналов.

\textбф{Русский эквивалент}: импульсная работа, синхронизация

\textбф{Назначение}: Обеспечивает правильный порядок вычислений в системе.

\section{Математические операции}

\subsection{Дискретные системы (Discrete Systems)}

Системы, оперирующие с дискретными, а не непрерывными значениями.

\textбф{Русский эквивалент}: дискретные системы

\textбф{Контраст}: Непрерывная математика работает с пределами, дискретная сохраняет конечные шаги.

\subsection{Коэффициент (Coefficient)}

Числовой множитель в уравнении.

\textбф{В модели}: Коэффициенты $k$, $k_0$, $k_1$ содержат физические характеристики системы.

\subsection{Тернарная система (Ternary System)}

Система с основанием 3, использующая цифры $0, 1, 2$.

\textбф{Русский эквивалент}: тернарная система, трёхзначная система

\textбф{Преимущество над двоичной}: Позволяет точнее представлять дробные значения (например, $1/3$, $2/3$).

\textбф{Примечание}: Цузе рекомендует использовать тернарную систему для более гибких моделей.

\subsection{Двоичная система (Binary System)}

Система с основанием 2, использующая цифры $0, 1$.

\textбф{Русский эквивалент}: двоичная система

\section{Квантово-механические аналогии}

\subsection{Принцип неопределённости Гейзенберга (Heisenberg Uncertainty Principle)}

Фундаментальное ограничение квантовой механики: невозможно одновременно точно измерить положение и импульс частицы.

\textбф{Аналогия в модели Цузе}: В цифровой частице положение может быть определено для целого периода, но не для отдельных фаз, что напоминает неопределённость Гейзенберга.

\subsection{Квантовая механика (Quantum Mechanics)}

Раздел физики, описывающий поведение микроскопических систем.

\textбф{Связь с Цузе}:
\begin{itemize}
    \item Только определённые конфигурации цифровых частиц возможны (квантование)
    \item Вероятностный характер взаимодействий
    \item Волнево-корпускулярный дуализм
\end{itemize}

\subsection{Матричная механика (Matrix Mechanics)}

Формулировка квантовой механики Гейзенберга, использующая матрицы переходов.

\textбф{Аналогия}: Матрицы переходов соответствуют таблицам состояний автоматов Цузе.

\section{Физические явления}

\subsection{Рассеяние (Scattering)}

Отклонение частицы от прямолинейного пути при взаимодействии.

\textбф{Русский эквивалент}: рассеяние

\textбф{В модели}:
\begin{itemize}
    \item В классической модели требуется явное программирование рассеяния
    \item В вычислительной модели естественным образом возникает из-за ошибок округления
    \item Оба механизма приводят к возрастанию энтропии
\end{itemize}

\subsection{Консервация (Conservation)}

Сохранение определённых величин при эволюции системы.

\textбф{Примеры}:
\begin{itemize}
    \item Консервация импульса
    \item Консервация энергии
\end{itemize}

\textбф{Замечание Цузе}: В его модели консервируется импульс, а не энергия, что приводит к отклонениям от физических законов.

\section{Практические выводы}

Перевод этих 15 страниц обнаруживает, что Конрад Цузе, будучи пионером компьютеростроения, предложил глубокие идеи о дискретной природе физического пространства. Его работа демонстрирует, как грубая дискретизация непрерывных физических законов может привести к качественно новому поведению, сохраняя при этом существенные черты исходной физики.

Терминология, использованная Цузе, устоялась в современной информатике и физике, и её русские эквиваленты являются стандартными в соответствующих областях.

\end{document}
